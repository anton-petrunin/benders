\chapter{Серьёзные испытнаия}


\setlength{\epigraphwidth}{.80\textwidth}
\epigraph{Со временем разум поднимется на более высокий уровень знаний, но никогда не поймёт как он там оказался.
%There comes a time when the mind takes a higher plane of knowledge but can never prove how it got there.
}{--- Альберт Эйнштейн (1879---1955)}

Ну да, как будто бы предыдущие головоломки не были сложными, и вот ещё несколько непростых задач.
Но на самом деле, некоторые из них могут показаться попроще.
Ведь то, что ставит одного в тупик, бывает легко другому.


\subsection*{Торт-мороженое}

На столе стоит цилиндрический торт-мороженое покрытый шоколадной глазурью сверху.
Мы последовательно отрезаем от него дольки с углом $x$, где $x$ произвольно, при этом
каждый раз долька переворачивается и вставляется обратно в торт.
(См. рис. \ref{pic:tort}.)

Докажите, что после конечного числа таких операций вся глазурь снова окажется сверху!


\begin{figure}[htb!]
\centering
\includegraphics[scale=1]{pics/tort}
\caption{Разрез, переворот, и снова разрез.}
\label{pic:tort}
\end{figure}

\parit{Примечание.}
Я называю такие задачи \emph{вроде неверно расслышанные}.
Да-да, угол $x$ может быть иррациональным, в этом случае вы никогда не отрежете одинаковый кусок дважды.
На самом деле придётся отрезать довольно много долек, но всё же не бесконечное число.
К счастью, при этом торт-мороженое самовосстанавливаются.

\subsection*{Прыгание и перепрыгивание}

Лягушка прыгает по длинной цепочке листьев кувшинок;
на каждом листе она подбрасывает монету, чтобы решить, прыгнуть ли ей на второй лист вперёд или на назад предыдущий лист.
На какой доле листов она побывает?

\subsection*{Три тени кривой}

Существует ли простая замкнутая кривая в трёхмерном пространстве, все три проекции которой на координатные плоскости являются деревьями?

\parit{Примечание.}
Это означает, что тени кривой, в трёх координатных направлениях, не содержат циклов.
На рис. \ref{pic:proj1} изображена кривая, которая почти подходит: две её тени являются деревьями, но третья содержит (и даже является) циклом.

\begin{figure}[htb!]
\centering
\includegraphics[scale=1]{pics/proj1}
\caption{Замкнутая кривая у которой пара проекций деревья.}
\label{pic:proj1}
\end{figure}

\subsection*{Игроки и победители}

Тристану и Изольде предстоит оказаться в ситуации с очень ограниченной связью, в которой Тристан будет знать, какие две из 16 баскетбольных команд сыграли матч, а Изольда узнает, кто выиграл.
Сколько битов необходимо передать между Тристаном и Изольдой, чтобы первый узнал, кто победил?

\parit{Примечание.}
Эта задача по коммуникационной сложности.
Если бы Изольда знала, кто играл, а также кто выиграл, она могла бы отправить один бит, чтобы сообщить Тристану, выиграла ли (скажем) команда, идущая первой по алфавиту.
Без этой информации она могла бы просто отправить четыре бита, чтобы определить победившую команду, но можно ли обойтись меньшим числом битов?

\medskip

И снова задача по коммуникационной сложности, но посложней.

\subsection*{Чарли и обманщики}

Алиса и Боб знают да-нет ответы на все $n$ вопросов экзамена, который предстоит сдавать Чарли.
Чарли нужен ответ только на $k$-й вопрос, но ни Алиса, ни Боб не знают значение $k$;
вместо этого Алиса знает число $i$, а Боб --- число $j$, такие, что $k = i + j \pmod n$.

Если бы Алиса не могла передать никакой информации,
то Бобу пришлось бы отправить Чарли все ответы (всего $n$ битов), чтобы Чарли смог узнать нужный ему ответ.

Докажите, что если Чарли получит всего один бит от Алисы, то Бобу достаточно отправить Чарли всего $n/2$ бит.

\subsection*{Сближение точек на кривой}

Плоская кривая на рис. \ref{pic:s-curv} обладает следующими свойствами:
(1) расстояние между её конечными точками больше, чем у любой другой пары точек на кривой (мы используем обычное евклидово расстояние на плоскости),
и
(2) взяв по карандашу в обе руки, можно установить их кончики в концах кривой и двигать их вдоль кривой до встречи, так, чтобы в процессе расстояние между ними не увеличивалось.

Существует ли кривая с свойством (1), но без свойства (2)?

\begin{figure}[htb!]
\centering
\includegraphics[scale=1]{pics/s-curve}
\caption{Подозрительная кривая.}
\label{pic:s-curv}
\end{figure}

\subsection*{Суммы и произведения}

В шляпе лежат все целые числа, большие 1, но меньшие 100.
Из неё достают два числа.
Саманта узнаёт их сумму, а Порфирио --- их произведение.
Далее, Саманта говорит

--- Я знаю, что ты не знаешь этих чисел.

--- Теперь я их знаю --- отвечает Порфирио.

--- Теперь и я знаю --- говорит Саманта.

Что это были за числа?

\subsection*{Сложенный многоугольник}

Какова минимально возможная площадь простого многоугольника с нечётным числом сторон, каждая из которых имеет единичную длину?

\parit{Примечание.}
Простой многоугольник это простая замкнутая кривая, состоящая из конечного числа отрезков.
Он не обязан быть выпуклым; например, он может выглядеть как один из многоугольников на рис. \ref{pic:korona},
\ref{pic:wreath} и \ref{pic:treug}.
Очевидно, что у многоугольника на рисунке \ref{pic:treug} площадь не менее площади равностороннего треугольника с единичными сторонами, то есть $\sqrt{3}/4$;
на самом деле, несложно заметить, что и площадь короны на рис. \ref{pic:korona} ограничена тем же числом.
Но про венок на рис. \ref{pic:wreath} это уже не так очевидно.

Равносторонние многоугольники с чётным числом сторон явно могут иметь площадь на столько малую, на сколько хочется, например, схлопываясь в очень остроконечную звезду.
Но может быть для нечётноугольников нельзя добиться площади меньше $\sqrt{3}/4$?
Попробуйте это доказать или опровергнуть?

\begin{figure}[htb!]
\centering
\includegraphics[scale=1]{pics/korona}
\caption{Корона.}
\label{pic:korona}
\end{figure}

\begin{figure}[htb!]
\centering
\includegraphics[scale=1]{pics/wreath}
\caption{Венок.}
\label{pic:wreath}
\end{figure}

\begin{figure}[htb!]
\centering
\includegraphics[scale=1]{pics/treug}
\caption{Схлопнутый треугольник.}
\label{pic:treug}
\end{figure}
