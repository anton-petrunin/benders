\section*{Источники и решения}

\subsubsection*{Монеты на столе}

Эта интересная головоломка досталась мне от специалиста по информатике Гая Киндлера во время замечательного года проведённого нами в Институте перспективных исследований Принстона.

\begin{figure}[t!]
\centering
\includegraphics[scale=1]{pics/coin1}
\caption{Нельзя добавить новую монету без перекрытия со старыми.}
\label{pic:coin1}
\end{figure}

\begin{figure}[b!]
\centering
\includegraphics[scale=1]{pics/coin2}
\caption{Стол покрыт удвоенными монетами.}
\label{pic:coin2}
\end{figure}

Для начала заметим, что если у каждой из исходных монет удвоить радиус (скажем, с $1$ до $2$), то как видно на рисунках \ref{pic:coin1} и \ref{pic:coin2}, покроется весь стол.
Почему?
Ну, если точка $P$ не покрыта, то она лежит на расстоянии $2$ или более от центра любой монеты, так что  к исходной конфигурации можно добавить (маленькую) монету, с центром в $P$.

Дело будет сделано, если найти способ покрыть каждую большую монету четырьмя маленькими.
Однако это невозможно.

Тем не менее, нужным свойством обладает \emph{сам} прямоугольник --- его можно разбить на четыре уменьшенные копии самого себя.
Итак, давайте сожмём вдвое всю картинку (рис. \ref{pic:coin2}, где большие монеты покрывают весь стол) и воспользуемся четырьмя её копиями (как на рис. \ref{pic:coin3}).
Так мы покроем весь исходный стол!


\begin{figure}[t!]
\centering
\includegraphics[scale=1]{pics/coin3}
\caption{Четыре уменьшенные стола покрывают стол целиком.}
\label{pic:coin3}
\end{figure}

\medskip

Это красивое, но на первый взгляд грубое рассуждение даёт (как ни странно) наилучшую оценку --- замените $4$ на что-то меньшее, скажем, $3{,}99$, и утверждение перестанет быть верным.

Чтобы понять это, рассмотрим предельный случай, когда стол очень большой, а монет так много, что граничные эффекты пренебрежимо малы.
Заменим стол на пол ванной комнаты, покрытый мозаикой в виде пчелиных сот;
то есть каждая плитка это правильный шестиугольник диаметра (скажем) 2.
Поскольку каждую плитку можно разбить на шесть равносторонних треугольников со стороной 1 и, значит, площади $\sqrt{3}/4$, а значит сама плитка имеет площадь $6\times\sqrt{3}/4=3\times\sqrt{3}/2$.

\begin{figure}[t!]
\centering
\includegraphics[scale=1]{pics/coin4}
\caption{Пкрытие описанными кругами правильных шестиугольников.}
\label{pic:coin4}
\end{figure}

\begin{figure}[b!]
\centering
\includegraphics[scale=1]{pics/coin5}
\caption{Разряженная конфигурация.}
\label{pic:coin5}
\end{figure}

Весь пол можно покрыть, положив на каждую плитку монету, граничная окружность которой описана вокруг плитки (см. рис. \ref{pic:coin4}).

Тогда каждая монета имеет радиус $1$ и, следовательно, площадь~$\pi$.
Если площадь всего пола $A$ то, игнорируя граничные эффекты, общая площадь монет будет $\pi A/(3\sqrt{3}/2)\sim 1{,}2092\times A$.

Теперь давайте разберёмся, насколько разрежённо можно расположить монеты на полу, чтобы нельзя было добавить новую монету без перекрытия со старыми.
Воспользуемся той же плиткой, но на этот раз покроем только треть плиток (рис. \ref{pic:coin5}).
В середину каждой такой плитки положим по монете с радиусом чуть больше радиуса вписанной в шестиугольник окружности. 
Это не позволит добавить ещё монет.
Какова же площадь всех этих монет?

Ну, радиус монеты чуть больше высоты одного из шести равносторонних треугольников, составляющих шестиугольник — а именно, $\sqrt{3}/2$.
Следовательно, площадь монеты чуть превысит $\pi \z\times (\sqrt{3}/2)^2 \z= 3\pi/4$.

Отсюда следует, что общую площадь монет на полу можно сделать произвольно близкой к $(1/3) \times (3\pi/4) \times A/(3\sqrt{3}/2) = \pi A/(6 \sqrt{3}) \z\sim 0{,}3023 \times A$, а это ровно четверть того, что было раньше!

\medskip

Заметьте, что в результате мы доказали не только утверждение головоломки, но ещё два экстремальных свойства кругов на плоскости.
Первое говорит, что нет лучшего способа покрыть плоскость единичными кругами, чем описать плитки шестиугольной мозаики, как мы сделали выше.
Второе, что нет более разряженной конфигурации монет \emph{без} возможности добавить лишнюю, чем помещать на каждой третьей плитке круг, чуть больший, чем вписанный, опять же как мы проделали выше.

Если вам эти свойства очевидны, то подумайте о следующем ещё более очевидном утверждении: самая плотная упаковка единичных кругов на  плоскости получается из кругов, вписанных в каждый шестиугольник пчелиных сот.
Это было доказано только в 1972 году великим венгерским геометром Ласло Фейеш Тотом (1915---2005)!

\begin{addedbytheeditors}
Доказательства всех трёх последних утверждений приводятся в замечательной книжке Фейеша Тота \cite[III §3]{tot}.
Конечно же все они были доказаны до публикации немецкого оригинала книги в 1953 году, а никак не в 1972 году!
\end{addedbytheeditors}

\subsubsection*{Четыре точки с двумя расстояниями}

Эта замечательная головоломка сгодится для болтовни за обедом;
она появилась как Задача 3a (предложенная Ш. Дж. Эйнхорном и И. Дж. Шёнбергом) в разделе головоломок журнала «Pi Mu Epsilon» за 1985 год \cite{16}.
Позже её поместили на первой странице книги Ноба Ёсигахары \cite{61};
там она приписывалась Дику Хессу.

Я заметил, что очень мало людей находят все шесть конфигураций;
кажется, что почти каждый упирается в какой-то барьер или совершает ошибку, пропуская одну из них.
При этом непредсказуемо, \emph{какую} именно; один из испытуемых упустил квадрат!

\begin{figure}[h!]
\centering
\includegraphics[scale=1]{pics/2dist}
\caption{Все шесть вариантов.}
\label{pic:2dist}
\end{figure}

Все конфигурации показаны на рис. \ref{pic:2dist}.
Последняя из них (трапеция) образована четырьмя из пяти вершин правильного пятиугольника.


\subsubsection*{Преступница и собака}

К этой интересной задаче о побеге привлёк моё внимание Жулио Дженовезе;
она появилась в книге Мартина Гарднера \cite{24}.

Будем считать, что круг имеет единичный радиус.
Представим, что преступница бегает в меньшем концентрическом круге радиуса $r$, где $r < \tfrac14$.
Тогда она сможет попасть в точку круга, которая находится на максимальном расстоянии от собаки (см. рис. \ref{pic:dog});
это потому, что длина окружности меньшего круга составляет меньше четверти длины забора.

Если $r$ достаточно близко к $\tfrac14$, то теперь преступница может бежать прямо к забору.
Это расстояние чуть больше чем $3/4$, а собаке придётся преодолеть половину окружности поля, что составляет $\pi$.
Так как $\pi > 3$, это более чем в четыре раза превышает путь преступницы.

\begin{figure}[h!]
\centering
\includegraphics[scale=1]{pics/dog}
\caption{Точка из которой преступница бежит к забору.}
\label{pic:dog}
\end{figure}

Преимущество собаки в скорости можно увеличить с $4$ до $4{,}6033388$; при этом лучшая стратегия обеих сторон приведёт к тому, что они финишируют одновременно.
На сайте головоломок IBM «Ponder This» \cite[май 2001]{ponder-this} можно найти более подробную информацию.

\subsubsection*{Теннисная загадка}

На этот недочёт указал Дик Хесс --- знаток головоломок и тенниса.
На рис. \ref{pic:tenis} показан след мяча; это ошибка при подаче --- мяч падает за пределы намеченной служебной коробки, но он не является ни длинным, ни широким.
Это не работает при использовании электронных систем контроля.
Интересно узнать, а часто ли такая подача бывает ошибочно засчитана.

\begin{addedbytheeditors}
\textbf{Редакторам:} Кто-то должен это переписать --- я задачу не понял.
\end{addedbytheeditors}

\begin{figure}[ht!]
\centering
\includegraphics[scale=1]{pics/tenis}
\caption{Кто должен увидеть эту ошибку при подаче?}
\label{pic:tenis}
\end{figure}

\subsubsection*{Двойное покрытие прямыми}


Некоторых читателей это разочарует, но ответ да (если принимать аксиому выбора) и есть уйма способов это сделать. % ??? может таких больше чонтинуума?
Однако доказательство требует трансфинитной индукции (!), не оставляя места геометрии.
Задачу (и её решение) мне подбросил физик Сеня Шлосман, который не знает её происхождения.

Данное решение мне нравится как пример простого применения мощного инструмента.
Идея в следующем: мы начинаем с трёх пересекающихся прямых, так что у нас уже есть три направления.
Пусть $\kappa$ --- наименьший ординал мощности континуум (тоже, что множество точек на прямой, точек на плоскости или углов на плоскости).
Посмотрим на множество ординалов ниже $\kappa$.
Каждый из них либо последовательный ординал (как, например, $17$, $188$ или $\omega + 1$), либо предельный ординал (как, например, $\omega$, первый бесконечный ординал);
и у каждого мощность строго меньше континуума.
Мощность ординалов ниже $\kappa$ --- это континуум, поэтому можно пометить все точки плоскости этими ординалами. Теперь точки плоскости образуют \emph{вполне упорядоченное} множество, то есть каждое непустое подмножество содержит точку с наименьшей меткой.


Приступим к трансфинитной индукции.
Предположим у нас есть конфигурация прямых, покрывающая все точки множества точек $S$ с метками меньше $\sigma$ дважды,
точка $P$ с меткой $\sigma$ покрыта менее двух раз,
и ни одна из точек плоскости не покрыта три раза или более.
Заметим, что $\sigma$ --- ординал меньший $\kappa$.
Можно считать, что каждая прямая в конфигурации проходит через одну из точек множества $S$, 
и, значит, прямых в конфигурации меньше чем континуум.
Значит и мощность всех двойных точек конфигурации меньше континуума.
Так как множество направлений прямых континуально, через точку $P$ можно провести прямую, не проходящую через двойные точки,
то есть её можно добавить к нашей конфигурации.
Если $P$ всё ещё не двойная, то придётся один раз повторить процедуру.

Похоже на обман?
Ну да; такое построение совсем не конструктивно.
Означает ли это, что нет хорошего двойного покрытия плоскости прямыми?
Нет, но я такой пример найти не смог; не смог и Сеня.

\begin{addedbytheeditors}
Всего есть $2^{\mathfrak{c}}$ решений --- столько даёт приведённое решение, а больше быть не может, поскольку $2^{\mathfrak{c}}$ это мощность всех подмножеств прямых на плоскости. 

\textbf{Редакторам:} Я переписал заново абзац с трансфинитной индукцией --- оригинал написан очень криво.
\end{addedbytheeditors}


\subsubsection*{Кривая на сфере}

Эту головоломку мне подкинул физик Сеня Шлосман, который услышал её от Алекса Красносельского.
Предложенное Сеней решение следующее.

Выберем любую точку $P$ на кривой, пройдём вдоль кривой половину её длины до точки $Q$.
Пусть $N$ будет точкой сферы на полпути между $P$ и $Q$.
(Будем думать, что $N$ это северный полюс; эта точка определена однозначно, поскольку сферическое расстояние $d(P, Q)$ от $P$ до $Q$ меньше $\pi$.)
Полюс $N$ определяет экватор, и если кривая полностью находится в северном полушарии, то мы закончили.
В противном случае кривая пересечёт экватор.
Пусть $E$ --- одна из точек пересечения.
Тогда $d(E,P) + d(E,Q) = \pi$, ведь если отразить $P$ в экваториальной плоскости, то полученная точка $P'$ будет антиподом $Q$; и, следовательно, $d(E, P') + d(E, Q) = \pi$.

Однако для любой точки $X$ на кривой сумма $d(P, X) + d(X, Q)$ должна быть меньше $\pi$, и это приводит к желаемому противоречию.

Омер Ангел из Университета Британской Колумбии
предложил совсем другое доказательство,
менее элементарное, но все же изящное и познавательное.
Пусть $C$ --- наша замкнутая кривая, а $\hat C$ --- её выпуклая оболочка, то есть наименьшее выпуклое множество, содержащее~$C$.
Если $C$ не содержится в полусфере, то $\hat C$ содержит начало координат;
в противном случае $0$ можно было бы отрезать от~$\hat C$ плоскостью.
Таким образом, по Теореме Каратеодори (смотри ниже), существует набор из четырёх точек на $C$, выпуклая комбинация которых даёт $0$.
Другими словами, тетраэдр, вершинами которого являются эти четыре точки, содержит начало координат.

Давайте теперь двигать эти точки непрерывно друг к другу вдоль кривой.
Когда точки слились вместе, их тетраэдр уже не содержит начало координат, так что где-то по дороге начало координат оказалось на одной из граней тетраэдра.
Три точки, определяющие эту грань, лежат на большом круге, и самый короткий маршрут между любой их парой идёт по этому экватору, не проходя через оставшуюся третью точку.
Следовательно, сумма попарных расстояний трёх точек равна $2\pi$, что невозможно, так как все они лежат на $C$.

Математик Константин Каратеодори (1873---1950) доказал множество красивых теорем.
Одна из наиболее известных говорит следующее: если $v$ содержится в выпуклой оболочке некоторых точек $d$-мерного  пространства, то $v$ лежит и в выпуклой оболочке подмножества из не более чем $d+1$ из этих точек.

Чтобы это доказать, заметим, что принадлежность точки выпуклой оболочке множества
эквивалентна тому, что точка представима как конечная линейная комбинация точек этого множества с положительными коэффициентами, сумма которых равна $1$.
Пусть $k>d+1$, и положим $v=\sum_{i=1}^k a_iv_i$, где $\sum_{i=1}^k a_i=1$ и $a_i>0$ при любом $i$.

Поскольку есть более чем $d$ векторов $v_1-v_i$ при $i=2,\dots,k$, эти вектора линейно зависимы;
следовательно, существуют коэффициенты $b_i$, не все равные нулю, такие что $\sum_{i=2}^k b_i(v_1-v_i)=0$.
Положим $b_1=-\sum_{i=2}^k b_i$; тогда $\sum_{i=1}^k b_i v_i=0$ и $\sum_{i=1}^k b_i=0$, но $b_i\ne 0$ для какого-то $i$.
Таким образом, $v=\sum_{i=1}^k a_iv_i-r\sum_{i=1}^k b_iv_i=\sum_{i=1}^k (a_i-rb_i)v_i$ для любого вещественного $r$.
В частности, если $r$ наименьшее отношение $a_i/b_i$  при $b_i>0$ (пусть оно достигается, скажем, при $i=j$), то $r$ положительно, и $a_i-rb_i\ge0$ для всех $i$.
Таким образом, $v$ представимо в виде выпуклой комбинации, по крайней мере, один из коэффициентов которой (а именно, $a_j-rb_j$) равен нулю, так что $v$ находится в выпуклой комбинации не более чем $k-1$ точек.
Остаётся повторять процесс, пока число $k$ не уменьшимся до $d+1$.

\begin{addedbytheeditors}
Головоломка использовалась как промежуточный результат \cite[Satz I$'$]{fenchel}
в доказательстве Вернера Фенхеля, что \emph{любая замкнутая кривая в пространстве обязана повернуть хотя бы на полный оборот}.
Его доказательство почти совпадает со вторым из приведённых выше.
(Кстати, согласно знаменитой теореме Фари --- Милнора, \emph{любой узел обязан повернуть хотя бы на два полных оборота}; обзор шести различных доказательств этой теоремы представлен в \cite{petrunin-stadler}.)

Этот результат и его обобщения востребованы в дифференциальной геометрии.
По следам одной беседы за обедом 1997 года, Боб Фут собрал из различных его доказательств короткую заметку.
Первое из доказательств в его коллекции практически совпадает с первым приведённым здесь;
его нашли Майк Керкхов, Дан Клинг и сам Боб Фут.
Улучшение этого доказательства принадлежит Стефани Александер, оно, между прочим, обсуждается в ютубовском ролике Серхио Заморы \cite{zamora}.
Ещё одно замечательное доказательство легко строится на основе сферической формулы Крофтона --- \emph{длина сферической кривой равна $\pi$ помноженному на среднее число пересечений кривой с экваторами}.
И ещё одно интересное уточнение этой задачи можно разглядеть в так называемой \emph{теореме мажоризации Решетняка} \cite{reshetnyak}.
\end{addedbytheeditors}

\subsubsection*{Лазерная пушка}

На эту головоломку мне указал Джулио Дженовезе, тот узнал её от Энрико Ле Донне;
они отследили её историю до Математической Олимпиады 1990 года в Ленинграде \cite{17}.
Удивительно, но шестнадцать охранников достаточно!

Похоже, что именно эта головоломка вызвала цепную реакцию исследований \emph{вопросов безопасности}, то есть в каких комнатах, кроме прямоугольных, достаточно конечного числа охранников.
Вопрос ещё не полностью решён даже для многоугольников с рациональными углами, но из работы Евгения Гуткина \cite{34} следует, что среди правильных многоугольников безопасны только равносторонний треугольник, квадрат, и правильный шестиугольник.

Но вернёмся к нашей головоломке.
Основная идея в следующем.
Будем думать, что комната это прямоугольник на плоскости, вы находитесь в точке $P$, а пушка --- в точке $Q$.
Замостим плоскость копиями комнаты, рекурсивно отражая комнату относительно её стен.
Будем считать, что каждая копия содержит копию пушки (рис. \ref{pic:room1}).

\begin{figure}[t!]
\centering
\includegraphics[scale=1]{pics/room1}
\caption{Замощение плоскости отражёнными копиями комнаты.}
\label{pic:room1}
\end{figure}

Каждый выстрел врага можно представить отрезком на этой картинке, идущим из какой-то копии точки $Q$ в точку $P$.
Каждый раз, когда такая линия пересекает границу между прямоугольниками, лазерный луч отражается.
На рисунке показана одна из таких (пунктирных) линий; сплошная линия это путь соответственного лазерного луча.

Наша цель --- перехватить любой выстрел на полпути.
Для этого снимем копию плоскости, показанной на рис. \ref{pic:room1}, прикрепим её к плоскости в точке $P$, и, наконец, уменьшим копию вдвое по вертикали и горизонтали.
Множество копий точки $Q$ на уменьшенной копии плоскости будут подходящими позициями охранников;
они служат нашей цели, потому что каждая копия точки $Q$ на исходном замощении появляется на полпути между ней и вами на уменьшенной копии.


На рис. \ref{pic:room2} уменьшенная копия изображена серым цветом, и указаны некоторые воображаемые пути лазера; можно видеть, что на полпути они проходят через соответствующие более мелкие точки серой сетки.

\begin{figure}[t!]
\centering
\includegraphics[scale=1]{pics/room2}
\caption{Уменьшенная копия плоскости с точкой $P$ наложенной на себя.}
\label{pic:room2}
\end{figure}

\begin{figure}[b!]
\centering
\includegraphics[scale=1]{pics/room3}
\caption{Позиции телохранителей в исходном прямоугольнике.}
\label{pic:room3}
\end{figure}

Конечно, таких точек бесконечно много, но мы утверждаем, что все они являются отражениями набора из 16 точек в исходной комнате.
Четыре из них уже в исходной комнате.
Четыре точки в комнате слева от исходной комнаты можно отразить обратно, чтобы получить четыре новые точки, и аналогично для комнаты выше исходной.
Наконец, четыре точки в комнате выше \emph{и} слева от исходной комнаты могут быть отражены дважды, чтобы получить последние четыре точки в исходной комнате.
На рис. 17 к исходной четвёрке точек, помеченных серым, добавлены двенадцать новых, помеченных чёрным.
Также показан воображаемый путь лазера, и его настоящий путь, проходящий через одну из новых точек.

Поскольку каждая комната выглядит точно так же, как и исходная, или одна из трёх других, которые мы только что рассмотрели, все позиции охранников в плоскости являются отражениями описанных шестнадцати точек в исходной комнате.
Поскольку каждая линия от копии $Q$ проходит через отражённого охранника, фактический выстрел попадает в поставленного охранника на его полпути (если не раньше) и поглощается.

Если тщательно подобрать местоположения точек $P$ и $Q$, то позиции некоторых охранников совпадут;
но в общем случае потребуется полный набор из шестнадцати.


\begin{addedbytheeditors}
Возможно проще разобрать сначала задачу на плоском торе (то есть прямоугольнике со склеенными противоположными сторонами).
Убедиться, что в этом случае достаточно четырёх охранников, а потом свести задачу про прямоугольник к четырём задачам о торе.
\end{addedbytheeditors}



\subsubsection*{Ящик в ящике}

Эту замечательную головоломку мне подкинул Энтони Квас (Университет Виктории), который услышал её и приведённое ниже решение от Исаака Корнфельда, профессора Северо-Западный университета (Иллинойс).
Корнфельд узнал о ней много лет назад в Москве.

Пусть $B_\varepsilon$ --- $\varepsilon$-орестность параллелепипеда $B$;
другими словами, это множество всех точек в пространстве, находящихся на расстоянии $\varepsilon$ или меньше от какой-либо точки $B$.
Если $B$ имеет размеры $a \z\times b \z\times c$, то множество $B_\varepsilon$ выглядит как параллелепипед $(a + 2\varepsilon) \z\times (b + 2\varepsilon) \z\times (c + 2\varepsilon)$ с закруглёнными краями и углами.
Точный объём $B_\varepsilon$ будет равен
$abc$ (объем $B$)
плюс $2ab\varepsilon + 2ac\varepsilon + 2bc\varepsilon$ (объем пластин, добавленных к шести граням)
плюс $4a\pi\varepsilon^2 /4 + 4b\pi\varepsilon^2 /4 + 4c\pi\varepsilon^2 /4$ (объем 12-ти штапиков вдоль рёбер --- каждый с поперечным сечением в виде четверти круга),
плюс $4\pi\varepsilon^3 /3$, так как восемь осьмушек, добавленных к углам, образуют шар.
Всего получается
\[V(B_\varepsilon)=\tfrac43\pi\varepsilon^3+(a+b+c)\pi\varepsilon^2+2(ab+bc+ca)\varepsilon+abc.\]

Точно изобразить $B_\varepsilon$ сложно, поэтому мы спустимся на плоскость и покажем на рисунке 18, как будет выглядеть $B_\varepsilon$, если $B$ был прямоугольником размера $a \times b$ на плоскости.
Здесь формула для площади $\varepsilon$-окрестности будет:
\[S(B_\varepsilon)=\pi\varepsilon^2+2(a+b)\varepsilon+ab.\]

\begin{figure}[ht!]
\centering
\includegraphics[scale=1]{pics/box}
\caption{$\varepsilon$-окрестность прямоугольник $a \times b$.}
\label{pic:box}
\end{figure}

Вернёмся в трёхмерное пространство.
Если ящик $A$ (с размерами, скажем, $a'\times b'\times c'$) находится внутри ящика $B$, то $A_\varepsilon$ также находится внутри $B_\varepsilon$ для любого $\varepsilon > 0$.
Следовательно, $V(A_\varepsilon ) \z< V(B_\varepsilon)$.
Однако, если выбрать \emph{огромное} $\varepsilon$, то доминирующим членом в разнице их объёмов станет
\[(a+b+c)\pi\varepsilon^2-(a'+b'+c')\pi\varepsilon^2\]
Поскольку этот член должен быть неотрицательным, получаем, что ящик $B$ дороже $A$.

Эта задача появлялась на Турнире городов 1998 года (5-я задача основного варианта).
Решение, предложенное там, было другим.
Оно принадлежало Андрею Сторожеву, выходцу из России, который сейчас работает в Австралийском Математическом Фонде.
Решение Сторожева основано на наблюдении, что площадь поверхности внутреннего ящика $A$ должна быть меньше, чем у  $B$.
Это можно увидеть, спроецировав каждую грань $A$ наружу перпендикулярно самой себе, 
и посмотреть на покрытые части поверхности $B$.
Утверждение следует поскольку эти шесть частей не пересекаются, и каждая не меньше соответствующей грани $A$.

Это сравнение площадей можно записать алгебраически 
\[2a'b'\z+2b'c'\z+2c'a' \z< 2ab\z+2bc\z+2ca,\] 
но мы также знаем, что $a'^2\z+b'^2\z+c'^2\z<a^2\z+b^2\z+c^2$, сравнивая диагонали двух ящиков.
Сложив эти два неравенства, получаем 
\[(a'+b'+c')^2<(a+b+c)^2\]
--- готово!

\begin{addedbytheeditors}
Первое из приведённых решений было найдено в 1999 году Александром Шенем \cite{shen}, но возможно это случилось и раньше.
Это же рассуждение доказывает, что приведённое неравенство выполняется для всех параллелепипедов (не обязательно прямоугольных) --- то есть если один параллелепипед содержит другой, то сумма всех рёбер внешнего не меньше суммы рёбер внутреннего.

Объём $\varepsilon$-окрестности любого выпуклого тела (не обязательно параллелепипеда) выражается многочленом, это было замечено Якобом Штейнером \cite{steiner}.
Коэффициенты этого многочлена можно выразить через так называемые \emph{поперечные меры}.
Про одномерную поперечную меру можно думать как про среднюю длину проекций тела на случайные прямые.
Двумерная мера это средняя площадь проекций на случайные плоскости и так далее.
В $d$-мерном пространстве у выпуклого тела есть поперечные меры всех размерностей от $1$ до $d$;
поперечная мера размерности $d$ это объём тела, а $(d-1)$-мерная пропорциональна площади его поверхности.

Ясно, что если одно тело содержится в другом, то $k$-мерная поперечная мера у первого не больше, чем у второго.
Это даёт серию обобщений нашей задачи на все размерности.

Саму задачу можно воспринимать как рекламу \emph{смешанным объёмам} --- замечательному инструменту в исследовании выпуклых тел.
С ним можно познакомиться по классической книге \cite{burago-zalgaller}.
\end{addedbytheeditors}



