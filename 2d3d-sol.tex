\section*{Источники и решения}

\subsection*{Монеты на столе}

Эта интересная головоломка досталась мне от специалиста по информатике Гая Киндлера во время замечательного года проведённого нами в Институте перспективных исследований Принстона.

\begin{figure}[t!]
\centering
\includegraphics[scale=1]{pics/coin1}
\caption{Нельзя добавить новую монету без перекрытия со старыми.}
\label{pic:coin1}
\end{figure}

\begin{figure}[b!]
\centering
\includegraphics[scale=1]{pics/coin2}
\caption{Стол покрыт удвоенными монетами.}
\label{pic:coin2}
\end{figure}

Для начала заметим, что если у каждой из исходных монет удвоить радиус (скажем, с $1$ до $2$), то как видно на рисунках \ref{pic:coin1} и \ref{pic:coin2}, покроется весь стол.
Почему?
Ну, если точка $P$ не покрыта, то она лежит на расстоянии $2$ или более от центра любой монеты, так что  к исходной конфигурации можно добавить (маленькую) монету, с центром в $P$.

Дело бы было кончено если мы сможем покрыть каждую большую монету четырьмя маленькими,
но это невозможно.

Однако  этим свойством обладает сам прямоугольник --- его можно разбить на четыре уменьшенных копии самого себя.
Итак, давайте сожмём вдвое всю картинку (ту где большие монеты, покрывают весь стол) и воспользуемся четырьмя её копиями (как на рисунке \ref{pic:coin3}), чтобы покрыть исходный стол!


\begin{figure}[t!]
\centering
\includegraphics[scale=1]{pics/coin3}
\caption{Четыре уменьшенные стола покрывают стол целиком.}
\label{pic:coin3}
\end{figure}

\begin{figure}[b!]
\centering
\includegraphics[scale=1]{pics/coin4}
\caption{Покрытие пола  сотовой ???.}
\label{pic:coin4}
\end{figure}

Удивительно (возможно), что это прекрасное, но, как кажется, довольно грубое рассуждение даёт наилучший возможный оценку: замените $4$ на что-то меньшее, скажем, $3{,}99$, и утверждение перестанет быть верным.

Чтобы понять это, рассмотрим предельный случай, когда стол очень большой, а монет много, так что граничные эффекты пренебрежимо малы.
Заменим стол на ванную комнату покрытию плиткой подобно пчелиным сотам;
то есть каждая плитка это правильный шестиугольник диаметра (скажем) 2.
Поскольку каждую плитку можно разбить на шесть равносторонних треугольников со стороной 1 и, значит, площадью $\sqrt{3}/4$, сама плитка имеет площадь $6\times\sqrt{3}/4=3\times\sqrt{3}/2$.

\begin{figure}[t!]
\centering
\includegraphics[scale=1]{pics/coin5}
\caption{Замощение пчелиными сотами.}
\label{pic:coin5}
\end{figure}

Весь пол можно покрыть, покрыв каждую плитку монетой, граничная окружность которой описана вокруг плитки (см. рисунок \ref{pic:coin4}).

Тогда каждая монета имеет радиус $1$ и, следовательно, площадь~$\pi$.
Если площадь пола $A$ то, игнорируя граничные эффекты, общая площадь монет будет $\pi A/(3\sqrt{3}/2)\approx 1{,}2092\times A$.

Теперь давайте разберёмся, насколько разряжено можно расположить монеты на полу, чтобы нельзя было добавить ещё монету без перекрытия со старыми?
Воспользуемся той же плиткой, но на этот раз мы покроем только треть плиток (рис. \ref{pic:coin5}).
Положим по монете раиуса чуть больше радиуса вписанной окружности в середине каждой такой плитки. 
Это не позволяет нам добавить ещё монет; какова же теперь общая площадь монет?

Ну, радиус монеты немного больше высоты одного из шести равносторонних треугольников, составляющих шестиугольник — а именно, $\sqrt{3}/2$.
Следовательно, площадь монеты просто превышает $\pi \z\times (\sqrt{3}/2)^2 = 3\pi/4$.

Отсюда следует, что общую площадь монет на полу можно сделать произволно близкой к $(1/3) \times (3\pi/4) \times A/(3\sqrt{3}/2) = \pi A/(6 \sqrt{3}) \z\sim 0{,}3023 \times A$, а это ровно четверть того, что было раньше!

Заметьте, что в результате мы доказали не только утверждение головоломки, но ещё два не очень простых экстремальных свойства кругов на плоскости.
Первое утверждает, что нет лучшего способа покрыть плоскость единичными кругами, чем описать плитки в шестиугольной плитке, как мы делали выше;
второе, что нет лучшего конфигурации монет без возможности добавить лишнюю, чем помещать на каждой третьей шестиугольной плитке круг, немного больший, чем вписанный, опять же, как мы делали выше.

Если вы думаете, что эти свойства очевидны изначально, то подумайте об ещё более очевидном факте, что самым плотным способом упаковки единичных кругов в плоскости является использование вписанных кругов в каждом шестиугольнике пчелиных сот.
Это было доказано только в 1972 году великим венгерским геометром Ласло Фейеш Тотем (1915---2005)!

\begin{addedbytheeditors}
Доказательства всех трёх последних утверждений приводятся в замечательной книжке Фейеша Тота \cite[III §3]{tot}.
Конечно же это было доказано до публикации немецкого оригинала книги в 1953 году, а никак ни в 1972 году!
\end{addedbytheeditors}

\subsection*{Четыре точки с двумя расстояниями}

Эта замечательная головоломка подойдёт для обсуждения за обедом;
она появилась как Задача 3a (предложенная Ш. Дж. Эйнхорн и И. Дж. Шёнберг) в разделе «Головоломки» журнала Pi Mu Epsilon за 1985 год [16].
Позже её поместили на первой странице книги Ноба Ёсигахары [61];
там она приписывается Дику Хессу.

Я заметил, что очень мало людей находят все шесть конфигураций;
кажется, что почти каждый упирается в какой-то барьер или совершает ошибку, и одна из конфигураций пропускается.
При этом непредсказуемо какая именно; один из подопытных не заметил квадрата!

\begin{figure}[h!]
\centering
\includegraphics[scale=1]{pics/2dist}
\caption{Все шесть вариантов.}
\label{pic:2dist}
\end{figure}

Все конфигурации показаны на рис. \ref{pic:2dist}.
Последняя из них (трапеция) образована четырьмя из пяти вершин правильного пятиугольника.


\subsection*{Преступница и собака}

К этой интересной задаче о побеге привлёк моё внимание Жулио Дженовезе;
она появилась в «Нескучной математике» Мартина Гарднера.

Определим единицу как радиус поля.
Представим, что преступница бегает в меньшем концентрическом круге радиуса $r$, где $r < \tfrac14$.
Тогда она сможет подбежать к самой удалённой доступной точке от собаки (см. рис. \ref{pic:dog});
это потому, что длина окружности меньшего круга будет меньше, чем $\tfrac14$ окружности поля.
Но если $r$ достаточно близко к $\tfrac14$, преступница может бежать прямо к забору.
Её расстояние всего немного больше чем $3/4$, а собаке придётся преодолеть половину окружности поля, что составляет $\pi$.
Так как $\pi > 3$, это больше чем в четыре раза дальше, чем путь преступницы.

\begin{figure}[h!]
\centering
\includegraphics[scale=1]{pics/dog}
\caption{Точка из которой преступница бежит к забору.}
\label{pic:dog}
\end{figure}

Преимущество собаки в скорости можно увеличить с $4$ до $4{,}6033388$, при этом лучшая стратегия обеих сторон приведёт к тому, что они финишируют одновременно.
Более подробную информацию можно найти на сайте головоломок IBM «Ponder This», май 2001 года.

\subsection*{Теннисная загадка}

На эту маленькую неточность обратил моё внимание любитель головоломок и тенниса Дик Хесс.
На рисунке 14 показан след мяча; это ошибка при подаче, поскольку мяч чисто падает за пределы намеченной служебной коробки, но он не является ни длинным, ни широким.
Это не помогает, если есть машина, вызывающая заднюю линию подачи.
Интересно узнать, а часто ли такая подача бывает ошибочно засчитана.

\begin{figure}[h!]
\centering
\includegraphics[scale=1]{pics/tenis}
\caption{Кто должен увидеть эту ошибку при подаче?}
\label{pic:tenis}
\end{figure}

\subsection*{Двойное покрытие прямыми}

Этот ответ разочарует некоторых читателей --- ответ да (если принимать аксиому выбора).
На самом деле существует бесконечно
много способов это сделать.
Однако доказательство требует трансфинитной индукции (!) и не даёт нам использовать геометрию.
Задачу (и её решение) мне предоставил физик Сеня Шлосман, который не знает её происхождения.

Тем не менее, мне нравится это решение как пример лёгкого применения мощного инструмента.
Идея в следующем: мы начинаем с трёх пересекающихся прямых, так что у нас уже есть три направления.
Теперь пусть $\kappa$ будет наименьшим ординалом мощности континуума (равномощно множеству точек на прямой, точек на плоскости или углов на плоскости).
Посмотрим на множество ординалов ниже $\kappa$.
Каждый из них либо последовательный ординал (как, например, $17$, $188$ или $\omega + 1$), либо предельный ординал (как, например, $\omega$, первый бесконечный ординал);
и у каждого мощность строго меньше континуума.
Мощность ординалов ниже $\kappa$ --- это континуум, поэтому можно пометить все точки плоскости этими ординалами. Теперь точки образуют «хорошо упорядоченное» множество, то есть каждое непустое подмножество плоскости содержит точку с наименьшей меткой.

Для проведения трансфинитной индукции ...

Похоже на обман?
Ну, да; такое построение не имеет никакого практического смысла.
Означает ли это, что нет хорошего двойного покрытия плоскости прямыми?
Нет, но я такой не смог найти; не смог и Сеня.

\begin{addedbytheeditors}
Трансфинитная индукция была написана криво --- по-моему надо переписать.
\end{addedbytheeditors}


\subsection*{Кривая на сфере}

Эту головоломку мне подкинул физик Сени Шлосмана, который услышал её от Алекса Красносельского.
Предложенное Сеней решение следующее.

Выберем любую точку $P$ на кривой, пройдём вдоль кривой половину её длины до точки $Q$.
Пусть $N$ (будем думать, что это северный полюс) будет точкой сферы на полпути между $P$ и $Q$.
(Поскольку сферическое расстояние $d(P, Q)$ от $P$ до $Q$ меньше $\pi$, точка $N$ определена однозначно).
Полюс $N$ определяет экватор, и если кривая полностью находится в северном полушарии, то мы закончили.
В противном случае кривая пересекает экватор.
Пусть $E$ будет одной из точек пересечения.
Тогда $d(E,P) + d(E,Q) = \pi$, ведь если вы отразить $P$ в экваториальной плоскости, то полученная точка $P'$ будет антиподом $Q$; и следовательно, $d(E, P') + d(E, Q) = \pi$.

Однако для любой точки $X$ на кривой сумма $d(P, X) + d(X, Q)$ должна быть меньше $\pi$, и это приводит к желаемому противоречию.

Омер Ангел, из Университета Британской Колумбии,
предложил совсем другое доказательство,
менее элементарное, но все же изящное и познавательное.
Пусть $C$ --- наша замкнутая кривая, а $\hat C$ --- её выпуклая оболочка, то есть наименьшее выпуклое множество, содержащее $C$.
Если $C$ не содержится в полусфере, то $\hat C$ содержит начало координат;
в противном случае $0$ можно было бы отрезать от $\hat C$ плоскостью.
Таким образом, по Теореме Каратеодори (смотри ниже), существует набор из четырёх точек на $C$, выпуклая комбинация которых даёт $0$.
Другими словами, тетраэдр, вершинами которого являются эти четыре точки, содержит начало координат.

Давайте теперь двигать эти точки непрерывно друг к другу вдоль кривой.
Когда точки сливаются, их тетраэдр больше не будет содержать начало координат, так что где-то по дороге, начало координат оказалось на одной из граней тетраэдра.
Три точки, определяющие эту грань, лежат на большом круге, самый короткий маршрут между любой парой идёт по этому экватору, не проходя через оставшуюся третью точку.
Следовательно, сумма попарных расстояний трёх точек равна $2\pi$, что невозможно, так как все они лежат на $C$.

Математик Константин Каратеодори (1873---1950) доказал множество элегантных теорем, одна из наиболее известных утверждает, что если $v$ лежит в выпуклой оболочке некоторых точек в пространстве размерности $d$, то $v$ лежит и в выпуклом оболочке подмножества из не более чем $d+1$ точки из них.

Чтобы это доказать, заметим, что принадлежность точки выпуклой оболочке множества
эквивалентна тому, что точка представима как конечная линейная комбинация точек этого множества с положительными коэффициентами, сумма которых равна $1$.
Пусть $k>d+1$, и положим $v=\sum_{i=1}^k a_iv_i$, где $\sum_{i=1}^k a_i=1$ и $a_i>0$ при любом $i$.

Поскольку есть более чем $d$ векторов $v_1-v_i$ при $i=2,\dots,k$, эти вектора линейно зависимы;
следовательно, существуют коэффициенты $b_i$, не все равные нулю, такие что $\sum_{i=2}^k b_i(v_1-v_i)=0$.
Положим $b_1=-\sum_{i=2}^k b_i$; тогда $\sum_{i=1}^k b_i v_i=0$ и $\sum_{i=1}^k b_i=0$, но $b_i\ne 0$ для какого-то $i$.
Таким образом, $v=\sum_{i=1}^k a_iv_i-r\sum_{i=1}^k b_iv_i=\sum_{i=1}^k (a_i-rb_i)v_i$ для любого вещественного $r$.
В частности, если $r$ наименьшее (пусть оно достигается при, скажем, $i=j$) отношение $a_i/b_i$  при $b_i>0$, то $r$ положительно, и $a_i-rb_i\ge0$ для всех $i$.
Таким образом,$v$ представимо в виде выпуклой комбинации, по крайней мере, один из коэффициентов которой (а именно, $a_j-rb_j$) равен нулю, так что $v$ находится в выпуклой комбинации не более чем $k-1$ точек.
Повторяем процесс до тех пор, пока число $k$ не уменьшимся до $d+1$ точек.

\begin{addedbytheeditors}
Это лемма из доказательства Вернера Фенхеля \cite[Satz I$'$]{fenchel} теоремы о том, что любая замкнутая кривая в пространстве обязана повернуть хотя бы на полный оборот.
Его доказательство почти совпадает со вторым из приведённых выше.
По следам одной беседы за обедом 1997 года, Боб Фут написал короткую заметку о нескольких других доказательствах.
Первое из доказательств в его коллекции практически совпадает с первым приведённым здесь;
его нашли Майк Керкхов, Дан Клинг и сам Боб Фут.
Улучшение этого доказательства принадлежит Стефани Александер, оно между прочим обсуждается в ютубовском ролике Серхио Заморы \cite{zamora}.
Ещё одно замечательное доказательство легко строится на основе сферической формылы Крофтона --- длина сферической кривой равна $\pi$ помноженному на среднее число пересечений кривой с экваторами.
Интересное уточнение этого утверждение дано так называемой теореме мажоризации Решетняка \cite{reshetnyak}.
\end{addedbytheeditors}

\subsection*{Лазерная пушка}

На эту головоломку обратил моё внимание Джулио Дженовезе, тот узнал её от Энрико Ле Донне;
они отследили её историю до Математической Олимпиады 1990 года в Санкт-Петербурге.
Удивительно, но шестнадцать охранников достаточно!

Похоже, что именно эта головоломка вызвала цепную реакцию исследований \emph{вопросов безопасности}, то есть для комнат каких форм, кроме прямоугольных, достаточно конечного числа охранников.
Вопрос ещё не полностью решён даже для многоугольников с рациональными углами, но из работы Евгения Гуткина [34] следует, что правильные безопасные многоугольники --- это равносторонний треугольник, квадрат, правильный шестиугольник и всё.

Но вернёмся к нашей головоломке.
Основная идея в следующем.
Будем думать, что комната это прямоугольник на плоскости, вы находитесь в точке $P$, а пушка в точке $Q$.
Замостим плоскость копиями комнаты, рекурсивно отражая комнату относительно её стен, будем считать, что каждая копия содержит копию пушки (рис. \ref{pic:room1}).

\begin{figure}[h!]
\centering
\includegraphics[scale=1]{pics/room1}
\caption{Замощение плоскости отражёнными копиями комнаты.}
\label{pic:room1}
\end{figure}

Каждый выстрел врага можно представить отрезком на этой картинке, идущим из какой-то копии точки $Q$ в точку $P$.
Каждый раз, когда такая линия пересекает границу между прямоугольниками, лазерный луч отражается.
На рисунке показана одна из таких (пунктирных) линий; сплошная линия это путь соответственного лазерного луча.

Наша цель --- перехватить любой выстрел на полпути.
Для этого снимем копию плоскости, показанной на рис. \ref{pic:room1}, прикрепим её к плоскости в $P$, и, наконец, уменьшим копию вдвое по вертикали и горизонтали.
Множество копий точки $Q$ на уменьшенной копии будут нашими позициями охранников; они служат нашей цели, потому что каждая копия точки $Q$ на исходном замощении появляется на полпути между ней и вами на уменьшенной копии.

\begin{figure}[t!]
\centering
\includegraphics[scale=1]{pics/room2}
\caption{Уменьшенная копия плоскости с точкой $P$ наложенной на себя.}
\label{pic:room2}
\end{figure}

На рис. \ref{pic:room2} уменьшенная копия изображена серым цветом, и указаны некоторые воображаемые пути лазера; можно видеть, что на полпути они проходят через соответствующие более мелкие точки серой сетки.

Конечно, таких точек бесконечное множество, но мы утверждаем, что все они являются отражениями набора из 16 точек в исходной комнате.
Четыре из них уже в исходной комнате.
Четыре точки в комнате слева от исходной комнаты можно отразить обратно, чтобы дать четыре новые точки, и аналогично для комнаты выше исходной.
Наконец, четыре точки в комнате выше и слева от исходной комнаты могут быть отражены дважды, чтобы получить последние четыре точки в исходной комнате.
На рисунке 17 к исходному прямоугольнику добавлены двенадцать новых точек (центры заполнены серым), показанных в чёрном цвете.
Добавлен воображаемый путь лазера, и его настоящий путь, проходящий через одну из новых точек.

Поскольку каждая комната выглядит точно так же, как и исходная, или одна из трёх других, которые мы только что рассмотрели, все точки охраны в плоскости являются отражениями шестнадцати точек, которые мы сейчас описали в исходной комнате.
Поскольку каждая линия от копии $Q$ проходит через отражённого охранника, фактический выстрел попадает в поставленного охранника на его полпути (если не раньше) и поглощается.

Если тщательно подобрать местоположения $P$ и $Q$, то некоторые из позиций охраны совпадут;
но в общем случае потребуется полный набор из шестнадцати позиций.

\begin{figure}[h!]
\centering
\includegraphics[scale=1]{pics/room3}
\caption{Позиции телохранителей в исходном прямоугольнике.}
\label{pic:room3}
\end{figure}

\begin{addedbytheeditors}
Чуть проще разобрать задачу на плоском торе (то есть прямоугольнике со склеенными противоположными сторонами).
Убедиться, что в этом случае достаточно четырёх охранников, а потом свести задачу про прямоугольник к четырём задачам о торе.
\end{addedbytheeditors}



\subsection*{Ящик в ящике}

Эту замечательную головоломку мне подкинул Энтони Квас (Университет Виктории), который услышал её и решение ниже от Исаака Корнфельда, профессора Университета Нортвестерн.
Корнфельд узнал о ней много лет назад в Москве.

Пусть $B_\varepsilon$ --- $\varepsilon$-орестность параллелепипеда $B$;
другими словами, это множество всех точек в пространстве, находящихся на расстоянии $\varepsilon$ или меньше от какой-либо точки $B$.
Если $B$ имеет размеры $a \times b \times c$, то $B_\varepsilon$ похож на $(a + 2\varepsilon) \times (b + 2\varepsilon) \times (c + 2\varepsilon)$ с закруглёнными краями и углами.
Точный объём $B_\varepsilon$ будет равен $abc$ (объем $B$) плюс $2ab\varepsilon + 2ac\varepsilon + 2bc\varepsilon$ (объем пластин, добавленных к шести граням) плюс $4a\pi\varepsilon^2 /4 + 4b\pi\varepsilon^2 /4 + 4c\pi\varepsilon^2 /4$ (объем 12-ти штапиков вдоль рёбер --- каждая имеет поперечное сечение в виде четверти круга, плюс $4\pi\varepsilon^3 /3$, так как восемь осьмушек, добавленных к углам, образуют шар.
Всего получилось
\[V(B_\varepsilon)=\tfrac43\pi\varepsilon^3+(a+b+c)\pi\varepsilon^2+2(ab+bc+ca)\varepsilon+abc.\]

Точно изобразить $B_\varepsilon$ сложно, поэтому мы спустимся на плоскость и покажем на рисунке 18, как будет выглядеть $B_\varepsilon$, если $B$ был прямоугольником размером $a \times b$ на плоскости.
Здесь формула для площади расширенной фигуры будет:
\[S(B_\varepsilon)=\pi\varepsilon^2+2(a+b)\varepsilon+ab.\]

\begin{figure}[ht!]
\centering
\includegraphics[scale=1]{pics/box}
\caption{$\varepsilon$-окрестность прямоугольник $a \times b$.}
\label{pic:box}
\end{figure}

Вернёмся в трёхмерное пространство.
Если ящик $A$ (с размерами, скажем, $a'$, $b'$, и $c'$) находится внутри ящика $B$, то $A_\varepsilon$ также находится внутри $B_\varepsilon$ для любого $\varepsilon > 0$.
Следовательно, $V(A_\varepsilon ) \z< V(B_\varepsilon)$.
Однако, если выбрать очень большое $\varepsilon$, доминирующим членом в разнице их объёмов будет
\[(a+b+c)\pi\varepsilon^2-(a'+b'+c')\pi\varepsilon^2\]
Поскольку этот член должен быть неотрицательным, $B$ --- более дорогой ящик.

Задача также появлялась на Турнире Городов 1998 года ---
задача 5 основном варианте.
Решение, предложенное там, было другим и принадлежало Андрею Сторожеву, выходцу из России, который сейчас работает в Австралийском Математическом Фонде.
Решение Сторожева основано на наблюдении, что площадь поверхности внутреннего ящика $A$ должна быть меньше, чем у  $B$.
Это можно увидеть, спроецировав каждую грань $A$ наружу перпендикулярно самой себе, чтобы она захватывала часть поверхности $B$.
Утверждение следует поскольку эти шесть частей не пересекаются, и каждая не меньше соответствующей грани $A$.

Это сравнение площадей можно записать алгебраически $2a'b'+2b'c'+2c'a' < 2ab+2bc+2ca$, но мы также знаем, что $a'^2+b'^2+c'^2<a^2+b^2+c^2$, сравнивая диагонали двух ящиков.
Сложив эти два неравенства, получаем 
$(a'+b'+c')^2<(a+b+c)^2$ --- готово!

\begin{addedbytheeditors}
Первое из приведённых решений было найдено Александром Шенем \cite{shen}, но возможно существенно старше.

Коэффициенты многочлена выражающего объём с точностью до нормировки называются \emph{поперечными мерами}.
Про одномерную поперечную меру можно думать как про среднюю длину проекций тела (в нашем случае параллелепипеда) на случайные прямые.
Двумерная мера это средняя площадь проекций на случайные плоскости и так далее.
В $d$-мерном пространстве у выпуклого тела есть поперечные меры всех размерностей от $1$ до $d$;
поперечная мера размерности $d$ это объём тела.

Ясно, что если одно тело содержится в другом то $k$-мерная поперечная мера первого не больше чем у второго.
Это даёт серию обобщений нашей задачи на все размерности.

Саму задачу можно воспринимать как рекламу \emph{смешанным объёмам} --- замечательную инструменту в исследовании выпуклых тел.
С ним можно познакомиться по классической книге \cite{burago-zalgaller}.
\end{addedbytheeditors}
