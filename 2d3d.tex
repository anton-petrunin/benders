\chapter{Двумерность и трёхмерность}

\setlength{\epigraphwidth}{.83\textwidth}
\epigraph{Математики давно считают унизительным заниматься задачами элементарной геометрии в размерностях два и три, несмотря на то, что как раз такая математика имеет практическую ценность.}{--- Бранко Грюнбаум и Джеффри Шепард,\\ Handbook of Applicable Mathematics}

Для многих первое знакомство с теоремами и доказательствами
происходит при изучении евклидовой плоскости
в старших классах школы.
Однако сейчас вам придётся решать задачи далёкие от «Начал» Евклида;
они проверят насколько глубоко вы проникли в мир двух и трёх измерений.

\subsection*{Монеты на столе}\label{Монеты на столе}

На прямоугольном столе лежат 100 идентичных монет так, что нельзя добавить ещё монету без налегания.
(Монета может выступать за край, главное, чтобы её центр находился на столе.)
Докажите, 400-от таких монет достаточно, чтобы покрыть весь стол!
(Монеты могут пересекаться и выступать за край стола).

\parit{Примечание.}
Предполагается, что каждая монета это круг пренебрежимо малой толщины.

\subsection*{Четыре точки с двумя расстояниями}

Опишите все расположения четырёх точек на плоскости такие, что между их парами есть только два различных расстояния.

\subsection*{Преступница и собака}

Преступница содержится на круглом  поле, окружённом забором.
Её охраняет свирепая собака, она способна бегать вчетверо быстрее преступницы, но приучена бегать только вдоль забора.
Если преступница подбежит к забору в месте, где собаки нет, то мгновенно его перелезет и сбежит.
Но сможет ли она добраться до какой-то точки забора быстрее собаки?

\subsection*{Теннисная загадка}

Мяч при подаче попал \emph{вне поля}, но ни один из линейных судей не может сказать «аут».
Где же он приземлился?

\parit{Примечания.}
На крупном теннисном турнире каждый линейный судья отвечает за одну линию и объявляет «аут» если мяч пролетает мимо этой линии и приземляется на неправильной стороне.
К сожалению, возможен удар вне поля, при котором линейные судья молчат; что это за удар?

\begin{addedbytheeditors}
\textbf{Редакторам:} Кто-то должен это переписать --- я задачу не понял.
\end{addedbytheeditors}


\subsection*{Двойное покрытие прямыми}

Используя два полных набора параллельных прямых, можно покрыть плоскость так, что каждая точка лежит ровно на двух прямых.
Можно ли сделать это иным способом, то есть покрыть каждую точку плоскости ровно дважды, используя набор прямых, с более чем двумя разными направлениями?

\parit{Примечания.} К примеру, можно попробовать взять все прямые, касающиеся некоторой окружности.
Это отлично работает за пределами окружности, но точки на окружности покрыты лишь раз, а те, что внутри вовсе не покрыты.

\medskip

Подошло время для трёхмерных задач.

\subsection*{Кривая на сфере}

Докажите, что если замкнутая кривая на единичной сфере короче $2\pi$, то она содержится в какой-то полусфере.

\parit{Примечания.} Похоже на правду, ведь периметр большого круга (границы полусферы) равен $2\pi$.
Но как это доказать?

\subsection*{Лазерная пушка}

Вы стоите в большой прямоугольной комнате с зеркальными стенами.
В другом месте этой комнаты стоит ваш враг с лазерной пушкой.
Вы и ваш враг --- идеальные точки.
Единственный способ защититься от лазера --- расставить телохранителей (тоже точки) в комнате так, чтобы они поглощали лазерные лучи.
Сколько телохранителей необходимо для защиты от всех возможных выстрелов?

\parit{Примечание.} «Бесконечно много» --- приемлемый ответ, если конечно он верен!

\medskip

Мы завершаем главу замечательной задачей, применимой в обычной жизни.
Часто для определения стоимости пересылки прямоугольного ящика складывают его длину, ширину и высоту,
и находят эту сумму в таблице;
конечно же, чем больше сумма, тем выше стоимость.
Можно ли сэкономить на стоимости отправки, засунув какой-то ящик в ящик побольше, но подешевле?

\subsection*{Ящик в ящике}\label{Ящик в ящике}


Допустим, что стоимость прямоугольного ящика равна сумме его длины, ширины и высоты.
Докажите или опровергните, что невозможно уместить ящик в более дешёвый ящик.

\parit{Примечания.}
Очевидно, что длинный тонкий ящик можно поместить в более короткий, расположив его вдоль диагонали;
но, похоже придётся слишком много пожертвовать другими двумя измерениями.
На плоскости, то есть с прямоугольниками, используя неравенство треугольника, легко увидеть, что аналогичная экономия невозможна.
Но вроде бы, этот метод не работает в трёхмерном пространстве.
