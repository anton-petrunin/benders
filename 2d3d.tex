\chapter{2D и 3D}

\setlength{\epigraphwidth}{.83\textwidth}
\epigraph{Математики давно считают унизительным заниматься задачами элементарной геометрией в размерностях два и три, несмотря на то что именно такая математика имеет практическую ценность.}{Бранко Грюнбаум и Джеффри Шепард,\\ Handbook of Applicable Mathematics}

Для многих первое знакомство с теоремами и доказательствами происходит в старших классах школы, когда мы изучаем евклидову плоскость.
Однако здесь вам придётся решать очень-очень далекие от «Начал» Евклида;
они проверят насколько глубоко вы проникли в мир двух и трёх измерений.

\subsection*{Монеты на столе}

На прямоугольном столе лежат 100 идентичных монет так, что нельзя добавить ещё монету без налегания.
(Монета может выступать за край, главное, чтобы её центр находился на столе.)
Докажите, 400-та таких монет достаточно чтобы покрыть весь стол!
(Монеты могут пересекаться выступать за край стола).

\parit{Примечание:}
Предполагается, что каждая монета является идеальным кругом пренебрежимо малой толщины.

\subsection*{Четыре точки с двумя расстояниями}

Опишите все расположения четырёх точек на плоскости так, чтобы меду их парами было только два различных расстояния.

\subsection*{Преступница и собака}

Преступница содержится в поле, окружённом круглым забором.
У забором бегает свирепая охранная собака, она способна бегать вчетверо быстрее преступницы, но приучена бегать только вдоль забора.
Если преступница подбежит к забору в месте, где собаки нет, то сможет мгновенно его перелезть и сбежать.
Но сможет ли она добраться до точки забора быстрее собаки?

\subsection*{Теннисная загадка}

Мяч находится «вне поля», но ни один из линейных судей не может это сказать.
Где же он приземлился?

\parit{Комментарий:}
На крупном теннисном турнире каждый линейный судья отвечает за одну линию и объявляет «вне поля», когда мяч пролетает мимо этой линии и приземляется на неправильной стороне.
К сожалению, возможен удар вне поля, который этим способом не регистрируется; что это за удар?

\subsection*{Двойное покрытие прямыми}

Используя два полных набора параллельных прямых, можно покрыть плоскость так, что каждая точка лежит ровно на двух прямых.
Можно ли сделать это иным способом, то есть покрыть каждую точку плоскости ровно дважды, используя набор прямых, содержащих прямые в более чем двух разных направлениях?

\parit{Комментарий:} К примеру, можно попробовать взять все прямые, касающиеся некоторой окружности.
Это отлично работает за пределами окружности, но точки на окружности покрыты лишь раз, а те, что внутри вовсе не покрыты.

\medskip

Подошло время для трёхмерных задач.

\subsection*{Кривая на сфере}

Докажите, что если замкнутая кривая на единичной сфере короче $2\pi$, то она содержится в какой-то полусфере.

\parit{Комментарий:} Похоже на правдой, ведь периметр большого круга (границы полусферы) равен $2\pi$.
Но как это доказать?

\subsection*{Лазерная пушка}

Вы стоите большой прямоугольной комнате с зеркальными стенами.
В другой месте этой комнаты ваш враг с лазерной пушкой.
Вы оба --- идеальные точки.
Единственный способ защиты вас от лазера это выставить в комнате телохранителей (тоже точки) так, чтобы они поглощали лазерный луч.
Сколько телохранителей необходимо для защиты от всех возможных выстрелов?

\parit{Комментарий:} «Бесконечно много» --- приемлемый ответ, если конечно он верен!

\medskip

Мы завершаем главу замечательной задачей, которая действительно имеет отношение к жизни.
Часто для определения стоимости отправки или доставки прямоугольного ящика складывают его длину, ширину и высоту,
а находят эту сумму в таблице;
конечно же, чем больше сумма, тем выше стоимость.
Можно ли сэкономить денег, упаковав какой-то ящик в больший, но при этом более дешёвый?

\subsection*{Ящик в ящике}

Допустим, что стоимость прямоугольного ящика равна сумме его длины, ширины и высоты.
Докажите или опровергните: невозможно уместить ящик в более дешёвый ящик.

\parit{Комментарий:}
Очевидно, что длинный тонкий ящик можно поместить в более короткий, расположив его вдоль диагонали;
но, похоже, что при этом придётся слишком много пожертвовать по другими двумя измерениям.
На плоскости, то есть с прямоугольниками, используя неравенство треугольника, легко увидеть, что аналогичная экономия невозможна.
Но вроде бы, этот метод не работает в трёхмерном пространстве.
