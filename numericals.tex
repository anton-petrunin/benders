\chapter{Числовые загадки}


\setlength{\epigraphwidth}{.85\textwidth}
\epigraph{У творца вселенной таинственный образ действий.
Но он использует десятичную систему счисления и любит круглые
числа.
}{--- Скотт Адамс (1957---)}


Свойства чисел --- странная и прекрасная штука, и неудивительно, что на них основано много замечательных головоломок; иногда они даже помогают нам кое-что понять о самих числах.

\subsection*{Строки и столбцы}

Докажите, что если упорядочить каждую строку матрицы, а затем каждый столбец, то строки останутся упорядоченными!

\subsection*{Нежелательное раскрытие}

Дано алгебраическое выражение с переменными, сложением, умножением и скобками.
Вы рекурсивно раскрываете скобки, используя распределительный закон умножения.
Как убедиться, что этот процесс не будет продолжаться вечно?

\parit{Примечания.}
Может показаться, что число скобок должно уменьшаться.
Однако если раскрыть первые скобки в
\[(x + y)(s(u + v) + t),\]
то получим выражение
\[x(s(u + v) + t) + y(s(u + v) + t),\]
с б\'{о}льшим числом скобок.

\subsection*{Хамелеоны}\label{Хамелеоны}

На острове живут 20 красных, 18 синих и 16 зелёных хамелеонов.
Если встречаются два хамелеона разных цветов, то каждый из них меняет свой цвет на оставшийся третий.
Может ли случиться так, что через некоторое время все хамелеоны будут одного цвета? 

\subsection*{Отсутствующая цифра}

Число $2^{29}$ состоит из $9$ различных цифр; какая цифра отсутствует?

\subsection*{Очень честное разбиение}

Можно ли разбить целые числа от $1$ до $16$ на два набора одинакового размера так,
чтобы у них были равные суммы, равные суммы квадратов и равные суммы кубов?

\subsection*{Восстановление чисел}
Для каких положительных целых чисел $n$ верно следующее: зная все $\binom n2$ попарных сумм $n$ различных положительных целых чисел, всегда можно однозначно восстановить сами числа?


\subsection*{Уравнение ирисок}

$n$ детей стоят кружком и каждый держит несколько ирисок.
Учитель даёт дополнительную ириску каждому ребёнку, у которого их нечётное число,
затем каждый ребёнок передаёт половину своих ирисок ребёнку слева от себя.
Эта пара процедур повторяется, до тех пор пока они уже ничего не меняют.
Докажите, что этот процесс на самом деле завершится, и у всех детей станет по одинаковому (чётному) числу ирисок.

\subsection*{Девяносто девятая цифра}

Какая цифра стоит на 99-м месте после запятой в десятичном разложении числа 
$(1+\sqrt2)^{500}$?

\subsection*{Подмножества с ограничениями}

Каково максимальное число целых чисел от $1$ до $30$, произведение любой пары которых не является полным квадратом?
А что если (вместо этого) ни одно число не кратно другому?
Или, ни у какой пары нет общего делителя (отличного от 1)?

\subsection*{Единообразие бубликов}\label{Единообразие бубликов}

Чёртова дюжина (то есть 13 штук) бубликов обладают таким свойством: любую дюжину из них можно разделить на две кучки по шесть, которые в точности уравновесят друг друга на весах.
Докажите, что все 13 бубликов одинакового веса.

\medskip

Следующая головоломка сложней большинства остальных;
она включена по особой причине.

\subsection*{Юбилейная головоломка}

Поскольку ряд $1 - 1 + 1 - 1 + 1 - \dots$ не сходится,  функция 
$f(x)\z=x-x^2+x^4-x^8+x^{16}-x^{32}+\dots$ не определена при $x=1$.
Однако $f(x)$ сходится при всех положительных вещественных чисел $x<1$.
Сходится ли $f(x)$ при $x$ стремящимся к 1 снизу?

\subsection*{Надёжные мигалки}\label{Надёжные мигалки}

Две обычные мигалки мигают одновременно в момент времени $0$
и мигают дальше; при этом у них вместе \emph{в среднем} получается по одному миганию в минуту.
Известно, что они больше никогда не мигнут одновременно (или, что то же самое, отношение их частот иррационально).

Докажите, что после первой минуты (во временном интервале между $0{:}00$ и $0{:}01$) в каждом интервале между $t$ и $t + 1$ минутой будет ровно одно мигание ($t$ --- натуральное число).

\subsection*{Красные и синие игральные кости}\label{Красные и синие игральные кости}

У нас есть два набора (красный и синий) из $n$ игральных костей с $n$ гранями на каждой;
на гранях каждой кости стоят числа от $1$ до~$n$.
Мы бросили все $2n$ костей одновременно.
Докажите, что можно выбрать непустое подмножество красных и непустое подмножество синих костей с одинаковой суммой!
