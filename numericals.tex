\chapter{Числовые загадки}

% Читает Устинов


\setlength{\epigraphwidth}{.70\textwidth}
\epigraph{Загадочны действия творца вселенной.
Но он использует десятичную систему счисления и любит круглые
числа.
}{--- Скотт Адамс (1957---)}


Поведение чисел загадочно и красиво, поэтому неудивительно, что многие замечательные головоломки основаны на этом поведении и в некоторых случаях даже помогают нам понять его.

\subsection*{Строки и столбцы}\rindex{Строки и столбцы}

Докажите, что если упорядочить каждую строку матрицы, а затем каждый столбец, то строки останутся упорядоченными!

\subsection*{Нежелательное раскрытие}\rindex{Нежелательное раскрытие}

Дано алгебраическое выражение с переменными, сложением, умножением и скобками.
Начнём рекурсивно раскрывать скобки, используя распределительный закон умножения.
Как убедиться, что этот процесс не будет продолжаться вечно?

\parit{Примечания.}
Может показаться, что число скобок должно уменьшаться.
Однако раскрыв первые скобки в
\[(x + y)(s(u + v) + t),\]
получим
\[x(s(u + v) + t) + y(s(u + v) + t),\]
с б\'{о}льшим числом скобок.

\subsection*{Хамелеоны}\rindex{Хамелеоны}\label{Хамелеоны}

На острове живут 20 красных, 18 синих и 16 зелёных хамелеонов.
Если встречаются два хамелеона разных цветов, то каждый из них меняет свой цвет на оставшийся третий.
Смогут ли все хамелеоны стать одного цвета?

\subsection*{Отсутствующая цифра}\rindex{Отсутствующая цифра}

Число $2^{29}$ состоит из $9$ различных цифр; какая цифра отсутствует?

\subsection*{Очень честное разбиение}\rindex{Очень честное разбиение}

Можно ли разбить целые числа от $1$ до $16$ на два таких набора по восемь в каждом,
чтобы у них были равные суммы, равные суммы квадратов и равные суммы кубов?

\subsection*{Восстановление чисел}\rindex{Восстановление чисел}

Для каких натуральных $n$ верно следующее: при известных всех $\binom n2$ попарных суммах $n$ различных натуральных чисел однозначно восстанавливаются сами числа?


\subsection*{Уравнение ирисок}\rindex{Уравнение ирисок}

$n$ детей стоят кружком, и каждый держит несколько ирисок.
Сначала учитель даёт дополнительную ириску каждому ребёнку, у которого их нечётное число,
затем каждый ребёнок передаёт половину своих ирисок ребёнку слева от себя.
Эта пара процедур повторяется до тех пор, пока они уже ничего не меняют.
Докажите, что этот процесс на самом деле завершится, и у всех детей станет по одинаковому (чётному) числу ирисок.

\subsection*{Девяносто девятая цифра}\rindex{Девяносто девятая цифра}

Какая цифра стоит на 99-м месте после запятой в десятичном разложении числа 
$(1+\sqrt2)^{500}$?

\subsection*{Подмножества с ограничениями}\rindex{Подмножества с ограничениями}

Каково максимальное число целых чисел от $1$ до $30$, произведение любой пары которых не полный квадрат?
А что если (вместо этого) ни одно число не кратно другому?
Или ни у какой пары нет общего делителя (отличного от 1)?

\subsection*{Единообразие бубликов}\rindex{Единообразие бубликов}\label{Единообразие бубликов}

Чёртова дюжина бубликов (то есть 13 штук) обладают таким свойством: любую дюжину из них можно разделить на две кучки по шесть, которые в точности уравновесят друг друга на весах.
Докажите, что все 13 бубликов одного веса.

\medskip

{\sloppy 

Следующая головоломка сложней большинства остальных;
она включена по особой причине.

}

\subsection*{Юбилейная головоломка}\rindex{Юбилейная головоломка}

Поскольку ряд $1 - 1 + 1 - 1 + 1 - \dots$ не сходится,  функция 
$f(x)\z=x-x^2+x^4-x^8+x^{16}-x^{32}+\dots$ не определена при $x=1$.
Однако $f(x)$ сходится при всех положительных вещественных числах $x<1$.
Сходится ли $f(x)$ при $x$, стремящемся к 1 снизу?

\subsection*{Надёжные мигалки}\rindex{Надёжные мигалки}\label{Надёжные мигалки}

Две обычные мигалки мигают одновременно в момент времени $0$
и мигают дальше; при этом у них вместе \emph{в среднем} получается по одному миганию в минуту.
Известно, что они больше никогда не мигнут одновременно (или, что то же самое, отношение их частот иррационально).

Докажите, что после первой минуты (во временном интервале между $0{:}00$ и $0{:}01$) в каждом интервале между $t$ и $t + 1$ минутой будет ровно одно мигание (здесь $t$ --- натуральное число).

\subsection*{Красные и синие игральные кости}\rindex{Красные и синие игральные кости}\label{Красные и синие игральные кости}

У нас есть два набора (красный и синий) из $n$ игральных костей с $n$ гранями на каждой;
на гранях каждой кости стоят числа от $1$ до~$n$.
Мы бросили все $2n$ костей одновременно.
Докажите, что можно выбрать непустое подмножество красных и непустое подмножество синих костей с одинаковыми суммами!
