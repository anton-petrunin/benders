\chapter{Разминка}


\setlength{\epigraphwidth}{.80\textwidth}
\epigraph{Мозг (сущ.) --- приспособление, которым думают, что думают.}{--- Амброз Бирс (1842---1914), Словарь Сатаны}

Начнём с нескольких довольно простых задач для разминки мозга.
Для них почти не потребуется математики, только чуть-чуть логического мышления.

\subsection*{Половина роста}\rindex{Половина роста}


В среднем, в каком возрасте ребёнок достигает половины роста, до которого он дорастёт, когда вырастет?

\subsection*{Шарики в мешочках}\rindex{Шарики в мешочках}

Сколько понадобится шариков, чтобы разложить в 15 мешочков так,
чтобы во всех мешочках было разное число шариков?

\subsection*{Степени двойки}\rindex{Степени двойки}

Сколько людей составляют \emph{дважды две пары двойняшек}?

\subsection*{Катящийся карандаш}\rindex{Катящийся карандаш}

Карандаш с пятиугольным сечением имеет надпись на одной из пяти граней.
Предположим, что наш карандаш катится по столу.
С~какой вероятностью он остановится надписью вверх?

\subsection*{Портрет}\rindex{Портрет}

Посетитель указывает на портрет и спрашивает, кто это. 
«Братьев и сестёр у меня нет,» --- отвечает хозяин, --- «но отец этого человека --- сын моего отца».
Кто изображён на картине?

\begin{addedbytheeditors}
По-русски эта загадка существует даже в стихах.\\ 
\quad В семье я рос один на свете,\\
\quad И это правда, до конца.\\
\quad Отец того, кто на портрете, ---\\
\quad Сын моего отца.
\end{addedbytheeditors}

\subsection*{Странная последовательность}\rindex{Странная последовательность}

Каким должен быть следующий символ в этой последовательности?

\begin{figure}[h!]
\centering
\includegraphics[scale=0.5]{pics/ZYXW}
\end{figure}

\subsection*{Параметр языка}\rindex{Параметр языка}

Для испанского, иврита и польского он равен 1.
Для немецкого и английского --- 7.
Для французского --- 14.
Чему он равен для русского?

\subsection*{Вниманию параскаведекатриафобов}\rindex{Вниманию параскаведекатриафобов}

Правда ли, что 13-е число месяца выпадет на пятницу чаще,
чем на любой другой день недели,
или так только кажется?

\medskip

Теперь пойдут задачи посерьёзнее.

\subsection*{Честная игра}\rindex{Честная игра}

Как сделать равновероятный выбор 50 на 50, подбрасывая гнутую монету?

\subsection*{Кривые на картофелинах}\rindex{Кривые на картофелинах}\label{Кривые на картофелинах}

Даны две картофелины.
Докажите, что на их поверхностях можно нарисовать по замкнутой кривой так, чтобы обе кривые были идентичны как кривые в трёхмерном пространстве.

\medskip

Завершим разминку тремя вероятностными задачами; в них придётся считать.

\subsection*{Победа на Уимблдоне}\rindex{Победа на Уимблдоне}

Временно получив магические способности, вы дошли до финала одиночного разряда Уимблдонского турнира и играете с Сереной Уильямс или Роджером Федерером.
Однако ваши способности не могут продлиться весь матч.
При каком счёте им лучше всего исчезнуть, чтобы максимизировать ваши шансы на победу?

\begin{addedbytheeditors}
Про задачу можно думать так: 
\textit{Есть возможность наколдовать себе произвольный промежуточный счёт игры и далее играть по-честному.
Какой счёт следует наколдовать чтобы максимизировать шансы на победу?}\pr
\end{addedbytheeditors}



\subsection*{Макаронные циклы}\rindex{Макаронные циклы}

100 концов 50-и сваренных длинных макаронин произвольно разбиты на пары и соединены вместе.
Сколько в среднем получится циклов?

\subsection*{Рулетка для ротозеев}\rindex{Рулетка для ротозеев}\label{Рулетка для ротозеев}

Элвин приехал в Лас-Вегас на математическую конференцию и оказался в казино.
У него есть немного времени перед докладом и 105 долларов в кармане.
Он подошёл к рулетке, увидел, что на колесе 38 чисел (0, 00 и от 1 до 36).
Если поставить 1 доллар на какое-то число, то выигрываешь с вероятностью 1/38, получая 36 долларов (взамен своего доллара, который в любом случае забирает казино).
В противном случае он просто теряет свой доллар.

Элвин решил сделать ровно 105 таких однодолларовых ставок, 
у него как раз хватит на это время.
Оцените вероятность того, что Элвин окажется в плюсе.
Скажем, превысит ли эта вероятность 10\%?
