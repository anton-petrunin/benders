\chapter{Новые встречи со старыми друзьями}

\setlength{\epigraphwidth}{.53\textwidth}
\epigraph{Забыть ли дружбу преждних дней\\
И не грустить о ней?}{--- Роберт Бёрнс (1759---1796)}

Как и вино, головоломки могут улучшаться с возрастом, приобретая новые, захватывающие версии, а иногда и лучшие решения для старых версий.
В этой главе представлены головоломки, которые знакомы многим.
Однако даже если вы очень хорошо знаете какие-то из них, 
вы увидите удивительные новые повороты!

Один из самых запоминающихся персонажей головоломок --- логик, который любит отдыхать на Южных Морях.
Если вы верите Мартину Гарднеру (см., например, его первую коллекцию для Scientific American [27]), то этот логик постоянно теряется и вынужден спрашивать как пройти местных жителей.

\subsection*{Трое аборигенов на перекрёстке}

Гарднеровский логик снова поехал на юг и, как обычно, стоит на развилке, желая узнать, по какой из двух дорог можно добраться до деревни.
На этот раз рядом находятся три аборигена, по одному из каждого племени:
из племени непререкаемых правдолюбов,
из племени непререкаемых лжецов
и из племени случайно отвечающих.
Конечно же, логик не знает, к какому племени относится каждый из аборигенов.
Ему разрешается задать только два вопроса с ответом «да» или «нет», каждый вопрос задается только одному аборигену.
Сможет ли он получить необходимую информацию?
А что если ему разрешено задать только один вопрос с ответом «да» или «нет»?

\medskip

Перейдём к известной и изящной геометрической головоломке, которая, к моему стыду, появилась на странице 46 моей предыдущей книги [59] с решением не вполне верным.

\subsection*{Новая встреча с тремя окружностями}

Назовём фокусом двух окружностей пересечение двух их общих внешних касательных.
Таким образом, если три окружности имеют разные радиусы (и ни одна из них не окружена в другой) то они определяют три фокуса (см. рис. 25).
Докажите, что эти три фокуса лежат на одной прямой.

\parit{Замечание.}
Решение из [59] предлагает построить три сферы, каждая из которых имеет одну из данных окружностей как экватор, а затем рассмотреть плоскости, касающейся этих трёх сфер.
Однако Джером Льюис, профессор информатики в Университете Южной Каролины в Упстейте, справедливо указал на то, что такая плоскость существует не всегда!
Например, их нет если две из окружностей большие, и между ними находится меньшая окружность.

Однако доказательство можно спасти, не отказываясь от этой хорошей идеи.
Попробуйте найти способ.

\medskip

Следующая задача является ярким представителем головоломок, основанных на знаниях о знаниях.

