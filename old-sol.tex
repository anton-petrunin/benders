\section*{Источники и решения}

\subsubsection*{Трое аборигенов на перекрёстке}

Этот вариант задачи про логика и аборигенов пришёл ко мне от двух матфизиков, Владаса Сидоравичюса и Сени Шлосмана.
Похоже, что случайный абориген путает все карты, однако решение есть.

Сначала надо позаботиться о том, чтобы второй спрошенный абориген не был из случайного племени.
Это необходимо, ведь за первый вопрос дорогу не узнать, а если второй абориген случайный, то мы вовсе ничего не узнаем.

С другой стороны, этого достаточно, ведь далее можно использовать традиционный вопрос типа «А если бы я спросил вас, ведёт ли первая дорога к деревне, вы бы сказали „да“?»

Чтобы достичь этой цели, вам нужно будет задать аборигену А что-то об оборигенах Б или В, а затем использовать ответ, чтобы выбрать между Б и В.
Вот вариант, который работает: «Верно ли, что Б ответит правду, с большей вероятностью чем В?»

Забавно, что если А ответит «да», то надо выбирать В, а если «нет», то выбираете Б!
Ведь если А говорит правду, вы хотите обратиться к тому, кто меньше всего склонен говорить правду, то есть к лжецу.
Если же А лгун, то вам нужен более правдивый из его спутников, а именно правдолюб.

Конечно же, если А --- случайный, то нет разницы, кого выбрать Б или В.

В колонках Мартина Гарднера отмечалось, что в первоначальной задаче с одним аборигеном логик может дойти до деревни даже если он забыл, какое из слов на местном языке (предположительно «пиш» и «туш») означает «да» и «нет».
Читатели, желающие поразвлечься, могут попытаться аналогично изменить вышеуказанный протокол.

Если случайный аборигенрешает сказать «да» или «нет», подбрасывая монету в уме, то, конечно же, одного вопроса недостаточно.
Однако, если предположить, что он заранее случайно выбирает между правдой и ложью, а затем отвечает логически,
то для этого случая Анупам Джайн из Университета Южной Калифорнии придумал следующий вопрос:

\begin{itemize}
 \item[] \emph{Если я выберу из двух ваших спутников того, чей ответ с наименьшей вероятностью будет совпадать с вашим, и спрошу его, ведёт ли первая дорога в деревню, ответит ли он „да“?}
\end{itemize}

Утверждается, что если ответ «нет», то первая дорога --- правильная, в противном случае вторая.

Ключевой случай возникает, когда логик задаёт этот вопрос случайному отвечающему.
Если случайный абориген решит солгать, то ответ правдолюба на вопрос будет наименее вероятно совпадать с его ответом.
Правдолюб скажет «да», и так как случайный отвечающий решил солгать, он скажет противоположное и, таким образом, ответит «нет».

Если случайный отвечающий решил сказать правду на этот вопрос, то ответ правдолюба будет наименее вероятно совпадать его ответом.
Лжец скажет «нет», и так как случайный отвечающий решил сказать правду, он также скажет «нет».

Если логик обращается к правдолюбу, то ответ лжеца наименее вероятно совпадёт с его ответом,
и он скажет то, что сказал бы лжец: «нет».

Точно так же, если вторая дорога правильная, все ответы будут «да».

\subsubsection*{Новая встреча с тремя окружностями}

Эту задачу иногда называют «окружностями Монжа».

На сайте Cut-the-knot, следующее доказательство приписывается Натану Боулеру из Тринити-колледжа в Кембридже.
Оно использует конусы построенные на окружностях вместо сфер.
Обозначим их через $C_1$, $C_2$ и $C_3$; мы предполагаем, что это прямые конусы, то есть с углом 90° при вершине.
(На самом деле, достаточно, чтобы уголы при их вершинах были одинаковы.)
Каждая пара конусов определяет две (внешние) касательные плоскости, скажем, $P_1$ и $Q_1$ (для конусов $C_2$ и $C_3$), $P_2$ и $Q_2$ (для конусов $C_1$ и $C_3$), и, наконец, $P_3$ и $Q_3$ (для конусов $C_1$ и $C_2$).

Каждая пара плоскостей $P_i$, $Q_i$ пересекается по прямой $L_i$, которая проходит через вершину обоих касаемых конусов, а также через точку, где соответствующие касательные окружности пересекаются.
Таким образом, прямые $L_1$ и $L_2$ проходят через вершину $C_3$,
прямые $L_1$ и $L_3$ через вершину $C_2$,
а $L_2$ и $L_3$ через вершину $C_1$.
Следовательно, эти три прямые копланарны (все они лежат на одной плоскости, определённой тремя вершинами конусов); пересечение этой плоскости с исходной плоскостью окружностей и есть та самая прямая через три фокуса --- победа!

\subsubsection*{Самоубийцы Точкинска}

Эта головоломка про знания о знаниях удивляет своей общностью.
Она досталась мне от Ника Рейнголда из AT\&T Labs.
Различные частные случаи (иногда с формулировками ещё хуже%???
) известны на протяжении многих десятилетий.
Многим из вас наверняка известен вариант когда у всех точки синие, а приезжий сказал --- «Есть хотя бы одна синяя точка».

Впечатляет, что, что не скажи будет катастрофа,
но, что более удивительно, все обречены даже если каждый видит, то что приезжий соврал. 
Мы это скоро докажем, но сначала рассмотрим простой частный случай, показывающий как это работает.

Предположим, что в Точкинске живут всего три жителя, и у всех на лбу синие точки,
а приезжий сказал им, что все точки красные.
Конечно же все видят, что это не так.
Однако первый из них думает следующее.
Предположим, что моя точка красная; тогда второй житель видит мою красную точку и задаётся вопросом, видит ли третий житель две красные точки?
Если да, думает второй, то третий житель поверит приезжему и совершит самоубийство этой же ночью, несмотря на то, что у него точка синяя.
Если же этого не произошло, то второй житель правильно заключит, что третий увидел только одну красную точку, и совершит самоубийство во вторую ночь.
Поскольку ни одно из этих событий не происходит, первый житель заключает, что второй не увидел красной точки.
Он заключает, что у него точка синяя и прощается с жизнью на третью ночь.

Для доказательства общего случая введём некоторые обозначения.
Пусть $S\subset\{0,1,\dots,n\}$ --- множество чисел $x$ с таким свойством: \emph{если у $n$ точкинцев $x$ синих точек, то утверждение приезжего верно}.
По предположению, $S$ --- собственное подмножество; то есть $S$ и его дополнение непусты.
Пусть $b$ --- число синих точек на самом деле;
$b$ может принадлежать $S$, а может и не принадлежать.

Обозначим через $B_i$ множество возможных чисел синих точек, с точки зрения $i$-того точкинца.
Если $b_i$ --- число синих точек, которое он видит на других, то до заявления приезжего, $B_i=\{b_i,b_i\z+1\}$.

Если в какой-то момент $B_i$ сокращается до одного значения, то $i$-ый точкинец обречён.
Это произойдёт в первую же ночь, в случае если  $|B_i\z\cap S|=1$, но это произойдёт и на следующую ночь после любого самоубийства.
Действительно, заметим сначала, что все точкинцы с одинаковым цветом точек ведут себя одинаково, ведь они все видят одинаковое число точек каждого цвета.
Таким образом, если какой-то точкинец видит, что кто-то совершил самоубийство, то (справедливо) считает, что цвет точки этого человека отличается от его собственного. 
Следовательно, он знает свой собственный цвет и обречён.

Для заданных $S$ и $b$, обозначим через $d(b)$ число шагов (увеличений или уменьшений на $1$), необходимое для того, чтобы попасть из $b$ за пределы $S$ или внутрь $S$.
Другими словами, $d(b)$ это наименьшее $k$, такое что $b+k$ или $b-k$ находится в $\{0, 1, \dots, n\}$, но не в $S$ (если $b$ не в  $S$) или в $S$ (если $b$ в $S$).

Например, если $n=10$ и 
$S=\{0,1,2,9,10\}$, то 
$d(0)=3$, 
$d(1)=2$, 
$d(2)=d(3)=1$, 
$d(4)=2$, 
$d(5)=d(6)=3$, 
$d(7)=2$, 
$d(8)=d(9)=1$ и
$d(10)=2$.

Как уже отмечалось, если $d(b)=1$, то самоубийства произойдут уже в первую ночь.
Теперь сделаем более общее утверждение: \emph{первые самоубийства произойдут именно в $d(b)$-ю ночь}.

Доказательство ведётся индукцией по $d(b)$.
Предположим, что это верно при $d(b)<t$, а теперь пусть $d(b)=t>1$.
После $(t-1)$-й ночи, поскольку самоубийств ещё не было, все знают, что $d(b)\ge t$.
Однако, если $d(b)=t$, то $d(b-1)$ либо $d(b+1)$ равно $t-1$.
В первом случае, синеточеники, которые знают, что синих точек $b$ (фактическое количество) или $b-1$, 
могут исключить случай $b-1$, им конец.
Во втором случае, красноточечники могут исключить $b+1$, и значит им придётся покончить с собой.
Наконец, если $d(b-1)\z=d(b+1)=t-1$, то никто не переживёт эту ночь.

Поскольку $d(b)$ не превышает $n$, получаем, что все погибнут к $n$-й ночи.
Также можно увидеть, что они продержатся столько времени только в четырёх крайних случаях:
если $b=0$ и $S=\{n\}$ или $\{0,1,\dots,n-1\}$,
а также если $b=n$ и $S=\{0\}$ или $\{1,2,\dots,n\}$.
Иначе говоря, время выживания максимально если приезжий либо делает наименее информативное верное заявление,
либо высказывает самую безумную ложь.
Также стоит отметить, что определение $d(b)$ не различает 
$S$ и его дополнение --- скажет ли приезжий «$X$» или «Не $X$», точкинцы будут вести себя точно так же.

Можно было бы задаться вопросом, а могут ли точкинцы, зная, что к ним едет кто-то, кто может нарушить явно оправданное правило молчания о цвете точек, организовать какую-то защиту.
Например, все, кто знает, что приезжий соврал, поднимаются и говорят об этом.
К сожалению, немного поразмыслив, можно прийти к выводу, что ни эта, ни какая-либо другая стратегия не спасёт город.

Жизнь точкинцев висит на волоске.
Но как ни странно, сама эта опасность может их спасти --- Стив Бэббидж, менеджер и криптограф из компании Vodafone, указал на то, что если точкинцы подумают, что чьё-то самоубийство было вызвано не знанием цвета, а тем, что кто-то не выдержал жизни в такой смехотворной среде, то при определённых обстоятельствах остальные точкинцы могут пережить вторжение приезжего.

\subsubsection*{Заражённые кубы}

То, что изначально требуется по крайней мере $n^{d-1}$ заражённых единичных кубов (для краткости \emph{участков}), доказывается прямым обобщением двумерного случая, в котором, как мы знаем, периметр заражённой области не увеличивается.
Надо только заменить периметр $(d-1)$-мерной площадью поверхности заражённой области.
Когда новый участок заражается, к поверхности заражённой области добавляется не более $d$ его $(d-1)$-мерных граней,
и в то же время по крайней мере $d$ граней удаляются (те, что разделяют новый участок от уже заражённых соседей).
Таким образом, эта площадь не увеличивается.
В самом конце, она равна площади поверхности большого куба, что составляет $2d \times n^{d-1}$.
Поскольку каждый участок имеет $2d$ граней единичной площади,
если изначально заражено $k$ участков, то начальная площадь поверхности не может быть превышать $k \times 2d$, 
Отсюда $k$ не меньше $n^{d-1}$.

Однако на этот раз выбрать начальные $n^{d-1}$ заражённых участков совсем не просто.
Мэтт Кук и Эрик Уинфри из Калтеха нашли способ, который, по их мнению, работает, но не смогли это доказать;
их коллега Лен Шульман придумал замечательное доказательство, приведённое ниже (присланное мне Уинфри).

Начнём с построения Мэтта и Эрика.
Обозначим участки векторами $(x_1 , x_2 , \dots , x_d )$, где $x_i \in \{1, 2, \dots , n\}$, так что два участка соседние, 
если все координаты кроме одной одинаковы, а одна координата различаются на $1$.

Возьмём любое целое число $k$ и заразим все участки такие, что $\sum_i x_i \equiv k\pmod n$.
Эти участки образуют \emph{диагональное подпространство}, оно разбивается на несколько кусков.
Оно располагается очень странным образом, довольно по-разному при разных $k$.
Кажется, ему повезло заразить всё из-за нескольких совпадений.
Картина очень отличается от двумерного случая!

Чтобы доказать, что это работает, Шульман придумал следующую игру,
в ней сила препятствующая заражению, передана демону-противнику, который мешает вам, заражающему.
Выберем $k$ и начнём с заражённых участков, описанных выше.

Демон начинает, помещая вас на участок $x = (x_1, \dots,x_d )$.
Далее, он выбирает координату $i$.
Вам предоставляется право переместится либо вперёд, либо назад вдоль $i$-й координаты (если $x_i$ равна $1$ или $n$, то у вас нет выбора).
Вы выигрываете, если можете достичь некоторой точки $x$, такой что $\sum_i x_i \equiv k\pmod n$;
демон выигрывает, если он сможет заставить вас бесконечно блуждать.

Утверждается, что если у вас есть стратегия, которая гарантирует победу, то большой куб полностью заразится.

Чтобы это доказать, уточним утверждение: если вы можете выиграть, начав с $x$, то сам $x$ тоже будет заражён.
Обратите внимание, что из точки $x$ демон может выбрать любое направление $i$.
Победная стратегия должна работать для всех $d$ таких возможностей.
Это означает, что ваша стратегия также побеждает, если начать с любого из $d$ соседей $x$, к которым вы собирались переместиться.
По индукции (по количеству шагов до победы), все эти $d$ соседей $x$ могут быть заражены, таким образом, и $x$ тоже.
База индукции, когда начальная точка $x$ имеет координаты, суммирующиеся до $k$ по модулю $n$, в этом случае она уже заражена.

Остаётся предоставить вам победную стратегию; Шульман называет то, что следует, \emph{алгоритмом телеги}.
Для любого участка $x$ обозначим через $x^*$ значение $k + \tfrac12 + \sum_i x_i \pmod n$.
После того как демон выберет некоторую координату $i$, если $x_i > x^*$, вы должны уменьшить $x_i$ (таким образом, $x^*$ также уменьшается, хотя, возможно, перепрыгнет от $\tfrac12$ к $n - \tfrac12$).
Если же, с другой стороны, $x_i > x^*$, вы увеличиваете $x_i$, чтобы $x^*$ также увеличилось, или возможно, перепрыгнет с $n - \tfrac12$ до $\tfrac12$.
Стойте --- если вы попали на участок с $x^* = \tfrac12$, то это ваша победа!

Отсюда следует, что ход, предписанный алгоритмом, всегда допустим:
вам не будет нужды ходить на участок с $x_i = 0$ или $x_i = n + 1$, до победы.

Теперь мы утверждаем, что демон не может заставить вас зацикливаться.
Предположим противное, $x$ циклически повторяется.
Пусть $I$ --- множество индексов, выбранных бесконечное количество раз демоном.
Мы можем считать, что вы уже прошли точку, где ни один индекс, не принадлежащий $I$, больше не будет выбран.
Пусть $y$ будет самым большим значением $x_j$, когда-либо встречавшимся для любого $j \in I$.
Пусть $J$ --- множество индексов в $I$, которые в данный момент имеют это максимальное значение $y$.

Если когда-либо произойдёт так, что $x^* > y$, тогда вы будете увеличивать его на каждом шаге, увеличивая $x^*$, пока он не перепрыгнет в $\tfrac12$, и вы выиграете.
Следовательно, должно быть так, что $x^*$ всегда ниже $y$.
Но тогда, когда демон выбирает $j \in J$, $x_j$ должен уменьшаться до $y - 1$.
В результате $J$ в конечном итоге исчезнет, оставив вас навсегда с меньшим максимальным значением $y$.
Такое не может продолжаться бесконечно --- противоречие.

Отсюда следует, что алгоритм телеги позволяет вам выиграть игру, независимо от того, куда вас высадил демон, или как он будет вас ограничивать.
Существование победной стратегии означает, что заразился весь большой куб --- ура!

\subsubsection*{Шляпы и бесконечность}

Оба ответа «Да, стратегия существует» и «Нет, её нет» оказываются верными!
Но разве такое возможно?

По моим данным, это прекрасная загадка со шляпами была разработана совместно Ювалем Габаем и Майклом О'Коннором (тогда аспирантами в Корнеллском университете), однако решение косвенно содержалось в работе Фреда Гальвина из Канзасского университета.
Кристофер Хардин (Колледж Смита) и Алан Д. Тейлор (Колледж Юнион) затем включили её в статью для American Mathematical Monthly [36].
Стэн Уэгон предложил её как задачу недели в Маккалестерском колледже; дополнительные интересные наблюдения об этой и следующей версии были сделаны Харви Фридманом (Огайо Стейт), Хендриком Ленстрой (Университет Лейдена) и Джо Булером (Рид Колледж).
Последний из них, а также (независимо) Мэтт Бейкер из Технологического института Джорджии, рассказали мне о ней.
Всё это только небольшая часть истории --- прошу прощения, если ваше имя было пропущено!

Сначала посмотрим, что если только конечное число заключённых получили красные шляпы.
Тогда все заключённые это увидят, и если они заранее договорились об этом, все угадают «красный» --- и, конечно же, только конечное число ошибётся.

Та же схема применима, и если только конечное число шляп будут синими, или, например, если только конечное число шляп с нечётными номерами будут красными, а конечное число шляп с чётными номерами будут синими.
Можно пойти ещё дальше: если последовательность шляп в конечном итоге периодична, то все могут угадать цвет, как если бы последовательность была периодичной с самого начала.

Другими словами, последовательность цветов шляп можно считать двоичным представлением некоторого вещественного числа $r$ в единичном интервале $[0,1]$, считая (скажем) синий как 1, а красный как 0.
\emph{В конечном итоге периодична} означает, что с какого-то момента некоторое слово из нулей и единиц повторяется в ней бесконечно;
это означает, что $r$ рационально.
Например, последовательность $100101001010101010\dots$, где $01$ повторяется бесконечно, является таким числом; в этом случае число равно дроби $7/12$.
Если бы это произошло, то каждый заключённый с нечётным номером может сказать «синий», а каждый с чётным номером «красный», и все, кроме номеров $3$, $4$, $5$, $6$ и $7$, окажутся правы.

Таким образом, заключённые спасутся, если последовательность шляп рациональна.
Однако нет нужды ограничивать стратегию рациональными числами.
Возможно, последовательность будет отличаться лишь в конечном числе мест от двоичного представления $\pi$, в таком случае заключённые могут согласиться угадывать, как если бы шляпы представляли собой $\pi$ точно.

На самом деле заключённым надо разделить все возможные последовательности шляп на \emph{классы}, имеющие свойство того, что в каждом классе две последовательности отличаются только в конечным числе мест.
Затем заключённые заранее договариваются о представителе из каждого класса, то есть о каком-то конкретном члене класса.
Если наблюдается последовательность из этого класса, то каждый угадывает, как если бы фактическая последовательность была равна согласованному представителю.

Математически говоря, мы определяем две последовательности как (скажем) соседние, если они отличаются только в конечном числе мест.
Заметим, что
(1) любое $r$ --- сосед самого себя;
(2) если $r$ --- соседом $s$, то $s$ --- сосед $r$; и
(3) если $r$ --- сосед $s$, и $s$ --- сосед $t$, то и $r$ --- сосед $t$.
Это означает, что соседство является тем, что математики называют \emph{отношением эквивалентности}.
Это, в свою очередь, означает, что существует разбиение множества всех последовательностей шляп на классы, так что в каждом классе любые две последовательности являются соседними, а любые две последовательности в разных классах соседями не являются.

Пока всё отлично, но сейчас появится трудность.
Большинство этих классов не будут обладать естественным представителем (как например $\pi$);
заключённым придётся договориться о представителе в каждом классе.
Утверждение, что это в принципе возможно сделать, называется \emph{аксиомой выбора}.
Если принять аксиому выбора, то заключённые смогут ей воспользоваться, чтобы выбрать представителя каждого класса.
Затем, удивив шляпы, заключённые поймут, к какому классу принадлежит последовательность шляп (для этого достаточно смотреть только на шляпы заключённых с более высокими номерами, чем их собственные).
Затем они все угадывают, что их собственный цвет шляпы соответствует тому, что было предсказано согласованным представителем класса, и ошибётся только конечное число заключённых.
Дело сделано!

А верна ли аксиома выбора?
Большинство активно работающих математиков предполагают, что это так, однако если основные аксиомы с аксиомой выбора образуют не противоречивую систему, то нет противоречия и без аксиомы выбора.
Так что так же справедливо предположить, что аксиома выбора не выполняется, как и предположить, что она выполняется.

А если аксиома выбора не выполняется, то заключённые окажутся в трудном положении.
Упомянутые выше Хардин и Тейлор, и независимо Харви Фридман, показали, что согласно стандартным аксиомам математики нет решения для заключённых.
Ещё хуже то, что любая стратегия потерпит неудачу, если только, в определённом смысле, который можно сделать математически точным, заключённые не будут невероятно удачливы.
Так что, если есть опасность стать залючённым, то лучше взять с собой аксиому выбора.

Вы верите в аксиому выбора?
Подумайте: множество возможных выигрышных стратегий для нашего предполагаемого решения является произведением всех вышеобсуждаемых классов эквивалентности; если нет решения, это означает, что произведение бесконечного числа непустых (да ещё и бесконечных) множеств пусто.
С другой стороны, знаменитый парадокс Банаха --- Тарского говорит, что используя аксиому выбора, можно разрезать шар на пять частей и собратьиз них два шара идентичных изначальному!

Мой собственный вывод из этой ситуации таков: хотя ни аксиому выбора, ни её отрицание нельзя опровергнуть, любую из них можно сделать смешной.


\subsubsection*{Все правы или все неправы}

Если вы следовали решению предыдущей головоломки, (основанному на аксиоме выбора), то здесь вы в хорошей форме. Заключённые соглашаются, как и раньше, на представителя из каждого класса эквивалентности, и здесь они также соглашаются, кроме того, что они будут угадывать цвет своих шляп исходя из предположения, что количество позиций, в которых последовательность шляп не совпадает с выбранным представителем, чётно.
Как и в случае конечного числа, если это количество чётно, то все заключённые будут правы; в противном случае --- все неправы.

Забавно, что алгебраисты обычно решают эту задачу следующим образом.
Отождествим каждый цвет шляпы с множеством $\{0, 1\}$, как и раньше, которое мы можем рассматривать как двухэлементную группу $\mathbb{Z}_2$.
Сумма $\Sigma$ из бесконечного числа копий $\mathbb{Z}_2$ является множеством всех последовательностей из 0 и 1, в которых только конечное число единиц (синие шляпы); существует естественный гомоморфизм (отображение) из $\Sigma$ в $\mathbb{Z}_2$, который даёт 0, если число единиц чётно, и 1, если нечётно.
Стандартная теорема о расширении в алгебре позволяет нам расширить этот гомоморфизм на произведение $\mathbb{Z}_2$, которое является множеством всех последовательностей $\{0, 1\}$.
Затем заключённые соглашаются угадывать, предполагая, что значение этого гомоморфизма равно (скажем) 0.

Конечно, доказательство теоремы о расширении требует аксиому выбора, поэтому это доказательство в действительности ничем не отличается от других.

Означает ли это, что для версии с «все правы» или «все неправы» бесконечной задачи о шляпах также требуется аксиома выбора?
Я так думал, пока Тина Кэрролл (студентка-математик в Технологическом институте Джорджии) указала мне на гораздо более простое решение, которое не требует аксиомы выбора или какой-либо математики вообще.

Каждый говорит «зелёный»!!

\subsubsection*{Цифры на лбах}

Эта головоломка пришла ко мне независимо от нескольких людей, включая Ногу Алона из Тель-Авивского университета.
Сам Нога многократно демонстрировал, что во многих задачах бывает полезно ввести вероятность, даже если в фромулировке её нет.
В нашей задаче, если предположить, что числа, написанные на лбах, выбираются независимо и равномерно случайным образом, то каждый заключённый отгадает правильно с вероятностью $\tfrac1n$, независимо от того, что он скажет.

Положим, что заключённые пронумерованы от $0$ до $n-1$.
Мы хотим, чтобы вероятность того, что какой-то заключённый угадает правильно, была равна $1$.
Значит нам нужно, чтобы $n$ событий \emph{$k$-й заключённый угадывает правильно} были взаимоисключающими.
Другими словами, ни одно из них не может произойти одновременно с другим.
В противном случае вероятность хотя бы одного успешного исхода была бы строго меньше, чем $\tfrac1n + \tfrac1n + \dots + \tfrac1n = 1$.

Для этого нам следует разделить множество возможных расстановок чисел на $n$ равновероятных сценариев, а затем потребовать чтоб каждый заключённый основывал свою догадку на разных сценариях.
Эта идея подводит к следующему простому решению.
Пусть $s$ будет суммой чисел на лбах заключённых по модулю $n$.
Пусть $k$-й заключённый будет считать, что $s = k$.
Другими словами, он считает, что его собственное число равно $k$ минус сумма оставшихся чисел по модулю~$n$.

В этом случае заключённый $s$-й заключенный будет прав (а все остальные нет).

\subsubsection*{Заключённый дальтоник}

Из-за неприятной ситуации с Майком и Шреком заключённые не могут обеспечить успешную реализацию схемы выше. Но это само по себе не означает, что не найти другое решение.

Тем не менее, мы видели, что успешная схема должна предотвратить возможность угадывания правильного числа любыми двумя заключёнными, и это, возможно, подразумевает проблему для Шрека. Предположим, что существует успешная схема; мы можем считать, что Шрек знает, что должен делать Майк. Тогда, поскольку он видит каждое число, которое видит Майк, на самом деле он знает, когда числа раскрываются, точно то, что сделает Майк. И поскольку он видит число Майка, он может с некоторой вероятностью ($\tfrac1n$) увидеть, что Майк собирается угадать правильно.

В этом случае Шрек должен угадать неправильно, но поскольку он не знает своего собственного числа, у него нет это понять.
(Выстелить в воздух, например, назвав $n + 1$, не сработает, ведь тогда сумма вероятностей успеха не достигнет 1.)
Значит заключённые потерпят неудачу с положительной вероятностью.

\subsubsection*{Числа со шляпами}

Эту головоломку мне отправила Николь Имморлика, постдок в Microsoft Research. Её формулировка (и решение) содержится в статье шести авторов [1], где она используется в явной конструкции для высокодоходных детерминированных аукционов.

На самом деле, заключённые могут гарантировать победу независимо от распределения чисел.
Прежде чем на лбах появятся числа, они выберут порядок, то есть пронумеруют себя от $1$ до $n$ (скажем, по алфавиту).
Когда все увидят числа, $i$-й заключённый (для каждого $i$) заводит новую нумерацию остальных числами $1$ до $i - 1$ и от $i + 1$ до $n$, в порядке чисел на их лбах.
Далее он вычисляет, сколько транспозиций старых номеров требуется, чтобы перевести их на место новых номеров.

Предположим, что позже этот $i$-й заключённый узнал своё число, и оказалось, что оно стоит на $j$-м месте всех $n$ чисел после упорядочивания по возрастанию.
Тогда ему придётся сделать ещё $|i - j|$ транспозиций, чтобы завершить перестановку $\sigma$ от всех $n$ старых номеров к $n$ новым номерам.
Действительно, $i$ и $j$ следует поменять местами, а числа между ними нужно сдвинуть вверх или вниз на одну позицию.

Например, предположим, что $n = 4$ и $3$-й заключённый видит на лбах $1$-го, $2$-го и $4$-го числа $2\pi$, $\pi$ и $4\pi$ соответственно.
Тогда он присваивает им новые номера 2, 1 и 4.
Чтобы получить эти номера из старых 1, 2 и 4, надо поменять местами 2 и 1 --- одна транспозиция.
Если он ставит себя на 3-м месте, то получает перестановку $1234 \to  2134$.
Его собственное число может быть скажем $\pi/2$, и значит он должен был быть первым, а не третьим.
Чтобы получить правильную перестановку  $\sigma  = 1234 \to  3214$, ему надо поменять местами 3 и 1, а затем 2 и 3, то есть нужны ещё две транспозиции.
Конечно же число могло оказаться другим, но этот последний шаг поможет понять рассуждение ниже.

Если $\sigma$ является чётной перестановкой, то есть такой, которую можно реализовать чётным числом транспозиций, то исходная перестановка, $i$-го заключённого, будет чётной, если $|i - j|$ чётно, и нечётной в противном случае.
Конечно, если $\sigma$ нечётна (как в примере), то справедливо обратное.

Итак, $i$-му заключённому следует выбрать красную шляпу, если он насчитает чётное число транспозиций (в его перестановке остальных $n - 1$ чисел) и его собственный старый номер $i$ чётен, или же если он насчитает нечётное число транспозиций, и $i$ нечётно.
В противном случае он выбирает синюю.
В приведённом примере $i$ нечётно, и $3$-й заключённый сделал нечётное количество (одну) транспозицию, поэтому он выберет красную шляпу.

Если перестановка $\sigma$ оказалась чётной, то этот выбор означает, что заключённый $i$ будет выбирать красную шляпу только тогда, когда $i$ и $|i - j|$ оба чётны или когда $i$ и $|i - j|$ оба нечётны --- другими словами, когда $j$ чётно.
Таким образом, в новом порядке каждый заключённый с чётным номером будет носить красную шляпу, а каждый с нечётным --- синюю.

Если же $\sigma$ нечётна (как в примере), то в новом порядке красные шляпы будут надеты на заключённых с нечётными номерами, и в любом случае они победили.

\subsubsection*{Кирпичная стена}

Вопрос об укладке $n$ кирпичей так, чтобы они максимально перевешивались за край стола, был явно поставлен в 1923 году Дж. Г. Коффином в журнале \emph{American Mathematical Monthly} [12].
Однако другие формулировки известны с 1850 года.
Благодаря Мартину Гарднеру этот вопрос стал очень известностным, его используют по всему миру при знакомстве с гармоническим рядом.

Забавно, что несмотря на широко распространённое мнение об обратном, знаменитая \emph{гармоническая стопка} (см. рис. \ref{pic:kirpich1}) далека от оптимального решения, если только, как иногда бывает, головоломка представлена с ограничением одного кирпича на уровень.
Более того, часто отмечается, что при постройке необходимо заранее знать, насколько башня должна выступать за край, и это тоже неверно.

\begin{figure}[ht!]
\centering
\includegraphics[scale=1]{pics/kirpich1}
\caption{Десятикирпичная гармоническая стопка.}
\label{pic:kirpich1}
\end{figure}

Чтобы получить гармоническую башню, заметим что верхний кирпич не может выступать больше, чем на $1/2$ (при единичной длине кирпича) за пределы кирпича под ним.
В этом случае, центр тяжести двух верхних кирпичей находится на $1/4$ кирпича левее края нижнего кирпича, поэтому следующий кирпич сверху не может выступать дальше, чем на $1/4$ над кирпичом, который его поддерживает.
Продолжая таким образом, $k$-ый кирпич сверху, выступает на $\tfrac1{2k}$ над $(k + 1)$-ым, a значит общий выступ составит $\tfrac12+\tfrac14+\dots+\tfrac1{2k}=\tfrac12(1+\tfrac12+\dots+\tfrac1k)=H_n/2$, где $H_n$ --- $n$-я частичная сумма гармонического ряда, что асимптотически равно натуральному логарифму~$n$.

Поскольку гармонический ряд расходится, можно сделать (правильный) вывод, что при наличии достаточного количества кирпичей, можно добиться произвольно большого выступа.
Но если строить башню таким образом, то придётся знать заранее, расстояние на которое мы хотим выйти за край --- оно ограничивается положением самого нижнего кирпича.

\begin{figure}[htb!]
\centering
\includegraphics[scale=1]{pics/kirpich2}
\caption{Максимальный выступ для 19-и кирпичей.}
\label{pic:kirpich2}
\end{figure}

Однако, как часто отмечается, можно добиться большего если использовать несколько кирпичей как противовес другим.
Уже в декабре 2005 года, в статье на обложке \emph{American Journal of Physics} [35], Дж. Ф. Холл отметил, что можно получить примерно вдвое больший выступ (то есть, около $\ln n$) используя стопку кирпичей, в которой каждый выступает над предыдущим, и противовесом к ним служат дополнительные кирпичи.
На самом деле вплоть до 19 кирпичей такие стенки дают максимальный выступ; см. рис. \ref{pic:kirpich2} для случая $n = 19$.
Однако Холл неверно заключил, что эти конфигурации (называемые хребтными из-за того, что кирпичи, поддерживающие крайний кирпич с максимальным выступом, содержат только один на каждом уровне), являются оптимальными в общем случае.

\begin{figure}[htb!]
\centering
\includegraphics[scale=1]{pics/kirpich3}
\caption{Максимальный выступ для 20-и кирпичей.}
\label{pic:kirpich3}
\end{figure}

\begin{figure}[htb!]
\centering
\includegraphics[scale=1]{pics/kirpich4}
\caption{???.}
\label{pic:kirpich4}
\end{figure}

\begin{figure}[htb!]
\centering
\includegraphics[scale=1]{pics/kirpich5}
\caption{Перевёрнутый треугольник разваливается начиная с трёх слоёв.}
\label{pic:kirpich5}
\end{figure}

На самом деле, в своей прорывной статье [47] на Симпозиуме SIAM по дискретным алгоритмам (январь 2006 года --- таким образом, написанной до публикации статьи Холла), Майк Патерсон и Ури Звик доказали, что Холл был прав относительно возможного выступа с помощью спинных стопок.
Однако они также показали, что спинные стопки перестают быть оптимальными для 20 и более кирпичей.
Ещё более поразительно, они представили другую конструкцию, которая обеспечивает экспоненциально лучший выступ, чем ранее считалось возможным.

Наилучшая стенка из 20 кирпичей изображена на рис. \ref{pic:kirpich3};
она лишь незначительно обгоняет соответвенную стенку Холла.
Однако, как видно из рис. \ref{pic:kirpich4}, лучшие конфигурации начинают выглядеть довольно далеко от спинных стопок с увеличением $n$.
Стрелки в верхней части рис. \ref{pic:kirpich4} представляют дополнительный вес, добавленный некоторыми не показанными кирпичами (из разрешённых 100), их положения не определены однозначно.

На самом деле, для очень больших значений $n$ лучший выступ достигается для конфигурациях, которые выглядят как будто вырезаны из обычной кирпичной кладки --- каждый кирпич располагается над границей двух соприкасающихся кирпичей уровнем ниже.
Тем не менее наиболее очевидные конфигурации разваливаются.
В книге «Без ума от физики» Джаргодзки и Поттера [38] --- книга рекомендуется, несмотря на следующую ошибку --- утверждалось, что стабильны перевёрнутые треугольники (один кирпич внизу, затем два, затем три и так далее), но на самом деле они разваливаются начиная с трёх слоёв (см. рисунок 30).  

Ромбы (с одним кирпичом на слое, до определённого момента, и снова с одним) остаются стабильными до семи слоёв, но рисунок 31 показывает, что происходит после этого.

\begin{figure}[htb!]
\centering
\includegraphics[scale=1]{pics/kirpich6}
\caption{Девятислойный ромб разваливается.}
\label{pic:kirpich6}
\end{figure}

Вместо этого Патерсон и Звик сконструировали кирпичные стены, приблизительно параболической формы, как показано на рисунке 32.
Они строятся (и их стабильность доказывается) рекурсивно, складывая то, что они называют $k$-плитами для последовательно возрастающих $k$.
$k$-плита состоит из $2k + 1$ чередующихся слоёв, в каждом из которых по $k + 1$ и $k$ кирпичей.
Выступ, достигнутый параболической кирпичной стеной из $n$ кирпичей, умеет порядок $\sqrt[3]{n}$ (кубический корень).%
\footnote{Мы говорим, что функция $f (n)$ --- в нашем случае выступ достижимый $n$ кирпичами --- имеет порядок $g(n)$ если существуют положительные константы $c$ и $c'$ такие, что $cg(n) < f (n) < c' g(n)$.}

\begin{figure}[htb!]
\centering
\includegraphics[scale=1]{pics/kirpich7}
\caption{Параболическая стена.}
\label{pic:kirpich7}
\end{figure}

Но является ли это наилучшим решением?
Ромбы или перевёрнутые треугольники, если бы они были стабильными, давали бы асимптотически лучший выступ, порядка $\sqrt{n}$ (квадратный корень).
Однако совсем недавно Патерсон и Звик вместе с Ювалем Пересом, Миккелем Торупом и автор этих строк [48] смогли показать, что ни одна конструкция не может добиться лучшего порядка чем $n^{1/3}$.

\begin{figure}[htb!]
\centering
\includegraphics[scale=1]{pics/kirpich8}
\caption{Эта лампоподобная форма стены возможно оптимальна.}
\label{pic:kirpich8}
\end{figure}

Это не означает, однако, что параболические кирпичные стены являются идеальными; они достигают свеса примерно $3/16n^{1/3}$ , но другие конструкции могут достигать значения $cn^{1/3}$ для некоторого более крупного $c$.
Патерсон и Звик считают, что лучшая форма для больших $n$ --- это форма «масляной лампы» на рис. \ref{pic:kirpich8}.

\begin{figure}[htb!]
\centering
\includegraphics[scale=1]{pics/kirpich9}
\caption{???.}
\label{pic:kirpich9}
\end{figure}

Параболическую кирпичную стену нельзя построить кирпич за кирпичом — как и все стабильные конфигурации, показанные выше, она находится на грани устойчивости и не допускает построения кирпич за кирпичом — но немного изменяя параболу, её всё же можно построить с нуля. На рисунке 34 показана модификация, все ещё обеспечивающая свес порядка $n^{1/3}$, которую можно сделать, укладывая кирпичи в указанном порядке.

Конечно же, стоит помнить, что настоящие кирпичи не идеально формированы и далеки от безотказности.
Поэтому не пытайтесь это делать дома.
