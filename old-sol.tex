\section*{Источники и решения}

\subsection*{Трое аборигенов на перекрёстке}

Этот вариант зачачи про логика и аборигенов пришёл ко мне от двух матфизиков, Владаса Сидоравичюса и Сени Шлосмана.
Кажется, что случайный абориген путает все карты, однако задачу решить можно.

Сначала надо позаботиться о том, чтобы второй спрошенный абориген не был из случайного племени.
Это необходимо, так как за один вопрос дорогу не узнать, и при этом если второй абориген случайный, то вы не узнаете ничего больше.

С другой стороны, этого должно хватить, ведь далее можно использовать традиционный вопрос типа «А если бы я спросил вас, ведет ли первая дорога к деревне, вы бы сказали „да“?»

Чтобы достичь этой цели, вам нужно будет задать аборигену А что-то об оборигенах Б или В, а затем использовать ответ, чтобы выбрать между Б и В.
Вот вариант, который работает: «Верно ли, что Б ответит правду, с большей вероятностью чем В?»

Забавно, что если А ответит «да», то надо выбирать В, а если «нет», то выбираете Б!
Ведь если А говорит правду, вы хотите обратиться к тому, кто меньше всего склонен говорить правду, то есть к лжецу.
Если же А лгун, то вам нужен более правдивый из его спутников, а именно правдолюб.

Конечно же, если А --- случайный, то нет значения, к кого выбрать из Б и В вы обратитесь.

В колонках Мартина Гарднера отмечалось, что в первоначальной задаче с одним аборигеном логик может дойти до деревни даже если он забыл, что означает слово каждого на местном языке (предполжительно «да» и «нет»).
Читатели, желающие поразвлечься, могут попытаться аналогично изменить вышеуказанный протокол.

Если случайный абориген решает сказать «да» или «нет», подбрасывая монету в уме, то, конечно же, одного вопроса недостаточно.
Однако, если предположить, что он случайно выбирает между правдой и ложью, а затем отвечает логически,
то для этого случая Анупам Джайн из Университета Южной Калифорнии придумал следующий вопрос:

\begin{itemize}
 \item[] \emph{Если я выберу из двух ваших спутников того, чей ответ с наименьшей вероятностью будет совпадать с вашим, и спрошу его, ведёт ли первая дорога в деревню, он ответит ли он „да“?}
\end{itemize}

Утверждается, что если ответ «нет», то первая дорога --- правильная, в противном случае вторая.

Критический случай возникает, когда логик задает этот вопрос случайному отвечающему. Если случайный отвечающий решил солгать на этот вопрос, то ответ искреннего парня на вопрос будет наименее вероятно соответствовать его искренности. Искренний парень скажет «да», и так как случайный отвечающий решил солгать, он скажет противоположное и, таким образом, ответит «нет». Если случайный отвечающий решил сказать правду на этот вопрос, то искреннего парня ответ будет наименее вероятно соответствовать его искренности. Лжец скажет «нет», и так как случайный отвечающий решил сказать правду, он также скажет «нет».

Если логик обращается к человеку, говорящему правду, то это будет лжец, чья искренность наименее вероятно совпадет с его искренностью, и он скажет то, что сказал бы лжец: «нет».

Точно так же, если Дорога 2 - правильная дорога, все ответы будут «да».
