\chapter{Разминка}


\setlength{\epigraphwidth}{.80\textwidth}
\epigraph{Мозг (сущ.) --- приспособление, которым думают, что думают.}{--- Амброз Бирс (1842---1914), Словарь Сатаны}

Начнём с нескольких довольно простых задач для разминки мозга.
Для них почти не потребуется математики, только чуть-чуть логики.

\subsection*{Половина роста}

В среднем, в каком возрасте ребёнок достигает половины роста, до которого он дорастёт, когда вырастет?

\subsection*{Шарики в мешочках}

Сколько понадобится шариков, чтобы разложить в 15 мешочков так,
чтобы во всех мешочках было разное число шариков?

\subsection*{Степени двойки}

Сколько людей составляют \emph{дважды две пары двойняшек}?

\subsection*{Катящийся карандаш}

Карандаш с пятиугольным сечением имеет надпись на одной из пяти граней.
Предположим, что наш карандаш катится по столу.
С какой вероятностью он остановится надписью вверх?

\subsection*{Портрет}

Посетитель указывает на портрет и спрашивает, кто это. 
«Братьев и сестёр у меня нет,» --- отвечает хозяин, --- «но отец этого человека --- сын моего отца».
Кто изображён на картине?

\subsection*{Странная последовательность}

Каким должен быть следующий символ в этой последовательности?

\begin{figure}[h!]
\centering
\includegraphics[scale=0.5]{pics/ZYXW}
\end{figure}

\subsection*{Параметр языка}

Для испанского, русского или иврита он равен 1.
Для немецкого --- 7.
Для французского --- 14.
Чему он равен для английского?

\subsection*{Вниманию параскаведекатриафобов}

Правда ли, что 13-е число месяца выпадет на пятницу чаще,
чем на любой другой день недели,
или так лишь кажется?

\medskip

Теперь задачи посерьёзнее.

\subsection*{Честная игра}

Как сделать равновероятный выбор 50 на 50, подбрасывая гнутую монету?

\subsection*{Кривые на картофелинах}

Даны две картофелины.
Докажите, что на их поверхностях можно нарисовать по замкнутой кривой так, чтобы обе кривые были идентичны как кривые в трёхмерном пространстве.

\medskip

Завершим разминку тремя вероятностными задачами; в них потребуется \emph{кое-что} вычислять.

\subsection*{Победа на Уимблдоне}

Временно получив магические способности, вы дошли до финала одиночного разряда Уимблдонского турнира и играете с Сереной Уильямс или Роджером Федерером.
Однако ваши способности не могут продлиться весь матч.
При каком счёте им лучше всего исчезнуть, чтобы максимизировать шансы на вашу победу?

\begin{addedbytheeditors}
Про задачу можно думать так: 
\textit{Есть возможность наколдовать себе произвольный промежуточный счёт игры и далее играть по-честному.
Какой счёт следует наколдовать чтобы максимизировать шансы на победу?}
\end{addedbytheeditors}



\subsection*{Макаронные циклы}

100 концов 50 сваренных длинных макаронин произвольно разбиты на пары и соединены вместе.
Сколько получится циклов в среднем?

\subsection*{Рулетка для ротозеев}

Элвин приехал в Лас-Вегас на математическую конференцию и оказался в казино.
У него есть немного времени перед докладом и 105 долларов в кармане.
Он подошёл к рулетке и увидел, что на колесе 38 чисел (0, 00 и от 1 до 36).
Если поставить 1 доллар на какое-то число, то выигрываешь с вероятностью 1/38, получая 36 долларов (взамен своего доллара, который в любом случае забирает казино).
В противном случае, он просто теряет свой доллар.

У Элвина достаточно времени ровно на 105 таких однодолларовых ставок, и он решает так и поступить.
Оцените вероятность того, что Элвин окажется в плюсе?
Скажем, превысит ли эта вероятность 10\%?
