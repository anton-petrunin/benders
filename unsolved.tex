\chapter{Не решённые и только-что решённые}


\setlength{\epigraphwidth}{.80\textwidth}
\epigraph{Мы должны знать --- мы узнаем!
}{--- Давид Гильберт (1862---1943)}

Плохая новость: эта глава может вас сломать.
Но есть и хорошая новость: нерешённая задача совсем не значит, что она неразрешима.
Например, две из нерешённых головоломок моей предыдущей книги были недавно решены.
Первая была особенно известной, и с ней произошло нечто поистине удивительное.

\subsection*{Ангел и дьявол Конвея}\rindex{Ангел и дьявол Конвея}

Ангел летает над бесконечной шахматной доской и время от времени
должен садиться на клетку.
Он может пролететь не более 1000 ходов короля до очередного приземления.

Пока ангел в небе, дьявол, живущий под доской, может уничтожить любую клетку на свой выбор.
На уничтоженную клетку ангел приземлиться не сможет уже никогда.

Сможет ли дьявол добиться того, чтоб ангелу было некуда приземлиться?

\medskip

Покажется невероятным совпадением, что эта открытая тридцать лет задача была внезапно решена независимо и почти одновременно
\emph{четырьмя} людьми из четырёх разных стран \cite{10, 20, 40, 43}.

При этом идеи были по большей части схожими и не опирались на недавно разработанные методы.
Более того, все доказательства строились на наблюдениях, сделанных самим Джоном Конвеем ещё в 1970-х.
Четверо решивших были:
Андраш Мате из Университета имени Этвёша Лоранда в Будапеште,
Брайан Боудич из Университета Саутгемптона,
Оддвар Клостер из SINTEF ICT в Осло
и Питер Гакс из Бостонского университета.

Было давно известно, что ангела силы $p=1$ (который перемещается на один ход короля) можно победить.
Мате и Клостер показали, что ангел силы $p=2$ выигрывает;
Боудич доказал, что ангелу достаточно силы $p=4$,
а Гакс --- что \emph{какой-то} силы $p$ достаточно.

Доказательства оказались достаточно простыми, так что Бела Боллобаш из Кембриджского университета и Университета Мемфиса смог разобрать их на восхитительном часовом докладе в Университете Иллинойса.
Далее следует выжимка из его доклада и статьи Мате, показывающая, что ангел силы 5 выигрывает.

Мы хотим показать, что если ангел выигрывает (в несколько более сильном смысле) против несколько более слабого противника, называемого \emph{добрым дьяволом}, то он сможет выиграть и против изначального (\emph{злого}) дьявола.
Как мы увидим, против доброго дьявола работает удивительно простая стратегия.

Доброму дьяволу запрещено уничтожать клетку, на которую ангел мог бы приземлиться ранее; 
другими словами, нельзя уничтожить клетки на расстоянии не больше $p$ от любой клетки ранее посещённой ангелом.
При этом доброго дьявола мы считаем победившим, если ему удаётся запереть ангела на ограниченной части плоскости (иначе ангел может просто перепрыгивать с одной из ранее посещенных им клеток на другую).

Ангел может выиграть в изначальной игре, если для любого $n$ он сможет уйти на расстояние $n$ --- это легко проверить.
Нам надо показать, что если такое можно проделать с добрым дьяволом, то можно и со злым.

Предположим, что у злого дьявола \emph{есть} стратегия, удерживающая ангела на расстоянии не более $n$ от начальной клетки.
Давайте покажем, как добрый дьявол сможет сделать то же самое.
По данной последовательности ходов ангела построим \emph{сокращённую} последовательность следующим образом.
Пусть $A_1$ --- самая ранняя клетка, посещённая ангелом, с которой он мог бы прыгнуть на последнюю клетку $A_0$.
Удалим все ходы между $A_1$ и $A_0$.
Далее пусть $A_2$ --- самая ранняя клетка, посещённая ангелом, с которой он мог бы прыгнуть прямо на~$A_1$.
Удалим все ходы между $A_2$ и $A_1$.
Продолжая таким образом, мы получаем сокращение исходной последовательности $A_k$, $A_{k-1}, \dots, A_1$, $A_0$, в котором ангел не совершает прыжки на те клетки, которые мог бы посетить раньше.

\begin{figure}[bt!]
\centering
\includegraphics[scale=1]{pics/angel}
\caption{Стена отмечена чёрным, ангел --- жирным шрифтом, добрый дьявол --- курсивом.}
\label{pic:angel}
\end{figure}

Теперь заставим доброго дьявола реагировать на данную последовательность ходов ангела так, как это сделал бы злой дьявол в ответ на сокращённую последовательность, с одним изменением --- если требуется съесть недозволенную клетку (или нужная клетка уже съедена), то вместо этого добрый дьявол съедает любую дозволенную.
Легко показать, что если данная последовательность работает против доброго дьявола (то есть ангелу не приходится садиться на съеденную клетку), то сокращение этой последовательности срабатывает против злого.
То есть если ангел сможет уйти на расстояние $n$, играя против доброго дьявола, то он сможет сделать то же самое и против злого, а значит, сможет выиграть.

Таким образом, мы свели задачу к убеганию от доброго дьявола, а сделать последнее очень просто ---
ангел может даже позволить себе бегать, не прыгая, по лабиринту из несъеденных клеток.
Он начинает на клетке, нижний левый угол которой находится в начале координат,
и мысленно рисует стену вдоль оси $y$.
Каждый раз, когда добрый дьявол съедает клетку, ангел мысленно обводит её стеной.
На каждом ходу ангел бежит касаясь стены левой рукой.
В основном он бежит на север, но иногда ему приходится бежать на юг, чтобы обойти какой-то участок стены.
Однако силы $5$ хватает для того, чтобы продвигаться на север в среднем на 1 клетку за ход.
В частости, ангел силы $5$ сможет уйти произвольно далеко.
Проделав дополнительную работу, можно убедиться, что хватит и силы $2$.
На рис. \ref{pic:angel} показан возможный путь ангела силы~$2$.

Стратегия ангела, описанная выше, конечно же, \emph{не сработает} против злого дьявола, который может, например, подготовить ловушку для ангела далеко по оси $y$ --- дьявол заманит ангела в конец полуострова, окружённого морем съеденных клеток, а затем отрежет его от берега.
Добрый дьявол не может так сделать, ведь ему нельзя уничтожить вход на полуостров после того, как ангел туда прошёл.

Хотя построение выше довольно прямолинейно, объяснить как стратегия против доброго дьявола превращается в стратегию против злого довольно трудно.
Возможно, что поэтому головоломка не решалась так долго, а ещё это подверждает, что сокращения --- мощный инструмент.

\begin{addedbytheeditors}
Решение Оддвара Клостера \cite{40} более конструктивно.
Мы опишем его идею, полное доказательство можно найти в оригинальной статье.

Во многом стратегия решения схожа с приведённой, но нам \emph{не} потребуется добрый дьявол.
Как и раньше ангел бежит вдоль воображаемой стены, держа её слева от себя.
Изначально стена идёт вдоль оси $y$, но она будет обновляться после каждого хода дьявола.

Каждый раз, когда дьявол съедает клетку, ангел обследует доску вокруг продолжения пути, выясняя, нет ли там ловушек.
Если необходимо, он изменяет несколько отрезков стены так, чтобы за стену попало побольше съеденных клеток.
(Если бы дьявол не ел клетки справа от стены, то ангел шёл бы всё время на север.)
То, что раз попало за стену, остаётся там навсегда.
Этот процесс надо организовать таким образом, чтобы после каждого обновления пути правая клетка каждого будущего сегмента пути почти всегда оставалась несъеденной.
Тогда ангел сможет двигаться бесконечно, обновляя путь.

Неформально, правило обновления можно описать так:
ангел стремится сделать будущую часть пути как можно короче, допуская увеличение её длины, если он обходит справа одну съеденную клетку на каждые два добавленных отрезка.
При соблюдении этого условия он хочет оставить за стеной как можно больше съеденных клеток.
%Каждый раз, когда дьявол съедает клетку, ангел перебирает всевозможные изменения линии стены.
%При этом рассматриваются только такие перемещения, в которых сдвиг $k$ съеденных клеток влево увеличит длину путь ангела не более чем на $2k$, а из всех таких выбирается то, которое перемещает наибольшее количество клеток.
%%Затем ангел делает два хода короля вдоль стены, держа её слева от себя.
%(Если бы дьявол не блокировал новые клетки, ангел шел бы на север бесконечно долго.)
%Если при этом в пути ангела встречаются "слепые кишки", по ним можно не ходить.
\pr
\end{addedbytheeditors}


\medskip

Следующей головоломке ещё далеко до полного решения,
однако до недавнего времени про неё вовсе ничего не было доказано.

\subsection*{Затор}\rindex{Затор}

Вершины бесконечной решётки на плоскости выбираются независимо с
фиксированной вероятностью $p\in (0,1)$.
В каждую из выбранных вершин
помещают автомобиль, направленный либо на север, либо на восток;
в каждом случае направление выбирается независимо подбрасыванием монеты.

Движение регулируется светофорами, которые включают поочерёдно:
«зелёный-восточный» и «зелёный-северный».
При включённом
зелёном-восточном каждый автомобиль, направленный на восток, правая
соседняя вершина от которого не занята, перемещается в эту вершину;
остальные (в том числе заблокированные другим восточным автомобилем),
остаются на месте.

Когда включается зелёный-северный, каждый незаблокированный
автомобиль, направленный на север, переезжает на следующий перекрёсток в
северном направлении.

Эксперименты показывают, что если $p$ меньше определённого
критического значения $p_0$, то автомобили постепенно разъедутся
(при этом каждый автомобиль будет иметь предельную скорость,
равную скорости автомобиля, который вовсе не блокируется).
Но когда
$p> p_0$, происходит обратное: автомобили попадают в безнадёжный
затор, то есть каждый автомобиль совершает лишь конечное число переездов
и останавливается навсегда.

Эту модель движения транспорта на перекрёстке двух широких односторонних улиц представили О. Бихам, А. А. Миддлтон и Д. Левин в 1992 году \cite{6}.
Её странное поведение привлекло много внимания.
%библиографию можно найти по ссылке http://cui.unige.ch/spc/Bibliography/traffic.html.
%недоступная и незархивированя ссылка

\begin{figure}[htb!]
\centering
\includegraphics[scale=1]{pics/gridlock}
\caption{Свободное движение слева и затор справа.}
\label{pic:gridlock}
\end{figure}

На рис. \ref{pic:gridlock} изображено как выглядят свободные и заторные конфигурации под конец, это то, что обычно появлялось в экспериментах, проводимых Раисой Де Соуза \cite{15}, ныне она преподаёт в Университете Калифорнии в Дэвисе.

Весной 2005 года в Исследовательском институте математических наук в Бёркли Омер Ангел, Эндер Холройд и Джеймс Мартин сделали первый существенный шаг: они доказали существование фазы затора.
Другими словами, при достаточно высокой плотности машин каждая машина совершит лишь конечное число переездов.
Мы не приводим доказательство, но оно весьма изобретательно использует теорию перколяций, и с ним очень стоит ознакомиться \cite{2}.

\medskip

Конечно же, каждый раз, когда решается одна математическая задача, появляются три новые.
Уверен, что следующие красавицы достойны любви и внимания.

\subsection*{Упаковка прямоугольников}\rindex{Упаковка прямоугольников}

Дан конечный набор точек в квадрате, включающий его нижний левый угол.
Разрешается выбрать набор непересекающихся прямоугольников, лежащих в квадрате, левые нижние вершины которых образуют данный набор точек.
Можно ли выбрать прямоугольники так, чтобы их общая площадь была не меньше половины площади всего квадрата?

\begin{figure}[hbt!]
\centering
\includegraphics[scale=1]{pics/square}
\caption{Прямоугольники покрывают больше половины площади.}
\label{pic:square}
\end{figure}

\medskip

Эту сбивающую с толку задачу я услышал больше десяти лет назад от Билла Паллиблэнка (математика и администратора) из IBM, который не помнил, откуда она к нему попала.
С тех пор задача всплывала то там, то сям, но мне не удалось найти более ранний источник.
В июне 2004 года она появилась на веб-странице головоломок IBM \cite{ponder-this}, но так и осталась нерешённой.
Я даже не могу доказать то, что прямоугольники могут покрыть 1\% площади квадрата.

На рис. \ref{pic:square} изображена конфигурация точек вместе с подходящим набором прямоугольников.

\subsection*{Произведения и суммы}\rindex{Произведения и суммы}

Можно ли раскрасить неотрицательные целые числа $\{0,1,2,\dots\}$ конечным числом цветов так, чтобы сумма $x+y$ и произведение $xy$ любых двух целых чисел были разных цветов?

\medskip

Эта задача была предложена Дэвидом Гэлвином, постдоком из Университета Пенсильвании.
Вроде как известен набор из шести чисел, таких, что любые два из них являются произведением и суммой некоторой пары, так что для раскраски потребуется по крайней мере шесть цветов.
С другой стороны, известно, что не существует произвольно больших наборов с указанным свойством.

\begin{addedbytheeditors}
Ответ отрицателен даже для более сильной формулировки --- в любой раскраске конечным числом цветов найдётся монохроматическая тройка $(x,x+y,xy)$.
Это доказал Джоэл Морейра в своей диссертации \cite{moreira}.
\pr
\end{addedbytheeditors}


\medskip

Следующая загадка связана с разновидностью карточной игры в пьяницу, придуманной Борисом Алексеевым из Университета Джорджии.
В неё активно играла команда США недавней математической олимпиады.

\subsection*{Разновидность пьяницы}\rindex{Разновидность пьяницы}

Двум игрокам сдают по некоторому числу карт, сначала \emph{в открытую} (рубашкой вниз).
На каждой карте написано целое число, все числа различны.
В каждом раунде игроки одновременно выкладывают по карте;
старшая карта сбрасывается, а младшая передаётся другому игроку.
Проигрывает тот у кого кончились карты.

При увеличении числа сдаваемых карт,
какова предельная вероятность того, что у одного из игроков будет выигрышная стратегия?

\medskip

Борис (как и я) подозревает, что эта вероятность стремится к нулю,
но мне не кажется, что в этой простой игре будет легко разобраться.
%но анализ этой простой вариации игры в пьяницу кажется сложным.

% From Peter: I don't know what, exactly, I had in mind with the name "Peer Pressure." A better name would have been "War Variant." 
% Игра "War" это вариант «Пьяницы».

\medskip

А вот неожиданная загадка от Стива Хедетниеми из Университета Клемсона:

\subsection*{Покрытие ферзями}\rindex{Покрытие ферзями}

Пусть $f(n)$ --- минимальное число ферзей, которые можно расставить на доске размера $n \times n$ так, чтобы каждая клетка была под ферзём или под боем ферзя.
Верно ли, что $f(n + 1) \geqslant f(n)$ при всех~$n$?

\medskip

Есть уйма головоломок о расстановке шахматных фигур (обычно ферзей или коней) на доске $n \times n$.
Ферзей обычно стараются расставить как можно большее, чтоб ни один не бил другого.
Однако нам нужно наименьшее число ферзей, контролирующих всю доску.
Трудно поверить, что для контроля меньшего числа клеток может понадобиться больше ферзей, однако на большей доске есть больше мест, откуда ферзь может контролировать свои владения, и это нужно учитывать!

\medskip

А вот увлекательная, но на самом деле довольно серьёзная головоломка, которая уже многие годы сбивает с толку специалистов по оптимизации.

\subsection*{Встреча}\rindex{Встреча}

Двое приятелей потеряли друг друга в огромном торговом центре. %The Mall of America (аббревиатура: MOA, с англ. — «Молл Америки») — торговый центр, расположенный в Миннесоте в пригороде Миннеаполиса, неподалеку от аэропорта Миннеаполис/Сент-Пол. Он был открыт в 1992 году и является одним из крупнейших торговых центров мира.
На поиск друг друга в одном магазине у них уходит по 15 минут,
при этом время перемещения от одного магазина к другому ничтожно мало
(торговый центр --- удобно устроенный огромный многоэтажный квадрат).
Они не договорились о месте встречи и не определили заранее, кто будет искать, а кто останется на месте.
Как им следует действовать, чтобы минимизировать ожидаемое время поиска?

\medskip

Если один из них ищет, а другой ждёт на месте, то в среднем потребуется проверить $n/2$ магазинов, здесь $n$ --- число магазинов в торговом центре (предполагается, что оно большое).
Однако наши правила запрещают протокол, нарушающий симметрию;
нельзя, например, чтоб младший искал, а старший сидел на месте.
Если оба ищут, то в среднем потребуется $n$ шагов, до того как они окажутся в одном и том же магазине в одно и то же время и найдут друг-друга.

В 1976 году этот вопрос был (иначе) сформулирован Стивом Альперном из Лондонской школы экономики.
О нём и некоторых других вопросах можно почитать на веб-сайте Ричарда Вебера из Кембриджского университета \cite{weber}.
Вебер и Э. Дж. Андерсон предложили алгоритм, согласно которому каждый из приятелей бросает изогнутую монетку, решая с вероятностью около $0{,}2475$ оставаться на месте или же проверять магазины в случайном порядке, а в случае неудачи повторять процесс каждые 15 минут.
Это приводит к успеху в среднем за $0{,}8289n$ шагов.
Пока никто не придумал ничего лучшего, может быть, что лучшего добиться нельзя.

%\begin{addedbytheeditors}
%\textbf{Редакторам:} Добавил «через каждые 15 минут,»
%\end{addedbytheeditors}


\subsection*{Подкрученный прямоугольник}\rindex{Подкрученный прямоугольник}

Вы, наверное, знаете, что ленту Мёбиуса можно получить из бумажной полоски, склеив её концы с подкруткой на пол оборота.
А какой длины нужна полоска?
Иными словами, какие пропорции прямоугольника оптимальны для склейки ленты Мёбиуса, без растяжения или сгибания?

Дмитрий Фукс и Сергей Табачников представили эту головоломку \cite[Лекция 14]{19} вместе с доказательством, что отношение длины к ширине не может быть меньше $\pi/2 \sim 1{,}57$, и примером для любого отношения больше $\sqrt{3} \sim 1{,}73$.
Однако точный ответ неизвестен.

\begin{addedbytheeditors}
Задача решена Ричардом Шварцем \cite{schwartz}; отношение обязано превосходить $\sqrt{3}$.
\pr
\end{addedbytheeditors}


\subsection*{Торговые автоматы}\rindex{Торговые автоматы}

Несколько торговых автоматов в местной игровой зоне работают случайным образом,
иногда выдавая несколько штук жевательных резинок за раз, а иногда не выдавая ничего.
Однако в среднем каждый автомат выдаёт одну штуку за раз.
Какова максимально возможная вероятность того, что все $n$ автоматов за раз выдадут больше чем $n$ жевательных резинок?

\medskip

Эта головоломка (сформулированная в терминах независимых случайных величин) принадлежит Ури Фейге из Microsoft Research.
Кажется оптимальным заставить каждый автомат выдавать $n + 1$ жевательную резинку с вероятностью $1/(n + 1)$,
а иначе ни одной.
В этом случае мы получим больше, чем $n$ жевательных резинок, если хоть один автомат сработает.
Это происходит с вероятностью
\[1-(1-\tfrac1{n+1})^{n+1},\]
которая близка к $1 - 1/e \sim 63\%$, при больших $n$.
Пока никто не придумал ничего лучшего.
Сам Фейге доказал, что вероятность получить более чем $n$ жевательных резинок не может превысить $12/13$.

Ну разве может такая задача быть трудной?

\subsection*{Круги на плоскости}\rindex{Круги на плоскости}

Дано множество открытых единичных кругов, которое тысячекратно покрывает плоскость;
то есть, каждая точка плоскости покрывается как минимум тысячью кругами.
Докажите, что круги можно раскрасить в красный и синий цвета так,
чтобы красные и синие круги по отдельности покрывали всю плоскость.

\medskip

Эту замечательную задачу придумал Янош Пах из Нью-Йоркского университета (он же и главный специалист в этом вопросе).
В своей статье \cite{46} он доказал, что для любого симметричного многоугольника $P$ и любого положительного целого числа $r$ существует число $k$, такое что любое $k$-кратное покрытие плоскости параллельными переносами $P$ можно разбить на $r$ покрытий.
Но если многоугольник заменить на круг, то даже при $r = 2$ неизвестно, найдётся ли такое $k$.

Я считаю, что должно хватить $k = 4$.
А вы, что скажете?
