\section*{Источники и решения}

\subsubsection*{Половина роста}

Родители маленьких детей знают ответ: два года!
(То есть между вторым и третьим днём рождения.)
Да, человечек растёт очень нелинейно.

Задача предложена Джеффом Стейфом из Университета Чалмерса в Швеции.

\subsubsection*{Шарики в мешочках}

Потребуется четырнадцать шариков.
Положите пустой мешочек в мешочек с одним шариком, 
далее второй мешочек в третий, с ещё одним шариком, затем третий в четвёртый, с ещё одним шариком, и так далее.
Таким образом в $i$-ом мешочке будет $i-1$ шарик (и $i-1$ мешочков).

Если вы не догадались засовывать мешочки в мешочки, или подумали, что так нечестно, то потребуется $0 + 1 \z+ \z\dots \z+ 14 \z= 15 \times 7 \z= 105$ шариков.

Задача предложена Диком Плотцем из Провиденса, штат Род-Айленд.

\subsubsection*{Степени двойки}

Ответ восемь.
Четвёрка слов «дважды», «две», «пары», и «двойняшкa» может натолкнуть на мысль, что должно получиться $2^4 \z= 16$.
Но двойняшка это всего один человек.

Классическая задачка.

\subsubsection*{Катящийся карандаш}

Мой коллега Лори Снелл подловил меня на этой задачке.
А вы попались?
Похоже, что ответ должен быть $\tfrac15$, но поскольку $5$ нечётно, карандаш будет лежать гранью вниз и ребром вверх.
Таким образом, ответ $0$ или, если хотите, $\tfrac25$, в зависимости от вашего толкования термина \emph{вверх}, но уж всяко не $\tfrac15$.

Эта головоломка приведена в провокационной книге Чамонта Ванга \cite{58}.

\subsubsection*{Портрет}

Это древняя загадка;
она приводится в классической книге Рэймонда Смаллиана \cite{55}.

«Сын моего отца» может означать лишь самого хозяина, поскольку у него нет ни братьев ни сестёр.
Значит на портрете сын хозяина.

\subsubsection*{Странная последовательность}

Эту загадку переслал мне Кит Кохон, юрист из Агентства по охране окружающей среды.
Это начало алфавита в обратном порядке, то есть ZYXW, но буква Z повёрнута на 90° (вправо или влево), и каждая последующая буква повёрнута на дополнительные 90°.
Следующим символом, следовательно, должна быть повёрнутая буква V, то есть < или >.

\subsubsection*{Параметр языка}

Ответ семь (seven).
Эта загадка слегка математическая;
она придумана Тиной Кэрролл, аспиранткой Технологического института Джорджии. 
Каждое число это первое многосложное натуральное число в данном языке.

\begin{addedbytheeditors}
Наверно лучше переделать вопрос про русский язык и лучше добавить татарский=турецкий=2, грузинский и финский=1, японский=6, китайский=?, чеченский=16.
\end{addedbytheeditors}


\subsubsection*{Вниманию параскаведекатриафобов}

Удивительно, но правда.
Насколько мне известно, это было обнаружено Банкрофтом Брауном (профессор математики в Дартмутском колледже, как и автор этих строк), который привёл свои расчёты в журнале American Mathematical Monthly \cite{11}.
На это мне указал мой нынешней коллега Дана Уильямс.

Не трудно проверить, что из 4800 месяцев в 400-летнем цикле григорианского календаря 13-е число выпадает на пятницу 688 раз.
Воскресенье и среда приходятся по 687 раз, понедельник и вторник по 685, а четверг и суббота только по 684.
При подсчёте нужно помнить, что годы, кратные 100, не являются високосными, если только (как 2000 год) они не делятся на 400.

Происхождение суеверия относительно пятницы 13-го обычно связывается с датой приказа, отданного французским королём Филиппом IV (Филиппом Красивым), о разгроме ордена храмовников.

Потренировавшись, можно научиться определять день недели любой даты в истории, даже учитывая прошлые календарные сложности
(по крайней мере, человеку, подобному глубокоуважаемому Джону Конвею из Принстонского университета).
Для более ленивых смертных, ориентированных на настоящее время, полезно помнить, что в любом году
04.04, 06.06, 08.08, 10.10, 12.12, 09.05, 05.09, 07.11, 11.07 и последний день февраля выпадают на один и тот же день недели.
(Это ещё легче запомнить, если вы играете в крэпс ежедневно с 9 до 5.)
Этот день недели --- среда для 2007 года;
перед невисокосным годом он сдвигается на один, и на два перед високосным.

\subsubsection*{Честная игра}

Подбросьте гнутую монету \emph{дважды} в надежде получить орёл и решку.
В случае если сначала выпадет орёл, будем считать, что выпал «ОРЁЛ»;
если сначала выпадет решка, считаем что выпала «РЕШКА».
Если получим две решки или два орла, то придётся повторить.

Мне напомнил об этой головоломке Тамаш Ленгель из Маккалестерского колледжа;
её решение приписывается покойному великому математику и пионеру информатики Джону фон Нейману и иногда называется «трюком фон Неймана».
Оно основано на том, что даже если монета гнутая, последующие броски являются (по крайней мере должны быть) независимыми событиями.
Конечно же придётся предположить, что гнутая монета может приземлиться на любую сторону!

Вышеупомянутую схему можно улучшить, уменьшив среднее число бросков.
Например, получив орёл-орёл при первой паре бросков и решку-решку при второй, можно считать результат «ОРЛОМ» (тогда конечно же решку-решку, за которой следует орёл-орёл, надо считать «РЕШКОЙ»).
Возможны и другие улучшения.
Статья Шербана Наку и Ювала Переса \cite{44} выдавливает последнюю каплю из минимизации ожидаемого числа бросков, независимо от вероятностей получения орла и решки.

Кстати сказать, последние годы вопрос извлечения честных %безпристрастных???
случайных битов из различных ненадёжных случайных источников становится важным в теории вычислений и является предметом многих научных работ и существенных прорывов.

\subsubsection*{Кривые на картофелинах}

Рассмотрите пересечение картофелин!
Другими словами, представьте, что каждая картофелина это призрак и воткните одну в другую.
Пересечение их поверхностей будет кривой на каждой из них; их и следует нарисовать.

Эту милую головоломку можно найти (среди прочего) в книге \cite{5}.

\begin{addedbytheeditors}
Несмотря на столь простое решение, точная математическая формулировка задачи остаётся не ясной.

Пересечение поверхностей картофелин может быть фракталом не содержащим замкнутых кривых даже если сами поверхности гладкие.
В случае если поверхности гладкие, картофелины легко расположить так чтобы пересечение было гладкой замкнутой кривой.
Тоже можно сделать и при более слабых предположениях.

Однако, без дополнительных предположений вопрос остаётся открытым \cite{agol};
то есть неизвестно, \emph{содержат ли две произвольные вложенные сферы в евклидовом пространстве пару конгруэнтных замкнутых кривых?} 
Более того, похоже, что вопрос откыт даже если обе вложенные сферы имеют конечную площадь.
Это предположение кажется разумным; как заметил Пер Александерсон,
«Я стараюсь не брать картофель с бесконечной площадью поверхности --- его слишком долго чистить.»
\end{addedbytheeditors}

\subsubsection*{Победа на Уимблдоне}

Кажется очевидным, что лучше всего, если вы выиграли два сета (для победы в
мужском финале требуется выиграть три сета из пяти), в третьем сете --- вы впереди 5:0 по
геймам и ведёте в счёте 40-0 в шестом гейме.
(Возможно, вы предпочтёте подавать в шестом гейме, но если вы подаёте как я, то лучше чтоб подавал ваш соперник --- тогда можно молиться о двойной ошибке, которая принесёт вам победу).

Но стойте, не так быстро!
Этот счёт даст вам три шанса, а можно добыть шесть --- три на вашей подаче и ещё три на подаче Роджера.
Как и раньше, вы выиграли два первых сета, но в третьем сете получили 6:6 по геймам и ведёте 6:0 на тай-брейке.

Амит Чакрабарти из Дартмута предложил ещё одно улучшение, основанное на идее, что традиционно полный счёт теннисного матча включает в себя счёт всех сетов и, если счёт в гейме был 6:6, то также и счёт тай-брейка.
Тогда можно запросить, например, чтобы счёт был 6-0, 6-6 (9999-9997), 6-6 (6-0).
Идея (этически сомнительная, конечно) в том, что пока работала магия, ваш соперник выдохся в тай-брейке второго сета и теперь с большей вероятностью оплошает в один из шести предстоящих матч-пойнтов.

\begin{addedbytheeditors}
Примечания для незнающих правил тенниса.
Матч-пойнт --- ситуация, когда выигрыш всего одного очка приводит к завершению матча.
Тай-брейк --- розыгрыш партии, который
случается при счёте 6:6 в решающем сете и играется до 7 очков или до
преимущества
одного из игроков в 2 очка (то есть, 7:6 ещё не победа на тай-брейке, для победы нужно 8:6, 9:7 и так далее. При этом смена подающего на тай-брейке происходит через каждые две подачи, начиная со второй.
При счёте 6:0 сначала будет ещё одна ваша подача (один матч-пойнт), потом две подачи Роджера (ещё два матч-пойнта), снова две ваши (+2) --- и даже если в этот момент будет 6:5, следующая подача Роджера всё равно будет шестым матч-пойнтом.
\end{addedbytheeditors}

\subsubsection*{Макаронные циклы}

Эта старинная задача передана мне коллегой из Дартмутского колледжа, Даной Уильямс.
Она эквивалентна «Игре ???» на странице 198 в книге Мартина Гарднера \cite{26}.
Нужно вычислить вероятность создания цикла на каждом этапе.
Тогда, используя \emph{линейность матожидания}, можно заключить, что ожидаемое число циклов это сумма полученных вероятностей.

При соединении $i$-го конца берётся конец цепи, и из оставшихся $101 - 2i$ концов, лишь один из них (противоположный конец этой цепи) приводит к циклу.
Следовательно, вероятность того, что ваше $i$-е соединение добавит цикл, равна $1/(101 - 2i)$, и, следовательно, ожидаемое общее число циклов равно $1/99 + 1/97 + 1/95 +\dots + 1/3 + 1/1 = 2{,}93777485\dots$; меньше трёх циклов!

Если у нас $n$ макаронин и $n$ большое число, то матожидание числа циклов близко к половине $n$-го гармонического числа, что примерно равно половине натурального логарифма $n$.

\subsubsection*{Рулетка для ротозеев}

Я услышал эту историю от Элвина Берлекэмпа на конференции
«Gathering for Gardner VII».
Позже она появилась в замечательном разделе «Головоломки» журнала Emissary \cite{3}, весна/осень 2006 года.

Игра в рулетку очень выгодна для казино (американский вариант ещё выгодней европейского, в котором нет двойного нуля).
Ясно, что если повторять невыгодную ставку достаточно долго, то обычно вы окажитесь в проигрыше.
Каждая однодолларовая ставка приносит средний убыток в  $1 - (1/38) \times 36 = 1/19$ доллара, то есть примерно в пять центов.

Однако 105 это не так уж много!
Эльвину достаточно выиграть три раза, чтобы оказаться в плюсе.
В этом случае он получит 108\$ за свои 105\$.
Вероятность того, что он никогда не выиграет, составляет $(37/38)^{105} \approx 0{,}0608$;
выиграет ровно раз, $105 \times (1/38) \times (37/38)^{104} \z\approx 0{,}1725$;
и два раза, $105 \times (104/2) \times (1/38)^2 \times (37/38)^{103} \approx 0{,}2425$.
Таким образом, вероятность оказаться в плюсе равна единице минус сумма этих трёх значений, что составляет $0{,}5242$ --- больше половины!

Конечно же это не означает, что Эльвин может дурачить Лас-Вегас.
Ведь когда ему \emph{не} удаётся получить три победы (а это случается в 48\% случаев!), он теряет как минимум 33 доллара, намного больше, чем 3 доллара прибыли, которую он получает, когда выигрывает ровно три раза.
\emph{В среднем} Эльвин потеряет $105 \times (1/19) \approx 5{,}53$ долларов.

Рассмотрим более жёсткий вариант этой задачи.
Предположим, что у Эльвина есть 255\$, но ему нужно 256\$ чтобы зарегестрироваться на математической конференции.
Тогда лучше всего сделать ставку в 1\$, затем 2\$, затем 4\$, 8\$, 16\$, 32\$, 64\$ и, наконец, 128\$ на красное (или чёрное).
Первый раз, когда он выигрывает, он получает в два раза больше своей ставки и прекращает игру с 256\$, ровно то, что ему нужно.
Он потерпит неудачу только если проигрывает все свои $8$ ставок (и все свои деньги), что происходит с вероятностью всего $(20/38)^8 < 0{,}006$.

Попробуйте сами, если не страшно потерять 255\$.
Так можно посетить казино и в 99\% случаев остаться в плюсе.
Ну а потом лучше бросить играть в азартные игры, очень рекомендую.
