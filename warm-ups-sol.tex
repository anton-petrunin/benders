\section*{Решения и комментарии}

\subsubsection*{Половина роста}

Родители маленьких детей знают ответ: два года!
(То есть между вторым и третьим днями рождения.)
Да, человечек растёт очень нелинейно.

Задача предложена Джеффом Стейфом из Университета Чалмерса в Швеции.

\subsection*{Шарики в мешочках}

Потребуется четырнадцать шариков.
Положите пустой мешочек в мешочек с одним шариком, 
далее второй мешочек в третий, с ещё одним шариком, затем третий в четвертый, с ещё одним шариком, и так далее.
Таким образом в $i$-ом мешочке будет $i-1$ шарик (и $i-1$ мешков).

Если бы вы не догадались засовывать мешочки в мешочки, или подумали, что так не честно, то вам потребуется $0 + 1 \z+ \z\dots \z+ 14 \z= 15 \times 7 \z= 105$ шариков.

Задача предложена Диком Плотцем из Провиденса, штат Род-Айленд.

\subsection*{Степени двойки}

Ответ восемь.
Каждое из четырёх слов «дважды», «две», «пары», и «двойняшкa» натакливает на мысль что должно получиться $2^4 \z= 16$ человек.
Но двойняшка это только один человек.

Классическая задачка.

\subsection*{Катящийся карандаш}

Мой коллега Лори Снелл подловил меня на этой задачке.
А вы попались?
Похоже, что ответ должен быть $\tfrac15$, но поскольку $5$ нечётно, карандаш будет лежать гранью вниз и ребром вверх.
Таким образом, ответ $0$ или, если хотите, $\tfrac25$, в зависимости от вашего толкования термина \emph{вверх}, но уж всяко не $\tfrac15$.

Эта головоломка приведена в провокационной книге Чамонта Ванга \cite{wang}.

\subsection*{Портрет}

Это древняя загадка;
она приводится в классической книге Рэймонда Смаллиана \cite{smullyan}.

Поскольку у него нет ни братьев ни сестёр, «сын моего отца» может означать только самого хозяина.
Значит на портрете сын хозяина.

\subsection*{Странная последовательность}

Эту загадку переслал мне Кит Кохон, юрист из Агентства по охране окружающей среды.
Это начало алфавита в обратном порядке, то есть ZYXW, но буква Z повернутой на 90° (вправо или влево), и каждая последующая буква повернута на дополнительные 90°.
Следующим символом, следовательно, должна быть повернутая буква V, то есть < или >.

\subsection*{Параметр языка}

Ответ семь (seven).
Этот любопытный загадочный вопрос был придуман Тиной Кэрролл, аспиранткой Технологического института Джорджии,
и он слегка математический. 
Каждое число это первое многосложное натуральное число в данном языке.

\subsection*{Вниманию параскаведекатриафобов}

Удивительно, но правда.
Насколько мне известно, это было обнаружено Банкрофтом Брауном (профессор математики в Дартмутском колледже, как и автор этих строк), который привёл свои расчёты в журнале American Mathematical Monthly \cite{brown}.
На это мне указал мой нынешней коллега Дана Уильямс.

Не трудно проверить, что в 688 из 4800 месяцев в 400-летнем цикле григорианского календаря 13-е число выпадает на пятницу.
Воскресенье и среда приходятся по 687, понедельник и вторник по 685, а четверг и суббота только по 684.
Для этого нужно помнить, что годы, кратные 100, не являются високосными, если только (как 2000 год) они не делятся на 400.

Происхождение суеверия относительно пятницы 13-го обычно связывается с датой приказа, отданного французским королем Филиппом IV (Филиппом Красивым), о разгроме ордена храмовников.

Потренировавшись, можно научиться определить день недели любой даты в истории, даже учитывая прошлые календарные проблемы
(по крайней мере, если вы человек, подобный внушающему восхищение Джону Конвею из Принстонского университета).
Для более ленивых смертных, ориентированных на настоящее время, полезно помнить, что в любом году
4/4, 6/6, 8/8, 10/10, 12/12, 9/5, 5/9, 7/11, 11/7 и последний день февраля выпадают на один и тот же день недели.
(Это еще легче запомнить, если вы играете в крэпс ежедневно с 9 до 5.)
Этот день недели --- среда для 2007 года, каждый год он сдвигается на один, и на два --- перед високосным годом.

\subsection*{Честная игра}

Подбросьте гнутую монету дважды в надежде получить орёл и решку.
Если сначала выпадет орёл, назовём результат «ОРЁЛ»;
если сначала выпадет решка, назовите его «РЕШКА».
Если получим две решки или два орла, то придётся повторить.

Мне напомнил об этой головоломке Тамаш Ленгель из Маккалестерского колледжа;
её решение приписывается покойному великому математику и пионеру компьютерных наук Джону фон Нейману и иногда называется «трюком фон Неймана».
Оно основано на том факте, что даже если монета гнутая, последующие броски являются (или по крайней мере должны быть) независимыми событиями.
Конечно же придётся предположить, что гнутая монета может приземлиться на любую сторону!

Если вы хотите минимизировать количество бросков, чтобы принять решение, вышеупомянутую схему можно улучшить. Например, если вы получаете орёл-орёл при первой паре бросков и решка-решка при второй, то можно назвать результат «ОРЛОМ» (конечно же решка-решка, за которым следует орёл-орёл, должен быть «РЕШКОЙ»).
Возможны и другие улучшения.
Статья Шербана Наку и Ювала Переса \cite{nacu-peres} выдавливает последнюю каплю крови из минимизации ожидаемого количества бросков, независимо от вероятностей получения орла и решки.

Кстати сказать, в последние годы, вопрос извлечения честных случайных битов из различных ненадёжных случайных источников становится важным в теории вычислений и является предметом многих научных работ и существенных прорывов.

\subsection*{Кривые на картофелинах}

Рассмотрите пересечение картофелин!
Другими словами, представьте каждая картофелина это призрак и воткните одну в другую.
Пересечение их поверхностей содержит кривые на каждом из них, которые дают решение.

Эту милую головоломку можно найти (среди прочего) в книге \cite{berlekamp-rodgers}.

\begin{addedbytheeditors}
Несмотря на столь простое решение, точная математическая формулировка задачи остаётся не ясной.

Пересечение поверхностей картофелин может быть фракталом не содержащим замкнутых кривых даже если сами поверхности гладкие.
В случае если поверхности гладкие, картофелины легко расположить так чтобы пересечение было гладкой замкнутой кривой.
Тоже можно сделать и при более слабых предположениях.

Однако, без дополнительных предположений вопрос остаётся открытым \cite{agol};
то есть неизвестно верно ли что две вложенные сферы в евклидово пространство содержат пару конгруэнтных замкнутых кривых. 

Если контрпример существует, то одно или оба вложения скорее всего придётся сделать бесконечной площади,
но как заметил Пер Александерсон,
«Я стараюсь не брать картофель с бесконечной поверхностью --- его слишком долго чистить.»
\end{addedbytheeditors}

\subsection*{Победа на Уимблдоне}

Кажется очевидным, что вы должны запросить преимущество в двух партиях с любовью (для победы в мужском теннисе требуется пять партий), и в третьей партии иметь преимущество 5-0 в партиях и 40-love в шестой партии. (Вероятно, вы хотите подавать, но если ваша подача такая же, как моя, вы можете предпочесть, чтобы ваш оппонент подавал в шестой партии, проигрывая 0-40, чтобы вы могли помолиться на двойную ошибку.)

Не так быстро!
Эти решения дают три шанса что вам повезёт и вы выиграете, но можно получить шесть шансов --- три ваших подачи и три подачи Роджера.
Вы все еще хотите иметь преимущество в двух партиях, но сделайте счёт быть 6-6 в третьей партии и 6-0 в вашу пользу, конечно же, в тай-брейкере.

Амит Чакрабарти из Дартмутского университета предложил следующее улучшение, основанное на идее, что традиционно полный счет теннисного матча включает игровые очки всех партий и, если игровой счет был 6-6, счет тай-брейкера. Тогда, например, вы могли бы запросить, чтобы счет был 6-0, 6-6 (9999-9997), 6-6 (6-0). Теория здесь (сомнительная, признано), заключается в том, что пока вы находились под вашим магическим заклинанием, ваш оппонент становился разочарованным и измученным в тай-брейкере второго сета и теперь более склонен совершить одну из шести предстоящих матч-точек.

\subsection*{Макаронные циклы}

Эта старинная задача, переданна мне коллегой из Дартмутского колледжа, Даной Уильямс.
Она эквивалентна «Игре в леске» на странице 198 Шестой книги математических развлечений Мартина Гарднера.
Вам нужно вычислить вероятность создания цикла на каждом этапе, а затем использовать \emph{линейность матожидания}, чтобы заключить, что ожидаемое число циклов это сумма этих вероятностей.

При соединении $i$-го конца, вы берёте конец цепи, и из оставшихся $101 - 2i$ концов, только один (другой конец этой цепи) приводит к циклу.
Следовательно, вероятность того, что ваше $i$-е соединение образует цикл, равна $1/(101 - 2i)$, и, следовательно, ожидаемое общее число циклов равно $1/99 + 1/97 + 1/95 +\dots + 1/3 + 1/1 = 2{,}93777485\dots$; меньше трех циклов!

Если у нас $n$ макаронин и $n$ большое число, то матожидание числа циклов близко к половине $n$-го гармонического числа, что примерно равно половине натурального логарифма $n$.

\subsection*{Рулетка для ротозеев}

Я услышал эту историю от Элвина Берлекэмпа на конференции
«Gathering for Gardner VII».
Позже она появилась в замечательном разделе «Головоломки» журнала Emissary \cite{berlekamp-buhle}, весна/осень 2006 года.

Игра в рулетку очень выгодна для казино (американский вариант ещё выгодней европейского у которого нет двойного нуля).
Как известно, если вы повторяете невыгодную ставку достаточно долго, то обычно оказываетесь в проигрыше.
Каждая однодолларовая ставка приносит средний убыток в  $1 - (1/38) \times 36 = 1/19$ доллара, то есть примерно в пять центов.

Однако 105 это не так уж много!
Эльвину достаточно выиграть три раза, чтобы оказаться в плюсе.
В этом случае он получит 108\$ за свои 105\$.
Вероятность того, что он никогда не выиграет, составляет $(37/38)^{105} \approx 0{,}0608$;
выиграет ровно раз, $105 \times (1/38) \times (37/38)^{104} \approx 0{,}1725$;
и два раза, $105 \times (104/2) \times (1/38)^2 \times (37/38)^{103} \approx 0{,}2425$.
Таким образом, вероятность оказаться в плюсе равна единице минус сумма этих значений, что составляет $0{,}5242$ --- больше половины!

Конечно, это не означает, что Эльвин может дурачить Лас-Вегас.
Ведь когда ему не удается получить три победы, он теряет как минимум 33 доллара, намного больше, чем 3 доллара прибыли, которую он получает, когда выигрывает ровно три раза.
(Учитывая, что это происходит в 43\% случаев.)
В среднем Эльвин потеряет $105 \times (1/19) \approx 5,53$ долларов.

Рассмотрим более жёсткий вариант этой задачи.
Предположим, что у Эльвина есть 255\$, но ему нужно 256\$ чтобы зарегестрироваться на математической конференции.
Его лучший способ действий это делать ставку в 1\$, затем 2\$, затем 4\$, 8\$, 16\$, 32\$, 64\$ и, наконец, 128\$ на красное (или чёрное).
Первый раз, когда он выигрывает, он получает в два раза больше своей ставки и сразу же прекращает, имея ровно те 256\$, которые ему нужны.
Он терпит неудачу только если проигрывает все 8 ставок (и все свои деньги), что происходит с вероятностью только $(20/38)^8 < 0{,}006$.

Попробуйте сами если согласны потерять 255\$ в худшем случае.
Вы можете посетить казино и в 99\% случаев останетесь в плюсе.
А потом бросьте играть в азартные игры, очень рекомендую.
