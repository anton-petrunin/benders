\chapter{Предисловие}

\setlength{\epigraphwidth}{.6\textwidth}
\epigraph{Математика "--- не церемониальный марш по гладкой дороге, а путешествие по незнакомой местности, где исследователи часто рискуют заблудиться.
}{У. С. Энглин
%https://biography.wikireading.ru/57144
}

Эта книга для тех кто любит математику, головоломки и сложные задачи;
для тех кому математический мир кажется упорядоченным логичным и интуитивно понятным, и тех кто готов убедиться в обратном!

Чтобы оценить и решить головоломку, нужно знать математику --- 
иногда требуется знать, что такое точка, прямая, что такое простое число, какова вероятность выпадения двух шестёрок подряд. Однако всего этого не достаточно, самое главное, нужно понимать, что такое \emph{доказательство}.

Вам не пригодится продвинутая математики.
Не потребуются компьютер, калькулятор и учебник по матану,
а вот переключатель сообразительности надо будет поставить на «вкл».
Иной раз окажется, что чем больше курсов по математике вы брали, тем меньше у вас шансов найти решение,
а иногда вы прочтёте и даже поймёте ответ, но не сможете в него поверить.

Сами головоломки приходят со всех уголков мира от людей из самых разных слоёв.
С момента появления моего предыдущего задачника, мне стали присылать ещё больше головоломок, и новых, и старых.
Меня удивило и обрадовало, что к моменту написания этих слов, новая коллекция, по объёму и качеству достигла того же уровня, для которого мне раньше требовалось около двадцати лет.

Есть некоторые отличия с предыдущим задачником.
Теперь задачи должны больше удивлять.
На самом деле некоторые из них взяты из моей статьи «Семь математических головоломок, которые кажутся не правильно услышанными» \cite{winkler-7} для на конференции «Gathering for Gard\-ner VII».
Теперь я уделил больше внимания поиску источников.
В результате кое-где информация о происхождении головоломки может оказаться действительно верной.
Однако, за исключением головоломок собственного сочинения, я могу лишь засвидетельствовать свои «благие намеренья».
По указанию некоторых читателей, я старался больше уделять внимания тому, как догадаться до решения,
но, увы, во многих случаях не смог это сделать достаточно убедительно, а иногда и вовсе не знал как это сделать.

Формулировки головоломок и их решения мои собственные --- на мне ответственность за ошибки и неоднозначности,
а они будут.

Головоломки этой книги должны быть красивы и развлекательны,
иметь простые, изящные решения, которые не просто найти,
иллюстрировать какую-то математическую идею,
и при этом не требовать продвинутой математики.
Прежде всего, я отбирал задачи, которые ломают интуицию и стимулируют мышление.
Но все ли задачи соответствуют всем этим критериям? --- ни в коем случае.
Однако здесь можно найти настоящие шедевры, каждый из которых с лихвой окупит мизерную цену самой книжки.
Взгляните на
«Кривые на картофелинах», стр. \pageref{Кривые на картофелинах}, или
«Рулетку для ротозеев», стр. \pageref{Рулетка для ротозеев}, или
«Любовь в Клептопии», стр. \pageref{Любовь в Клептопии}, или
«Черви и вода», стр. \pageref{Черви и вода}, или
«Дефективный кодовый замок», стр. \pageref{Дефективный кодовый замок}, или
«Имена в ящиках», стр. \pageref{Имена в ящиках}, или
«Хамелеоны», стр. \pageref{Хамел»еоны}, или
«Единообразие бубликов», стр. \pageref{Единообразие бубликов}, или
«Надёжные мигалки», стр. \pageref{Надёжные мигалки}, или
«Красные и синие игральные кости», стр. \pageref{Красные и синие игральные кости}, или
«Уже упал?», стр. \pageref{Уже упал?}, или
«Элис на окружности», стр. \pageref{Элис на окружности}, или
«Монеты на столе», стр. \pageref{Монеты на столе}, или
«Ящик в ящике», стр. \pageref{Ящик в ящике}, или
«Вменяемые мыслители», стр. \pageref{Вменяемые мыслители}, или
«Лемминг на шахматной доске», стр. \pageref{Лемминг на шахматной доске}, или
«Шляпы и бесконечность», стр. \pageref{Шляпы и бесконечность}, или
«Кирпичная стена», стр. \pageref{Кирпичная стена}, или
«Торт-мороженое», стр. \pageref{Торт-мороженое}, или
«Три тени кривой», стр. \pageref{Три тени кривой}, или
«Сложенный многоугольник», стр. \pageref{Сложенный многоугольник}, или...

Немного о формате книги.
Для удобства головоломки грубо классифицированы по областям математики и
разбиты на главы.
Решения представлены в конце каждой главы (кроме последней);
это сделано, в надежде, что читатель успеет подумать прежде чем смотреть на решение --- я не хотел, чтобы это было слишком легко.
Информация о происхождении и источнике каждой головоломки представлена вместе с решением.

Эти головоломки сложные.
Решением любой из них можно гордиться, и в некоторых случаях можно гордится о пониманием решения.

Всего наилучшего, и, как говорят в мире механических головоломок, счастливого разгадывания!

\rightline{Питер Уинклер}
