\section*{Источники и решения}

Это классическая теорема, простая и удивительная; о ней мне напомнил Дэн Ромик, сейчас он в Иерусалимском университете.
В третьем томе «Искусство программирования» \cite{} Дональд Кнут проследил историю этого результата до сноски в книге 1955 года Германа Бёрнера \cite{}.
Бриджет Теннер, студентка знаменитого комбинаторика Ричарда Стэнли из Массачусетского технологического института, недавно обобщила эту теорему \cite{}.
Теорему Бёрнера --- одна из задач, где таинственность и очевидность чередуются при каждой попытке найти решение.

Предположим, что матрица имеет $m$ строк и $n$ столбцов.
Положим, что $a_{ij}$ --- значение $i$-й строке и $j$-м столбце матрицы
после упорядочивания каждой строки (будем считать, что самые маленькие значения находятся слева).
После упорядочивания через $b_{ij}$ значение в той же клетке матрицы.

Нам нужно показать, что $b_{ij} \le b_{ik}$ если $j < k$.
Заметим, что $b_{ik}$ это $i$-й наименьший элемент в старом столбце $\{a_{1k}, a_{2k}, \dots, a_{mk}\}$.
Далее, $a_{i'j}\le a_{i'k}$ поскольку $a_{i'j}$ стоит левее $a_{i'k}$ в той же строке.
В частности, для всякого $a_{i'k}\le b_{ik}$ получаем, что $a_{i'j}\le b_{ik}$.
Мы получили как минимум $i$ элементов из старого $j$-го столбца, которые не превосходят $b_{ik}$.
Значит $i$-й наименьший элемент в старом $j$-м столбце не больше $b_{ik}$;
то есть, $b_{ij} \le b_{ik}$ --- конец доказательства.

Но не кажется ли более убедительным разобрать конкретный пример?
Решайте сами.

\subsection*{Нежелательное раскрытие}

Эта любопытная головоломка была передана мне Ричардом Липтоном из Колледжа вычислительной техники в Джорджии.
Выражение можно проанализировать с точки зрения глубины деревьев, но есть более простой способ:
установите все переменные равными 2!
Поскольку применение закона дистрибутивности не меняет значения выражения,
получаем ограничение на длину любого выражения которое можно получить из него раскрывая скобки.

\subsection*{Хамелеоны}

Борис Шейн, алгебраист из Университета Арканзаса, прислал мне эту головоломку; она может быть довольно древней.
Мне известен случай когда её дали ученику восьмого класса в Харькове,
и ещё другой, когда её дали выпускнику Гарварда, проходившему собеседование в крупной финансовой фирме;
оба с ней справились!

Главное увидеть, что после каждой встречи пары хамелеонов разница между числом особей любых двух цветов не меняется по модулю~3.
Обозначим через $N_R$ количество красных хамелеонов,
а также $N_B$ и $N_G$ аналогично для синих и зеленых хамелеонов;
мы утверждаем, что, например, $N_R - N_B$ имеет одинаковый остаток от деления на~3 после каждой встречи двух хамелеонов, как и до этого.
Это легко проверить, рассмотрев все случаи.
Таким образом, эти различия остаются одинаковыми по модулю 3 навсегда, и поскольку в данной колонии ни одно из этих различий не равно нулю по модулю 3, нам никогда не удастся получить две популяции одного цвета нулевой.

С другой стороны, если разница между числом особей двух цветов (скажем, $N_R - N_B$) является положительным кратным $3$, то можно уменьшить эту разницу, заставив красного хамелеона встретиться с зеленым (или, если зеленых нет, сначала заставив красного встретиться с синим).
Повторяя это, мы добьёмся того, что $N_R = N_B$.
Затем пусть красные встречаются со синими, пока не останутся только зеленые хамелеоны.
Собрав все воедино и отметив, что если две разницы кратны $3$, то третья также должна быть кратной $3$, мы видим, что
\begin{itemize}
\item если все три различия кратны 3, то любой цвет может захватить колонию;
\item если только одна из разниц кратна 3, то оставшийся цвет --- единственный, который может захватить колонию; и наконец,
\item если ни одно из различий не кратно 3, как в данной задаче, то колония не сможет стать монохроматической и останется таковой до тех пор, пока не вмешаются другие обстоятельства (например, рождение или смерть).
\end{itemize}

Эта головоломка была предложена осенью 1984 года на Турнире городов (с числами 13, 15 и 17), как Задача 5 как базовом варианте для старших, так и сложном варианте для младшей.
Турнир городов, из котором вы увидите больше головоломок позже, был основан в 1980 году Николаем Константиновым из Москвы.
В то время уже начинал дуть ветер перемен и начиналась перестройка и гласность, и соревнования по математике были затронуты так же, как и другие аспекты советской жизни.
Константинов был недоволен переменами, ушел из Центрального жюри и организовал турнир среди маленьких городов в сельской России.
Он собрал вокруг себя ядро выдающихся математиков, и успех турнира в конечном итоге привел к тому, что Москва сама стала одним из «городов».
Эта группа также основала Независимый университет Москвы в 1993 году.
Позже эта организация превратилась в МЦНМО.

Турнир пришёл в Польшу и Болгарию, а в 1989 году --- в Австралию, благодаря усилиям Питера Тейлора из Университета Канберры.
В настоящее время Тейлор является исполнительным директором Австралийского математического фонда, под чьим патронатом он выпустил пять книг о турнире.

В 1990 году Энди Лиу привез Турнир в Канаду.
С тех пор он распространился по всему миру, с участниками из Соединенных Штатов, Западной Европы, Азии и Южной Америки.
Английский вариант и решения подготавливаются Андреем Сторожевым и Энди Лиу.

\begin{addedbytheeditors}
\textbf{Редакторам:} Вроде история тургора не соответствует действительности.
\end{addedbytheeditors}

\subsection*{Отсутствующая цифра}

Эту забавную загадку подобрали из колонки головоломок Берлекэмпа и Бюлера в журнале Emissary [3], весна/осень 2006 года,
а они услышали её от теоретика чисел Хендрика Ленстры.
Конечно же можно загуглить "\texttt{2\^{}29}" и увидеть число, а можно ли решить эту задачу в уме, без головной боли?

Ну возможно, вы помните технику из начальной школы, называемую вычетом девяток, когда вы складываете цифры и получаете значение числа по модулю 9 (то есть остаток от деления на 9).
Иногда называемая индуской проверкой или арабской проверкой, она использует факт, что $10 \equiv 1 \pmod 9$, следовательно, $10n \equiv 1n \equiv 1 \pmod 9$ для любого $n$.
Если обозначить через $x^*$ сумму цифр числа $x$, то получим $(xy)^* \equiv x^* y^* \pmod 9$ для любых $x$ и $y$.
В частности, $(2n)^* \equiv 2n \pmod 9$.
Остатки от деления степеней числа $2$ на $9$ начинаются с $2$, $4$, $8$, $7$, $5$, $1$, а затем повторяются; так как $29 \equiv 5 \pmod 6$, $2n \pmod 9 \equiv$ пятое число в этой серии, которое равно $5$.
Теперь сумма всех цифр равна $10 \times 4.5 = 45 \equiv 0 \pmod 9$, поэтому пропущенная цифра должна быть $4$.

И действительно, $2^29 = 536\,870\,912$.
