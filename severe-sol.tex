\section*{Источники и решения}

\subsection*{Торт-мороженое}

Эта замечательная головоломка досталась мне от французского аспиранта Тьерри Мора,
а тот узнал её от своего школьного учителя Томаса Лаффорга.
Головоломка (происхождения которой Лаффорг точно не знает) на самом деле включала в себя ещё один угол, определяющий зазор между дольками торта.
Даже в этом случае, глазурь вернётся наверх после конечного числа операций;
упорные читатели могут в этом убедиться.
Наша формулировка (с нулевым зазором) уже достаточно удивительна и, думается, достаточно сложна.

Если вы решили, что при иррациональном $x$ понадобится бесконечное число операций, то вы совсем не одиноки.
В конце концов, если $n$ операций достаточно, то $n$-ый разрез, должен пройти по границе области покрытой глазурью;
как же смогла эта линия оказаться там, где торт никогда не разрезался?

Однако она там окажется, причина в том, что когда долька переворачивается, её покрытая/непокрытая области  меняются не только местами, но и порядком.

В этой головоломке, как и во многих алгоритмических задачах, полезно переопределить операцию так, чтобы менялось только пространство состояний (в данном случае, область покрытая глазурью), а не сама операция, которая у нас меняется на каждом шагу.
В нашем случае достаточно добавить поворот торта после каждой операции, чтобы разрез всегда приходился на одно и то же место.


\begin{figure}[htb!]
\centering
\includegraphics[scale=1]{pics/tort2}
\caption{Разрезы и перевороты с вращением.}
\label{pic:tort2}
\end{figure}

Будем считать, что $0\degree$ --- направление на север,
$90\degree$ --- на восток и так далее.
Каждый раз будем разрезать по направлениям $0\degree$ и $-x$.
Затем кусок переворачивается через линию $0\degree$ и оказывается между $0\degree$ и $x$.
В это же время остальная часть торта поворачивается по часовой стрелке на угол~$x$.
На рис. \ref{pic:tort2} пунктирные линии показывают разрезы на шагах 1, 2, 3 и 4.

Чтобы понять дальнейшее рассуждение, проще всего думать, что $x$ чуть больше $360\degree/k$ для некоторого целого $k$.
В этом случае первый разрез по дольке произойдёт на $k$-м шаге, после того как торт повернётся на полный оборот.

Пусть $k$ будет наименьшим числом долек, которые надо вырезать, чтобы добраться до конца торта.
Другими словами, $k$ --- наименьшее целое число, большее или равное $360\degree/x$.
Тогда $x \z= y + z$, где $y = 360\degree/k$ и 
\[0\le z <\frac{360\degree}{k-1}-\frac{360\degree}{k}=\frac{360\degree}{(k-1)k}.\]
Конечно же, если $z = 0$, то $x = y = 360\degree/k$, и тогда $k$ операций поместят всю глазурь на дно торта, а ещё $k$ вернут её наверх.
В противном случае, как мы увидим, нам уже не достичь момента, когда вся глазурь окажется снизу.

По мере выполнения шагов, будут появляться граничные линии (между покрытой и непокрытой областями) под углами $0$, $x$, $2x$, $3x, \z\dots , (k - 1)x$, а затем $x - kz$, $2x - kz$, $3x - kz, \dots , (k - 1)x - kz$,
а затем они повторяются.
Действительно, легко проверить, что описанный набор $S$ граничных линий замкнут относительно операции разрез---переворот---поворот.
Одна такая операция переносит $ix$ в $(i + 1)x$, кроме $(k - 1)x$, который переносится в $x - kz$;
она же переносит $ix - kz$ в $(i + 1)x - kz$, кроме $(k - 1)x - kz$, который идёт в $x$.
Тем временем разрез $0\degree$ остаётся на месте.
Отсюда следует, что набор граничных линий всегда является подмножеством $S$.

Уже можно заключить, что конечное число операций вернёт всю глазурь на верх, ведь у нас только $2k - 1$ областей торта между $2k - 1$ линиями из $S$.
Каждая область может быть покрыта или не покрыта глазурью, поэтому число всевозможных конфигураций не превышает $2^{2k-1}$.
Значит процесс зациклится за не более чем $2^{2k-1}$ шагов.
Но обязательно ли он вернётся к начальной конфигурации (все области покрыты глазурью)?
Да, потому что операция обратима.
Если бы процесс зациклился на другой конфигурации $C$, то существовали бы две разные конфигурации, которые приводили бы к $C$, что невозможно.

Однако несложно понять в точности, что происходит.
Между граничными линиями в $S$ есть $k$ областей с углом $x - kz$, которые мы назовём $A_1$, $A_2, \dots , A_k$, и $k - 1$ областей с углом $4z$, которые мы назовём $B_1$, $B_2, \dots , B_{k-1}$.
(См. рис. \ref{pic:tort3} для случая $k = 4$, $x = 93.5\degree$.)

\begin{figure}[htb!]
\centering
\includegraphics[scale=1]{pics/tort3}
\caption{Области при $x = 93{,}5\degree$ и значит $z = 3{,}5\degree$.}
\label{pic:tort3}
\end{figure}

С начала процесса, $A$-области становятся не покрытыми глазурью, по порядку.
После $k$ операций все они становятся не покрытыми.
Затем они снова покрываются глазурью, снова по порядку, до тех пор, пока после $2k$ операций все покроются глазурью.
Тем временем $B$-области также становятся не покрытыми глазурью по порядку.
Поскольку их только $k - 1$, после $k - 1$ шагов все они станут не покрытыми, и все снова покроются после $2k - 2$ шагов.

Значит число шагов, необходимых чтобы покрыть оба типа областей глазурью, является наименьшим общим кратным чисел $2k$ и $2k - 1$, то есть $2k(k - 1)$.
Значит потребуется $2k(k - 1)$ шагов, если конечно $x \ne 360\degree/k$ для некоторого целого $k$, в этом случае нету $B$-областей и достаточно $2k$ шагов.

Рассуждая таким же образом, заключаем, что если все области не покрыты глазурью после $n$ шагов,
то $n$ должно равняться нечётному числу домноженному на $k$
и в то же время нечётному числу домноженному на $k - 1$.
Поскольку ровно одно из чисел $k$ и $k - 1$ нечётно, $n$ чётное и нечётное одновременно,
что невозможно.
Значит, если есть $B$-области то не наступит момент, когда вся глазурь окажется снизу.

А вот реакция одного очень известного математика, на эту головоломку: «Мне трудно поверить, что вся глазурь вернётся наверх.
Но я уверен, что если это произойдёт, то будет момент, когда вся глазурь окажется снизу!»

\subsection*{Прыгание и перепрыгивание}

Эту на вид простую головоломку придумал Джеймс Б. Ширер --- математик из IBM,
она появилась на сайте головоломок IBM \cite[апрель 2007]{ponder-this}.
Она вовсе не простая, но раскалывается с помощью пары полезных трюков.

Пронумеруем кувшинки по порядку.
Естественно начать с подсчёта вероятности $p$ того, что лягушка, стартуя с $1$, вернётся в какой-то момент в $0$.
Чтобы никогда не возвращаться, она должна совершить прыжок вперёд (вероятность $1/2$) и не вернуться обратно трижды (вероятность $1 - p^3$).
Таким образом, $(1 - p) = 1/2(1 - p^3)$.
Поделив на $(1 - p)/2$, получим $2 = 1 + p + p^2$, что даёт $p = (\sqrt5 - 1)/2 \sim 0{,}618034$ --- знакомое золотое сечение.

Однако вычислять вероятность того, что лягушка перескакивает конкретную позицию (скажем, $1$) так легко не получается.
Можно было бы начать вычисления с момента, когда лягушка первый раз попадает в $0$, если это не произойдёт, то ей придётся попасть в $1$.
Оказывается легче вычислить вероятность того, что во время конкретного прыжка лягушка перескакивает через кувшинку, которую она никогда не посещала и больше не посетит.

Для того чтобы это произошло, лягушка должна
(а) прыгать вперёд в данный момент,
(б) никогда не возвращаться обратно от того места, где она приземлилась,
и (в) не посещала ту кувшинку, через которую она прыгает, в прошлом.

Мы можем считать, что кувшинка, которую перепрыгивает лягушка, имеет номер 0.
Ключ в том, что если развернуть нумерацию и время, то событие (c) становится независимой копией события (a).
Лягушка будет вести себя ровно также если обратить время и рассматривать кувшинки в обратном порядке;
она с равными шансами прыгает на два вперёд или на один назад.
Событие (c) означает, что достигнув $-1$, она не «вернётся» в $0$.

Таким образом, вероятность того, что все три события произойдут, составляет $1/2 \cdot (1 - p) \cdot (1 - p) = (1 - p)^2 / 2$.
Однако это ещё не вероятность того, что $0$ пропускается, а только вероятность того, что в данный момент лягушка перепрыгивает через пропущенную кувшинку.

Поскольку лягушка, в среднем, перемещается со скоростью $1/2$, она производит $(1 - p)^2$ пропущенных листьев на единицу пройденного пути.
Отсюда следует, что доля кувшинок, на которые она попадает, составляет $1 - (1 - p)^2 \z= (3\sqrt5 - 5) / 2 \sim 0{,}854102$.

\subsection*{Три тени кривой}

Эту головоломку подкинул мне Рик Кенион (Университет Британской Колумбии), который увидел её на двери Джорджа Бергмана в Бёркли. 
Бергман услышал её от Хендрика Ленстры из Бёркли и Университета Лейдена.
По словам Бергмана, Ленстра видел игрушку, состоящую из кубической пластиковой коробки с прорезями, образующими своего рода лабиринт на каждой грани, при этом каждая пара противоположных граней имела одинаковые прорези, так что стержень, перпендикулярный этим двум граням, одновременно проходил через два лабиринта.
Но вместо одного стержня был объект, который, по сути, состоял из трёх взаимно перпендикулярных стержней, соединённых в точке, каждый из которых концами проходил через лабиринты на паре противоположных граней.
Цель игры заключалась в том, чтобы добраться из одной позиции в другую.

\begin{figure}[htb!]
\centering
\includegraphics[scale=1]{pics/tree3}
\caption{Кривая три тени которой деревья.}
\label{pic:tree3}
\end{figure}

Эту головоломку теперь можно купить.
Её придумал Оскар ван Девентер --- блестящий голландский изобретатель, чьи механические головоломки часто воплощают увлекательные математические идеи.

Так или иначе, Ленстра обратил внимание на то, что лабиринт на каждой грани обязан быть деревом,
если бы он имел замкнутый цикл, то часть пластика вывалилась бы.
И он задался вопросом, какова будет область доступных позиций для центральной точки трёх стержней.
Её проекция на каждую грань должна быть деревом, но может ли это множество иметь цикл?
Если да, то такой цикл должен был проецироваться на каждую грань как дерево.
Так вопрос и возник.

Ленстра задумался над этим в феврале 1994 года, или около того.
Бергман расспрашивал об этом разных людей, но безуспешно.
Наконец, в сентябре 1995 года Кевин Буззард, который был постдоком в Бёркли, сообщил Бергману и Ленстре, что вопрос возник раньше в Кембридже (Англия), и был найден пример.
Буззарду этот пример показал Имре Лидер, комбинаторик из Кембриджского университета, который услышал о нём от Джона Рикарда, который его и придумал.
Рикард работал в отделе математики в Кембридже, но теперь программист.

Пример Рикарда, который обладает очень красивой шестикратной симметрией, изображён на рис. \ref{pic:tree3}.

\subsection*{Игроки и победители}

Эта задача ко мне пришла от Алона Орлицкого из Университета Калифорнии в Сан-Диего.
Она иллюстрирует силу коммуникации \emph{от студента к учителю}.

Тристан и Изольда могут пометить команды четырёхзначными двоичными числами от 0000 до 1111 в алфавитном порядке. Затем, когда Тристан узнает, кто играл, он может отправить Изольде 00, 01, 10 или 11 передав \emph{первую позицию в которой отличаются две метки команды}, это может быть первый, второй, третий или четвёртый бит.
Затем Изольда может отправить обратно значение этого бита.

Например, если команда 0011 играла с командой 0110, и одержала победу,
то Тристан отправит Изольде «01», указывая, что метки играющих команд, отличаются во втором бите.
Изольда отправит обратно «1» --- значение второго бита победившей команды.

Эта схема требует отправки трёх битов, на один бит меньше, чем метод, при котором Изольда просто отправляет метку победившей команды.
На самом деле, этот бит даёт экспоненциальное улучшение!
Если у нас $n = 2^{2^k}$ команд, то один метод требует $2^k$ битов, а другой всего $k + 1$.



\subsection*{Подсказка для Чарли}

На самом деле это серьёзная задача по сложности коммуникаций.
Она была рассмотрена Лесом Валиантом из Гарварда в 70-х годах,
а сообщил мне о ней Амит Чакрабарти из Дартмутского колледжа.
Решения этой и более общей задачи приведены в статье Павла Пудлака, Войтеха Рёдля и Йижи Шгалла \cite{49}.

Пусть $x_1, \dots , x_n$ --- биты, представляющие ответы, где (допустим) $1 = \text{«да»}$ и $0 = \text{«нет»}$.
Индексы берутся по модулю $n$.
Алиса отправляет Чарли $x_{-i}$,
а Боб отправляет Чарли все значения $x_a + x_b$ для всех пар $(a, b)$, таких что $a + b = j$;
сумма битов берётся по модулю $2$.
Обратите внимание, что таких пар $n/2$ (после округления вверх, если $n$ нечётное).

Чарли знает $x_{-i}$, а также $x_{-i} + x_{i+j}$, сложив их, он получит $x_{i+j}$.

Выглядит просто, но догадаться сложно.

\begin{addedbytheeditors}
\textbf{Редакторам:} Вроде ответ $\lfloor n/2\rfloor$, но этого НЕ сказано.
Думаю надо добавить в скобки «Однако в этом случае, есть пара $(a,b)$, такая, что $a=b$ и её можно пропустить.»)
\end{addedbytheeditors}

\subsection*{Сближение точек на кривой}

Вопрос о существовании такой кривой задал Оскар ван Девентер, который уже упоминался выше.
Он думал использовать такую кривую в механической головоломке.
Такие кривые действительно существуют, сложнее найти пример для которого можно доказать, что он не обладает вторым свойством.

На рис. \ref{pic:ss-curve} показана такая кривая, которую по своим причинам ван Девентер называет \emph{неконкинкульной}.
%non-conquinculous

\begin{figure}[htb!]
\centering
\includegraphics[scale=1]{pics/ss-curve}
\caption{Кривая со свойством (1), но без сойства (2).}
\label{pic:ss-curve}
\end{figure}

Пунктирный круг на рисунке добавлен для того, чтобы сразу стало ясно то, что кривая удовлетворяет первому свойству.
Чтобы убедиться, что кривая не удовлетворяет свойству (2),
предположим обратное, и пусть $t$ --- первый момент времени, когда один карандаш пройдёт от белого квадрата до белой стрелки,
либо же другой пройдёт от серого квадрата, до серой стрелки.
Какое-то время до момента $t$ два карандаша должны оказаться друг напротив друга, как белый и серый круги.
Но после времени $t$ точкам придётся разойтись, чтобы сойтись вместе, и это приведёт к увеличению расстояния.

А как же нашли эту кривую?
(Если страшно, то пропустите следующие три абзаца.)
Представьте себе, что кривая параметризована параметром $t$;
это означает, что существует непрерывная функция $C$ из $[0, 1]$ в плоскость такая, что $C(0)$ --- один конец кривой (скажем, левый), $C(1)$ --- другой её конец, и $C(t)$ описывает кривую когда $t$ бежит $0$ до $1$.
Успешное управление карандашами в соответствии со свойством (2) означает пару непрерывных функций $f$ и $g$ из $[0,1]$ в себя, где в момент времени $t$ карандаши расположены в точках $C(f(t))$ и $C(g(t))$.
Таким образом, мы хотим, чтобы $f (0) = 0$, $g(0) = 1$, $f (1) = g(1)$, и чтобы расстояние от $C(f (t))$ до $C(g(t))$ неубывало при увеличении $t$.
Вместе $f$ и $g$ описывают кривую от $(0,1)$ до линии $x = y$, в треугольнике с вершинами в $(0,1)$, $(0,0)$ и $(1,1)$.

Для того чтобы это было невозможно, мы бы хотели иметь двойственный путь между прямыми $x = 0$ и $y = 0$, не заходящий на $x = y$, и соотвтестующий парам точек на локально минимальном расстоянии.
Если такая кривая найдётся, то она должна пересечься с нашей кривой $(f, g)$, что вызовет подъём расстояния.

Этот двойственный путь соответсвует другому виду манипуляций с карандашами:
начинём с левого карандаша в левой конечной точке и правого карандаша где-то на кривой,
затем перемещаем обе точки в одном направлении вдоль кривой, пока правый карандаш не достигнет правой конечной точки.
Если при этом ни одну точку нельзя переместить относительно другой без увеличения расстояния между ними, то мы достигли цели.
На данной фигуре карандаши начинают с белого квадрата и серой стрелки, затем движутся вместе, пока не достигнут белой стрелки и серого квадрата, соответственно.

В награду за этот пример ваш автор получил прототип механической головоломки, замечательной как и все творения ван Девентера.

\subsection*{Суммы и произведения}

Эта забавная головоломка всплывала на различных формах в течение многих лет; она появилась в колонке Мартина Гарднера для Scientific American в декабре 1979 года, но по какой-то причине была пропущена, когда эта колонка была включена в антологию \cite{29}.
Кажется удивительным, что столь неопределённая информация достаточна для определения чисел.

Эта головоломка, по сути такая же, как здесь, была предложена независимо Стивом Феннером из Университета Южной Каролины и Биллом Готтесманом, дизайнером и производителем солнечных часов. Предложенный ниже ход рассуждения был предложен Готтесманом.

Начнём с того, что обозначим число Порфирио как $P$, число Саманты как $S$, а неизвестную пару $\{X, Y\}$.
Поскольку Порфирио сначала не знает $X$ и $Y$, оба числа $X$ и $Y$ не могут быть простыми числами,
ни простым числом и его квадратом,
ни простым числом и его кубом.
Более того, ни одно из них не может быть большим простым числом (более $50$), так как в этом случае оно должно было бы быть одним из множителей числа $P$.

Следовательно, поскольку Саманта знает о колебаниях Порфирио заранее,
$S$ не может быть больше $53$ или же равно сумме двух простых чисел.
Все чётные числа исключены; гипотеза Гольдбаха, которая была проверена далеко за пределами $53$, гласит, что каждое чётное число больше $2$ является суммой двух простых чисел.
Это оставляет только числа, превышающие какое-то нечётное составное число на $2$, а именно $11$, $17$, $23$, $27$, $29$, $35$, $37$, $41$, $47$, $51$ и $53$.
Готтесман называет эти 11 чисел «золотыми».
Нам не стоит беспокоиться о суммах простого числа и его квадрата или куба, потому что они все чётные.

Из того, что Порфирий теперь знает $X$ и $Y$ (и из того, что сумма и произведение двух положительных целых чисел полностью их определяет), мы можем вывести, что только один из способов факторизации $P$ даёт золотую сумму. 
Каждое золотое число $G$ для Саманты даёт $(G - 3)/2$ возможных пар $\{X, Y\}$ (например, для $11$ это могут быть $2 + 9$, $3 + 8$, $4 + 7$ или $5 + 6$); и должно быть так, что только одна из этих пар имеет произведение $P$ с желаемым свойством.
Фактически, для $G = 11$ первые две пары выше уже позволяют Порфирию вывести $X$ и $Y$.
В случае $2 + 9$ $P = 18$, которое раскладывается только как $2 \times 9$ (где $2 + 9 = 11$, что является золотым) или $3 \times 6 (дающее S = 13, что не является золотым). В случае 3 + 8 P = 24, который раскладывается как 2 \times 12 (давая 14, не золотое) или 3 \times 8$ (давая обратно $11$, золотое) или $4 \times 6$ (давая $10$, не золотое).
Таким образом, $G$ не может быть $11$.

Следующее золотое число, $17$, действительно работает!
Ровно одна пара слагаемых даёт $P$, у которого единственная золотая сумма множителей равна $17$, а именно $4$ и $13$ (поскольку $2 + 26 = 28$, не золотое).

Остаётся перебрать остальные шесть пар слагаемых для $17$, чтобы убедиться, что ни одна из них не даёт подходящего $P$;
это даёт надежду, что $X$ и $Y$ могут быть $4$ и $13$.
Теперь надо проверить, что $\{4, 13\}$ --- единственный ответ.
Нам придётся повторить весь процесс для остальных девяти золотых чисел, 
и мы убедимся, что ни одно из них не подходит.

Эта головоломка отклоняется от некоторой степени элегантного требования к решению (для решённых головоломок) для включения в эту книгу, но идея того, что слушатель этого короткого разговора может восстановить числа без их суммы или их произведения, является некоторой компенсацией.

\subsection*{Сложенный многоугольник}

Эта головоломка пришла ко мне (как открытый вопрос) от Роберта Вейта из Юго-Восточного университета Индианы, который долго и тщетно пытался её решить.
Мне удалось найти (на мой взгляд) изящное решение, представленное ниже.
Позже выяснилось, что задача была уже решена в статье К. Бёроцки, Г. Кертеша и Э. Макая \cite{9}.

Ответ такой: У каждого нечётноугольника с единчными сторонами площадь не меньше $\sqrt{3}/4$, причём равенство достигается только для треугольника.

Как такое можно доказать?
Утверждение тривиально для треугольников,
естественно возникает искушение воспользоваться индукцией по числу сторон.
Как мы увидим,  многоугольник с четырьмя и более сторонами можно разрезать диагональю на два многоугольника с меньшим числом сторон каждый.
Однако, вообще говоря, новые многоугольники не будут равносторонними.
Таким образом, нужно придумать индукционное предположение применимое к более широкому классу многоугольников, возможно, ко \emph{всем}.

Приведённое ниже индукционное предположение отлично работает, хоть и выглядит топорно.
Важно правильно подобрать параметр.

Обозначим через $\mathbb{O}^n$ множество всех целочисленных $n$-векторов нечётного веса, то есть 
\[\mathbb{O}^n=\left\{\,\vec x=(x_1,\dots,x_n)\in \mathbb{Z}^n\,\middle|\, \sum_ix_i\equiv 1\pmod 2\,\right\}.\]
Чтобы измерить насколько многоугольник $P$ близок к нечётноугольнику с единичными сторонами, мы воспользуемся функцией $u(P)$, определённой как
\[u(P)=1-\min_{\vec x\in \mathbb{O}^n} \left\{\sum_i |e_i-x_i|\right\}.\]
где  $e_1,\dots,e_n$ --- длины сторон $P$.
Заметим, что $u(P)\z\leqslant1$, и $u(P)\z=1$ если $P$ --- нечётноугольник с единичными сторонами, или даже любой многоугольник с целочисленными сторонами и нечётным периметром.
С другой стороны $u(P)\leqslant 0$ если, например, две из сторон многоугольника $P$ имеют длину $\tfrac12$, или же если его периметр является чётным целым числом.

Многоугольник будет считаться \emph{хорошим}, если у него нет вырожденных вершин,
то есть вершин с внутренним углом $180\degree$.
Если $P$ имеет стороны длины больше $1$, то из него можно получить (возможно нехороший) многоугольник $P^*$ со сторонами длины не более одного, разбив каждую длинную сторону $P$ на стороны длины не более~$1$.
При этом, можно добиться того, что $u(P^*)=u(P)$.

Теперь мы переходим к индукционному доказательству того, что площадь $A(P)$ любого многоугольника $P$ не меньше $\tfrac{\sqrt{3}}{4}u(P)$.
Отсюда немедленно последует, что площадь нечётноугольника с единичными сторонами не меньше  $\sqrt{3}/4$.

Ключ к доказательству --- \emph{субаддитивность} функции $u(P)$;
то есть $u(P)\leqslant u(Q)+u(R)$, где $P$ --- многоугольник (возможно нехороший) и диагональ, скажем $D$, делит его на многоугольники $Q$ и~$R$.

Докажем субаддитивность.
Пусть  $e_1,\dots,e_n$ --- длины сторон $P$, а его диагональ $D$ имеет длину $d$. 
Далее, выберем $\vec x\in\mathbb{O}^n$ такой, что $u(P)=1-\sum_i |e_i-x_i|$.

Пусть $I$ --- множество индексов сторон $P$, которые также являются сторонами $Q$, а $J$ --- индексы сторон $R$.
Обозначим через $\vec x|I$ и $\vec x|J$ сужение $\vec x$ на сторону (кроме диагонали) $Q$ и $R$ соответственно;
можно предположить, что $\vec x|I$ имеет нечётный вес.
Пусть $d_0$ --- ближайшее к $d$ чётное целое число, а $d_1$ --- ближайшее к $d$ нечётное целое число, так что $|d_1-d_0|=1$.

Взяв $\vec x|I$ вместе с $D$-координатой $d_0$ для $Q$ и 
$\vec x|J$ вместе с $D$-координатой $d_1$ для $R$, получим
\begin{align*}
u(R)+u(Q)
&\leqslant
1-\sum_{i\in I}|e_i-x_i|-|d_0-d|
+
1-\sum_{i\in J}|e_i-x_i|-|d_1-d|
\leqslant
\\
&\leqslant2-\sum_{i}|e_i-x_i|-1=
\\
&=u(P),
\end{align*}
что и требовалось.

Теперь докажем главное утверждение для \emph{маленьких} треугольников.
А именно, если $T$ --- треугольник со сторонами не длинней $1$, то 
$A(T)\geqslant \tfrac{\sqrt{3}}{4}u(T)$.
Более того, равенство достигается только для равностороннего треугольника с единичными сторонами.

Пусть длины сторон треугольника равны $a$, $b$ и $c$.
Можно предположить, что целочисленный вектор, дающий $u(T)$, либо $(1, 0, 0)$, либо $(1, 1, 1)$.
Поскольку $a < b + c$, в первом случае имеем $u(T)=1-(1-a)-b-c<0$ и значит нечего доказывать.

Во втором случае,
$u(T)=1-(1-a)-(1-b)-(1-c)=2s-2$, где $s=\tfrac12(a+b+c)$ --- полупериметр треугольника.
Можно считать, что $s > 1$, иначе опять нечего доказывать.
В частности, каждое из чисел $a+b$, $b+c$ и $c+a$ больше $1$.

Мы утверждаем, что при данном $s$, треугольник $T$ имеет наименьшую площадь если две его стороны единичные (а третья, соответственно, длины $2s - 2$).

Дабы это увидеть, зафиксируем $a$. 
По формуле Герона (красивое доказательство которой можно найти в \cite{39}),
\[\frac{A(T)^2}{s(s-a)}=(s-b)(s-c)=s^2-(b+c)s+bc.\]
Поскольку сумма $b + c$ постоянна, минимум достигается если $b = 1$ или $c = 1$.

Далее, переобозначив стороны, можно считать что $a = 1$.
Повторив рассуждение, получим две единичные стороны.
Иначе говоря, площадь $T$ не меньше площади треугольника со сторонами $1$, $1$, $2s - 2$, то есть \[\sqrt{s(s-1)(s-1)(2-s)}.\]

Поскольку $s\in(1,\tfrac32]$, получаем, что $s(2-s)\geqslant \tfrac34$.
Значит 
\[A(T)\ge \sqrt{\tfrac34(s-1)^2}=\tfrac{\sqrt{3}}2(s-1)=\tfrac{\sqrt{3}}4u(T),\]
и равенство достигается только при $s=\tfrac32$.

Перейдём к доказательству основного утверждения.
Предполагаем, оно неверно.
Тогда найдётся хороший многоугольник $P$ с минимальным числом сторон, скажем $n$, такой, что $A(P)<\tfrac{\sqrt{3}}4 u(P)$.
Если $n=3$, упорядочим треугольники лексикографически по ($\lceil c\rceil$, $\lceil b\rceil$, $\lceil a\rceil$), где 
$a\leqslant b\leqslant c$ --- длины сторон $P$, и потребуем, чтоб $P$ был минимальным в этом (частичном) порядке.

Предположим сначала, что $n>3$, и пусть $D$ --- любая внутренняя диагональ~$P$.
Такая диагональ найдётся, потому что если $P$ выпуклый, то любые две не соседние вершины можно соединить отрезком внутри $P$.
В противном случае есть вершина $v$ с внутренним углом больше $180\degree$.
Если просканировать внутренность $P$ из $v$, начиная с направления одной из сторон при $v$ и заканчивая другой,
то мы увидим более чем одну из оставшихся сторон.
Там где сканирование переходит от одной такой стороны к другой, мы увидим вершину, с которой и можно соединить $v$ диагональю.

Можно считать, что диагональ разбивает $P$ на хорошие многоугольники $Q$ и $R$, каждый с менее чем $n$ сторонами и, следовательно, каждый удовлетворяющий неравенству теоремы.
По субаддитивности, имеем
\[A(P)<u(P)\leqslant \tfrac{\sqrt{3}}4(u(Q)+u(R))\leqslant A(Q)+A(R)=A(P),\]
--- противоречие.

Остаётся рассмотреть случай $n=3$.
Пусть $A$ --- вершина противоположная стороне $a$, и так далее.
Из рассмотрения маленького треугольника, мы знаем что $\lceil c\rceil>1$.
Если $\lceil b\rceil<\lceil c\rceil$, проведём диагональ от $C$ к любой новой вершине $P^*$;
длина этой диагонали меньше, чем $b$, так как углы, прилегающие к длинной стороне, острые.
Оба треугольника, на которые эта диагональ разбивает $P^*$, лежат ниже $P$ в лексикографическом порядке, применив субаддитивность, приходим к противоречию.

В случае если $\lceil c\rceil=\lceil b\rceil>1$, выберем $P^*$ так, чтобы на стороне $b$ (соответственно на стороне $c$) на расстоянии $1$ от вершины $C$ (соответственно от вершины $B$) была новая вершина $U$ (соответственно $V$).
Проведём две диагонали, одну от $U$ к $V$% (длиной $d$)
, и другую от $V$ к $C$% (длиной $e$)
.
Снова, применив субаддитивность, заключаем, что один из трёх полученных треугольников $BCV$, $CVU$ и $VUA$, должен быть контрпримером.
Однако все эти треугольники предшествуют $P$ в лексикографическом порядке, и это противоречие завершает доказательство.

Обратите внимание, что индукция, вместе со строгим неравенством для маленьких треугольников, влечёт строгость неравенства для любого хорошего многоугольника $P$, если только $P$ не является именно равносторонним треугольником с единичными сторонами.

Уф!
