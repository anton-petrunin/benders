\chapter{Многословное отступление: Игра ХОМО}

%You could leave it out, or keep it in English, but the best option would be to redo the game in Russian!
%To do that, I'd suggest getting a good computer wordlist (e.g., a list of Scrabble-eligible Russian words), and write a program that checks for rare substrings.
%Then pick out the ones that are particularly clever or amusing, and you'll have a great list of Russian HIPEs. 

%Of course it is possible that, since spelling in Russian is more reliable than in English, the game of HIPE doesn't work as well in Russian.
%But why not give it a try?

%---Pete

Рассматривайте эту главу как антракт, --- перерыв, не связанный с математикой. Однако многие математики любят игры со словами, и (по моему личному опыту) эту в особенности. 

Возможно, вы слышали такую загадку: какое английское слово содержит четыре последовательных буквы, которые являются последовательными буквами алфавита?
Ответ: undeRSTUdy.
Вдохновившись этой и другими словесными головоломками, я и ещё трое старшеклассников на летней программе Национального научного фонда 1963-го года начали обстреливать друг друга комбинациями букв, требуя найти слово, содержащее эту комбинацию%
\footnote{Здесь и далее автор, разумеется, «обстреливает» читателя английскими примерами.
Мы постарались, сохранив дух главы, насытить её аналогичными русскими примерами, и даже название игре дали своё собственное. \pr}.

Между буквами, входящими в комбинацию, не должно быть других букв.
Пример:
$\textbf{ТСЧ} = \textrm{оТСЧёт}$,
$\textbf{МПЦ} = \textrm{презуМПЦия}$.
В некоторых загаданных комбинациях были всего две буквы, например, 
$\textbf{ЦД} \z=\textrm{плаЦДарм}$,
$\textbf{ГШ} = \textrm{флаГШток}$ (или другое заимствованное из немецкого языка слово, зинГШпиль).
Начните с \textbf{ОДС} (подсказка: здесь более одного возможного ответа, но один лежит совсем на поверхности). 

Удвоенные буквы 
(\textbf{ЧЧ}, \textbf{ГГ} и даже \textbf{АА}) могут быть интересными, но самые сложные загадки, которые мы отыскали, --- это комбинации из трёх или четырёх букв.
Мы назвали игру в честь комбинации \textbf{ХОМО} $=$ муХОМОр%
\footnote{В оригинале было название \textbf{HIPE} от arcHIPElago.}.
Конечно, ХОМО, несомненно, изобретали и переизобретали тысячи раз, и вы не обязаны использовать наше название --- но всё равно полезно как-то её назвать. 

При придумывании заданий в ХОМО есть очень естественная цель --- искусно замаскировать отгадку;
ещё приятнее, если отгадка окажется распространённым словом, которое, тем не менее, тяжело найти. 

Например, \textbf{ОТОЙ} решается одним из самых распространённых слов, но сможете ли вы его найти?
И что за радость поставить приятеля в тупик, задав ему, например \textbf{СОБЬ}, а потом ещё сказать, что нужно добавить всего одну букву!

В идеале хочется, чтоб решение было единственным (среди нарицательных существительных, имена собственные в этой игре не используются), но это не является строго обязательным требованием, если вы играете с друзьями.

ХОМО-загадки из более чем четырёх букв редко бывают интересными, потому что много известных букв подряд даёт уж очень много информации.
Так что мы ограничились буквосочетаниями из двух, трёх и четырёх букв.
Хотя, например, \textbf{УКВОС} тоже вполне неплохая загадка, не так ли?

В общем, попробуйте свои силы.

Загадки из двух букв:

\begin{multicols}{4}
{\bf
ГБ

ГЗ

ГШ

ЖЧ

ЖЮ

ЗФ

ЙГ

ЙЯ

КП

КФ

ЛФ

МД

МТ

ПД

СЖ

ТЖ

ФЧ

ХХ

ХЪ

ЦН

ЧЧ

ШЦ

ЩМ

ЫИ

ЬЭ

ЮИ
}
\end{multicols}

Ответы:

\begin{multicols}{2}

реГБи

шлаГБаум

зиГЗаг

флаГШток

зинГШпиль

муЖЧина

ЖЮри 

фиЗФак

таЙГа 

саЙГак

маЙЯ 

папаЙЯ

секвоЙЯ

параноЙЯ

аллилуЙЯ

блоКПост

роКФор

саЛФетка

шаЛФей

фиЛФак

аЛФавит

заМДекана

лоМТик

драМТеатр

экспроМТ

проМТовары

креПДешин

СЖигание

СЖатие

СЖимаемость

оТЖим

оТЖиг

борТЖурнал

шкаФЧик

треХХвостка

сверХЪестественность

спеЦНаз

каприЧЧио

мыШЦа

веЩМешок

вЫИгрыш

белЬЭтаж

сЮИта 

флЮИд

трЮИзм
\end{multicols}

Загадки из трёх букв:
\begin{multicols}{4}
{\bf
БСЦ

ДДВ

ДМН

ДСН

ДЧУ

ИЭД

ЛГЕ

МБД

МРУ

НКН

НТВ

НУУ

ОЯБ

ПФР

РАЭ

РГК

РГС

РКК

РРУ

РХЗ

САЭ

ТАЭ

ТИЭ

ТСР

ТСЧ

ТФИ

УСФ

УЧЛ

ФМЕ

ФМИ

ХГР

ХУГ

ХЧЛ

ЦШК

ЯТЫ
}
\end{multicols}

Ответы:


\begin{multicols}{2}

аБСЦисса

преДДВерие

поДМНожество

поДСНежник

преДЧУвствие

полИЭДр

аЛГЕбра

ляМБДа

изуМРУд

баНКНота

глиНТВейн

контиНУУм

нОЯБрь

грейПФРут

тетРАЭдр

оРГКомитет

оРГСтекло

аРККосеканс

аРККотангенс

аРККосинус

коРРУпция

свеРХЗадача

гекСАЭдр

икоСАЭдр

окТАЭдр

пенТАЭдр

гепТАЭдр

пяТИЭтажка

отсрочка

оТСЧёт

мульТФИльм

полУСФера

двУЧЛен

ариФМЕтика

логариФМИрование

треХГРанник

четыреХГРанник

четыреХУГольник

треХЧЛен

спеЦШКола

свЯТЫня
\end{multicols}

Загадки из четырёх букв
\begin{multicols}{4}
{\bf
АТЫН

АЧЕЛ

ВАБР

ГОУГ

ГРИР

ДМНО

ЕПИП

ЗОЦИ

ИМАД

ИФМИ

ИЩЕТ

КАЧО

ЛЕЛЕ

ЛУОК

МАША

НАЧО

НЕРЛ

НСОИ

РКТА

СС-Р

УШЛА

ФМОМ

ФУРК

ХУГО

ЯГИН
}
\end{multicols}

Ответы:

\begin{multicols}{2}
лАТЫНь

кАЧЕЛи

шВАБРа

мноГОУГольник

интеГРИРование

поДМНОжество

параллелЕПИПед

бенЗОЦИстерна

прИМАДонна

логарИФМИрование

нИЩЕТа

сКАЧОк

паралЛЕЛЕпипед

поЛУОКружность

маМАША

зНАЧОк

пиоНЕРЛагерь

тангеНСОИда

аРКТАнгенс

преСС-Релиз

бУШЛАт

ариФМОМетр

биФУРКация

четыреХУГОльник

кнЯГИНя
\end{multicols}
