\section*{Источники и решения}

\subsubsection*{Покер: быстрый вопрос}

Этим вопросом меня озадачил Стэн Уэгон, а он нашёл его в книге Аарона Фридланда \cite{18}.

Суть в том, что все \emph{фулл-хаусы} с тремя тузами одинаково сильны, ведь из одной колоды двух таких не набрать.
Однако фулл-хаусы бьются другими комбинациями: любое \emph{каре}, а их 11 штук, и, что важнее для нас, \emph{стрит-флэшами}.
Фулл-хаусы ТТТ99, ТТТ88, ТТТ77 и ТТТ66 выбивают из колоды максимальное число стрит-флэшей.
Каждый туз выбивает два стрит-флэша, и каждая фоска (карта младшего достоинства) выбивает по пять --- всего 16.
Эти фулл-хаусы и являются лучшими.

Если же вы всё равно потребуете ТТТКК, то вас побьют $40 - 9 \z= 31$ стрит-флэшей, или даже $40 - 16 = 24$, если в вашем фулл-хаусе не нашлось всех четырёх мастей.

\subsubsection*{Восстановление многочлена}

Эту загадку мне подкинул Джо Булер (Рид-колледж), который считает, что она должна быть очень древней.

Как вы наверно уже догадались, достаточно двух запросов:
если $p(1)=n$, то коэффициенты не превышают $n$.
Далее можно взять $x = n + 1$, и записать ответ $(n + 1)$-ичной системе счисления.
Получим все коэффициенты многочлена!

Джо отметил, что если бы разрешалось задавать произвольные числа, то было бы достаточно одного запроса $x = \pi$.
Надо думать, что Оракул найдёт способ выдать значение $p(\pi)$ за конечное время;
если же вместо этого он станет выдавать его десятичное разложение по одной цифре,
то вам придётся понять, когда остановиться.

Хельге Тверберг заметил, что эта задача имеет смысл для многочленов с неотрицательными вещественными коэффициентами.
Чтобы восстановить $p$, сначала спросим $p(1)$; если ответ $0$, то $p = 0$ и задача решена.
В противном случае будем стоить \emph{конечные разности}.
Положим $p_0(x) = p(x)$, и определим рекурсивно последовательность многочленов $p_{i+1}(x) = p_i (x+1) - p_i(x)$.
Отметим, что коэффициенты всех $p_i$-х неотрицательны.
За $k$ вопросов мы можем узнать $p(1),\dots,p(k)$,
а из них вычислить $p_{k-1}(1)$.
Наименьшее $k$ при котором мы получим $0$ это $d+2$, где $d$ --- степень $p$.
Как только мы знаем $d$, любых $d+1$ из $d+2$ имеющихся у нас значений $p$ достаточно, чтобы восстановить многочлен.

\subsubsection*{Спасите наши души}

На эту игру указал мне аспирант Рэйчел Эссельштейн.
Вместе с множеством других игр, она обсуждается в книге Тома Фергюсона по теории игр \cite{ferguson}.
Этот вариант игры также предлагался на 28-ой Американской математической олимпиаде 1999 годa.

Вопрос кажется туманным, пока не увидишь следующее.
Единственный способ заставить противника походить так, чтобы выиграть на следующем ходу это заставить его ходить внутрь конфигурации S-пусто-пусто-S (мы её будем называть \emph{ямой}).
Например, Тристан может выиграть, если $n = 7$, поставив $S$ в середину, а затем поставив ещё одну $S$ в конце, подальше от ответного хода Изольды.
Теперь у нас есть яма.
После пары ходов на другой стороне, Изольде придётся сходить в яму и проиграть.

То же самое справедливо для любого нечётного $n$ больше $7$ --- Тристану достаточно поставить $S$, в $4$-х или более клеток от обоих концов, построить яму с одной или другой стороны, а затем ждать.

Если $n$ чётное, то у Тристана нет шансов, ведь Изольде не могут достаться одни ямы --- каждый раз на её ходе у неё нечётное число пустых клеток.
Если же $n$ чётное и большое, то Изольда выигрывает, сыграв $S$ далеко от концов и от первого хода Тристана.
Однако, если Тристан начинает с $O$, то Изольде нельзя поставить $S$ рядом, поэтому потребуется дополнительное место.

В случае $n = 14$, если Тристан поставит $O$ на $7$-ю клетку (нумерация от $1$ до $14$), то лучший ответ Изольды --- $S$ на в позиции $11$ (угроза создать яму с $S$ на $14$-й клетке).
В ответ, Тристан может поставить $O$ на $13$-ю или $14$-ю клетку (или $S$ на $12$-ю). 
Теперь Изольда хотела бы построить яму поставив $S$ на $8$-ю клетку, но ей нельзя, ведь тогда Тристан выиграет, поставив $S$ в $6$-ю позицию.

Таким образом, при $n = 14$ --- ничья;
Изольде нужно, чтобы $n$ было чётным и не менее 16.
В итоге, Тристан выигрывает, если $n$ --- нечётное и не менее $7$,
Изольда --- если $n$ --- чётное и не меньше $16$;
все остальные значения $n$ приводят к ничьей при оптимальной игре.

\subsubsection*{Пасьянс с шарами}

Эту головоломку (немного в другом виде) можно найти в книге Мартина Гарднера \cite[2.16]{30}, однако там ответ дан без доказательства.
Доказательство из статьи \cite{45}, которое там упомянуто, занимает три страницы и слишком техническое и для Гарднеровской книги, и для моей.

К счастью, есть простой способ понять, что вероятность выигрыша всегда равна $1/2$, независимо от соотношения красных и зелёных шаров в урне.
Ниже приведено рассуждение Серджио Харта из Иерусалимского университета, который и обратил моё внимание на эту задачу.

Иногда в анализе случайного процесса лучше переместить случайность в другое место.
В этой задаче полезно (и допустимо) представить, что перед каждым раундом оставшиеся шары случайно упорядочиваются, и после этого выбираются слева на право.
Тогда в последним раунде, все шары одноцветны.
В предыдущем раунде есть шары обоих цветов, но все красные шары стоят слева, а все зелёные справа, или наоборот.
Независимо от числа шаров каждого цвета на этом этапе (или изначально), эти два порядка равновероятны.
Поскольку первый приводит к выигрышу, а второй к проигрышу, получаем, что вероятность выигрыша равна 1/2.

Серджиу заметил, что этот пасьянс в некотором смысле эквивалентен «Потерянному посадочному талону» \cite[стр. ???]{59}.
Хельге Тверберг (Бергенский университет в Норвегии) отметил что есть также не столь хитрое, но вполне изящное решение индукцией по числу шаров.

\subsubsection*{Пиратская демократия}

Мне напомнил об этой старой загадке аспирант Дартмутского университета Джулио Дженовезе.
Как и многие игры, она решается обратным ходом.
Давайте пронумеруем пиратов от младшего к старшему, и пока будем счиать, что у них всего $n$ монет.
Если дело дойдёт до самого младшего $P_1$, то, он конечно же возьмёт всё золото себе.
(Будем надеяться, что в одиночку он сможет довести корабль к пристани!)

Однако, если дело дошло до $P_1$, то оно дошло и до $P_2$, а
$P_2$, конечно же, оставит себе все монеты и проголосует «за», останется в живых, и возглавит корабль.

Далее, если дело дошло до $P_3$, то он купит голос $P_1$ за одну монету, взяв оставшиеся $n - 1$ себе.
Следовательно, наилучшим вариантом для $P_4$ будет подкупить $P_2$ одной монетой ---
этого достаточно, ведь иначе $P_2$ не получит ничего.

$P_5$ нужны уже два голоса, и он получит их отдав по монете $P_1$ и $P_3$ --- этого достаточно.

Уже видна закономерность, и даже есть что доказывать индукцией:
если осталось нечётное число пиратов, то следует предложить по одной монете каждому оставшемуся нечётному пирату (предполагая, что монет хватит);
если же их чётное число, то следует предложить по одной монете каждому оставшемуся чётному пирату.
По предположению индукции, все получающие монеты пираты проголосуют «за», и теперь уже нечего доказывать.

Ну так что там по поводу капитана?
Чтобы точно выжить, ему потребуется 49 монет; дабы предложить по монете всем чётным пиратам ниже 100.

\subsubsection*{Волшебные рамки}

Эту головоломку предложил Эхуд Фридгут из Еврейского университета; это вариация одной из задач, которая появилась на израильском молодёжном математическом соревновании.
В соревновании рамки были размером $3 \times 3$ и $4 \times 4$.
В этом случае подсчёт вариантов говорит, что всех конфигураций цветов достичь невозможно.
Суть в том, что порядок, в котором располагаются рамки, не имеет значения;
достаточно знать какими способами установки рамок мы воспользовались.
У нас есть $5^2$ способов установки рамки $4 \times 4$
и $6^2$ способов установки рамки $3 \times 3$.
Таким образом, всего $2^{25} \times 2^{36} = 2^{61}$ варианта, этого мало.

Однако, в варианте Фридгута, у нас уже $2^{49} \times 2^{36}$ вариантов, и теоретически этого хватает для получения всех $2^{64}$ конфигураций.
Думаете, теперь получится?

\begin{figure}[ht!]
\centering
\includegraphics[scale=1]{pics/chess}
\caption{Особые клетки (помечены буквой S) и пара рамок.}
\label{pic:chess1}
\end{figure}

Назовём клетку \emph{особой}, если она находятся в третьей или шестой строке или в третьем или шестом столбце, но не в обоих таких позициях сразу (рис. \ref{pic:chess1}); тогда каждая рамка $2 \times 2$ или $3 \times 3$ покрывает чётное число особых клеток.

Поскольку на доске изначально чётное число чёрных особых клеток, у нас не получится достичь конфигурации, с нечётным  числом чёрных особых клеток.

\subsubsection*{Больше рамок на меньшей доске}

Эту головоломку подкинул мне Джулио Дженовезе, он узнал её от Владимира Чернова, тренеровавшего Джулио к Олимпиаде Патнема; сам Владимир, нашёл её в книге «Новые олимпиады по математике» \cite{markova}.

Как и в предыдущей задаче, нужно найти чудесный инвариант.
Однако начнём с совсем неправильной идеи.

Конечно же, достаточно рассматривать только рамки $2 \times 2$, $3 \times 3$ и $5 \times 5$, ведь из них можно составить все остальные.

Как уже было отмечено, стоит проверить, хватит ли всего того, что \emph{разрешается}, для того чтобы сделать всё, что \emph{нужно}.
Можно думать, что номера на доске это числа по модулю $3$ ($0$, $1$ или $2$, и $2 + 1 = 0$).
Следовательно, нам надо беспокоится о $3^{6^2} = 3^{36}$ возможных конфигурациях.
На каждой клетке доски каждый тип квадрата может быть установлен, не установлен или установлен дважды; установка трижды ничего не даёт.
У нас $5^2$ мест для установки квадрата $2 \times 2$,
$4^2$ для $3 \times 3$
и $2^2$ для $5 \times 5$, так что в общей сложности есть $3^{25} \times 3^{16} \times 3^4 = 3^{45}$ возможных действий, и этого более чем достаточно.
Видимо, многие из этих действий имеют один эффект.
Так или иначе, пока неясно, можем ли мы получить желаемый результат.

Математически говоря, у нас есть линейное отображение из векторного пространства $\mathbb{Z}_3^{45}$ в векторное пространство $\mathbb{Z}_3^{36}$, и мы хотим узнать всё ли покрывает образ.
(Возможность перехода от конфигурации со всеми нулями к любой произвольной конфигурации эквивалентна обратному.)

Если ответ да, то мы способны перейти от всех нулей к конфигурации, где все нули, кроме одной единицы в выбранном месте.
Более того, если такое возможно, то задача решена, ведь можно проделать это для каждого местоположения, которому не хватает единицы, и дважды для каждого местоположения, которому не хватает двойки.
Это напоминает поиск решения (с нуля) для кубика Рубика --- нужны операции, которые мало, что меняют.
Например: начнём с двух диагонально смежных квадратов $3 \times 3$ так, что они перекрываются по квадрату $2 \times 2$.
Теперь, если добавить по два подквадрата $2 \times 2$ в каждый угол квадрата $4 \times 4$ и ещё два центральных квадрата, то всё отменится, кроме двух диагонально противоположных углов квадрата $4 \times 4$.
Таким образом, можно увеличить две позиции, одна из которых находится на трёх шагах от другой по диагонали.

Однако сложно представить, что возможно изменить одно значение в произвольной позиции.
Давайте поменяем подход.
Если это \emph{не так}, то есть невозможно получить любую конфигурацию, тогда должен существовать \emph{инвариант}: некоторое число, связанное с конфигурацией, которое ни один шаг не меняет.
В линейной задаче, подобного рода, этот инвариант сам должен быть линейной функцией.
Это означает, что должно быть два подмножества $A$ и $B$ клеток таких, что если сложить числа в $A$ и прибавить удвоенную сумму чисел в $B$ (в нашей арифметике по модулю $3$ это тоже, что сумма чисел в $A$ минус сумма чисел в $B$), то это даст инвариант.

\begin{figure}[t!]
\centering
\includegraphics[scale=1]{pics/chess2}
\caption{Множества $A$ и $B$ помеченные знаками $+$ и $-$.}
\label{pic:chess2}
\end{figure}

Построение выше влечёт, что если некоторая позиция находится в $A$, то позиция (всегда есть ровно одна такая) на расстоянии трёх шагов по диагонали должна находиться в $B$, и наоборот.
Исходя из этого наблюдения и зная, что нам нужно в каждом возможной рамке равное число позиций в $A$ и позиций в $B$ (по модулю 3), мы можем найти чудесный узор на рис. \ref{pic:chess2}, в котором точки в $A$ помечены плюсами, а точки в $B$ --- минусами.
И так, мы утверждаем, что сумма значений в позициях $+$, минус сумма значений в позициях $-$, не может измениться.
Отсюда следует, что мы не можем перейти от любой позиции, где это значение не равно $0$, к позиции, в которой все значения равны~$0$.

\subsubsection*{Простой блеф}

Эта головоломка была предложена Джереми Торпом и Луизой Фуше из Калифорнийского технологического института, но похожие игры были известны и раньше.

Отметим, что имея на руках пику, Луиза всегда выгодно поднимать ставку.
Поэтому у неё есть две \emph{чистые} стратегии:
\begin{itemize}
 \item \textbf{честная} --- поднимать ставку только если есть пика.
 \item \textbf{нахальная} --- всегда поднимать ставку.
\end{itemize}
Если Луиза подняла ставку, то у Джереми есть два варианта ответа:
\begin{itemize}
 \item \textbf{робкий} --- сбросить.
 \item \textbf{смелый} --- проверить.
\end{itemize}

Вероятность вытянуть пику составляет $1/4$.
Значит честная стратегия против робкой даёт Луизе 1 доллар $1/4$ времени, и остальное время 1 доллар выигрывает Джереми.
Ожидаемый выигрыш Джереми составит $1/2$ доллара.
Честная стратегия против смелой даёт Луизе 11 долларов, если у неё пика,
a в среднем приносит ей 2 доллара ($\tfrac14 \times 11 - \tfrac 34 \times 1 = 2$).

Далее, нахальная стратегия против робкой приносит Луизе 1 доллар каждый раз,
в то время как нахальная против смелой обходится ей в $5{,}50$ долларов в среднем ($\tfrac34 \times 11- \tfrac14 \times11 = 5{,}50$).
Если вставить эти числа в матрицу игры $2 \times 2$,
то мы не увидим доминирующей стратегии ни для одного из игроков.
Значит, как и следовало ожидать, придётся использовать вероятностную стратегию.

Из работ Джона фон Неймана (ещё до Джона Нэша) известно, что существует \emph{равновесие Нэша} для этой игры --- пара стратегий, при которых ни один из игроков не может улучшить свою стратегию, при условии, что другой игрок не меняет свою.
Посмотрим, что это означает для Луизы: если ей не выгодно переходить к честной или нахальной стратегии, то в среднем ей все равно, проверит ли её Джереми или сбросит.

Предположим, что Луиза решила блефовать с вероятностью $p$, если у неё нет пики.
Против робкой стратегии её ожидаемый выигрыш в среднем составит $\tfrac14 \times 1 + p \times \tfrac34 \times 1 - (1 - p) \times \tfrac34 \times 1 =(\tfrac32p - \tfrac12)$ долларов.
Ну а против смелой, $\tfrac14 \times 11 - p \times \tfrac34 \times 11 - (1 - p) \times \tfrac34 \times 1 \z= (2 - \tfrac{15}2p)$ долларов.

Поскольку Луизе должно быть всё равно, эти две величины должны совпасть, и это даёт $p = 5/18$.
То есть, Луизе следует блефовать $5$ из $18$ раз, когда у неё нет пики и, конечно, всегда повышать ставку, если пика есть.
Её ожидаемый выигрыш независимо от стратегии Джереми будет
\[\frac32\times\frac5{18}-\frac12=2-\frac{15}2\times\frac5{18}=-\frac1{12}\]
долларов.
То есть в среднем Луиза теряет по $\tfrac1{12}$ доллара за игру.

Немного поразмыслив, можно убедиться, что Луизе выгодно увеличить размер повышения ставки,
однако в среднем она останется в минусе, если конечно играют честно.
Причина в том, что если у неё нет пики, то она не может позволить себе блефовать чаще чем один раз из трёх.
Иначе, с точки зрения Джереми, вероятность того, что у неё была бы пика, составила бы не больше половины, и поэтому Джереми мог бы просто всегда проверять.
Луиза в лучшем случае выйдет в ноль, при повышениях ставки, и будет терять по доллару без повышений, и значит в среднем будет проигрывать.
Но если $p < 1/3$, то Луиза проигрывает робкой стратегии Джереми;
в этом случае она чаще теряет свою ставку, чем выигрывает.

Здесь важно, что вероятность вытянуть пику составляет одну четверть.
Если бы шансы были чуть выше (скажем, если в колоде нет дамы червей), то большая ставка повернула бы удачу в сторону Луизы.

Вернёмся к исходным 10 долларам.
Можно вычислить стратегию равновесия для Джереми (хотя нам этого и не требуется).
Предположим, что если Луиза повышает ставку, то Джереми проверяет с вероятностью $q$.
Тогда, против луизиной честной стратегии, он получает $\tfrac34 \times 1 - \tfrac14 \times q \times 11 - \tfrac14 \times (1 - q) \times 1 = \tfrac12 - \tfrac52q$ доларов,
а против смелой стратегии $\tfrac34 \times q \times 11 - \tfrac14 \times q \times 11 - (1 - q) \times 1 = \tfrac{13}2q - 1$ долларов.
Полагая, что эти величины равны, получаем $q = \tfrac3{18}$; то есть, Джереми должен проверять только $\tfrac3{18}$ всего времени.
Подстановка $q = \tfrac3{18}$ обратно в выражения даёт Джереми $\tfrac1{12}$ доллара за игру в среднем, как и должно быть, ведь ровно столько теряет Луиза.

\subsubsection*{Китайский Ним}

Эта игра известна как Китайский ним или как Игра Витоффа;
она появилась в его статье 1907 года \cite{60}.
Игра обсуждается несколько раз в первом и втором томе классической книги Элвина Берлекампа, Джона Конвея и Ричарда  Гая \cite{4}.
Связь с «Надёжными мигалками» приведённой выше была замечена в прекрасной книге Сергея Табачникова \cite{56}.
Однако ни одна из этих книг не приводит вывод стратегии.

Каждая позиция $\{x, y\}$ в игре Алекса и Бет является либо выигрышной, либо проигрышной для игрока, который делает следующий ход, при условии оптимальной игры обоих.
Как и в классическом ниме, проще всего попытаться характеризовать проигрышные позиции, поскольку их меньше.

Как только известны проигрышные позиции, можно вывести правильную стратегию.
Если, например, Алекс находится в выигрышной позиции, то у него должна быть возможность одним ходом перейти к проигрышной позиции для Бет.
Если же Алекс находится в проигрышной позиции, то он может только надеяться на ошибку Бет, или же он по-джентльменски предложит ей сделать первый ход.
Таким образом, стратегия сводится к списку проигрышных позиций.
Но разве здесь нет порочного круга?
Разве нам не нужно знать правильную стратегию, чтобы найти проигрышные позиции?
К счастью, поскольку количество бобов всегда уменьшается, мы можем начать снизу и постепенно подниматься вверх.

Любая позиция с одной пустой кучей или с кучами одинакового размера автоматически является выигрышной.
Не сложно понять, что самая простая проигрышная позиция это $\{1, 2\}$.
После этого можно увидеть, что $\{3, 5\}$, $\{4, 7\}$ и $\{6, 10\}$ также проигрышные.
Но где же закономерность?

Пусть $\{x_1 , y_1\}$, $\{x_2 , y_2\},\dots$ будут проигрышными позициями для первого игрока (не считая $\{0, 0\}$);
мы предполагаем что $x_i < y_i$ и $x_i < x_j$ при $i < j$.
Заметим, что $x_i \ne x_j$ для $i \ne j$, ведь если $x_i = x_j$ то Алекс мог бы сделать ход, от большего из $y_i$ и $y_j$ к меньшему, оставляя Бет в проигрышной позиции --- противоречие.

Немного поразмыслив, приходим к выводу, что если известны все проигрышные позиции от $\{x_1 , y_1\}$ до $\{x_{n-1}, y_{n-1}\}$, то $x_n$ есть наименьшее положительное число, которого нет среди чисел из $\{x_1, \dots , x_{n-1}\} \z\cup \{y_1, \dots , y_{n-1}\}$, а $y_n = x_n + n$.
Заметим, что в этом случае $y_n$ превосходит любое число из $\{x_1, \dots , x_{n-1}\} \cup \{y_1, \dots , y_{n-1}\}$.

Доказательство ведётся индукцией по $n$.
Мы уже знаем, что $x_n$ не может быть среди чисел в $\{x_1, \dots , x_{n-1}\} \cup \{y_1, \dots , y_{n-1}\}$, а также, что не может быть более одного $y_n$, который соответствует этому $x_n$.
Остаётся показать, что позиция $\{x_n, y_n\}$ проигрышная.

Если $\{x_n, y_n\}$ была бы выигрышной, то из неё можно было бы прийти в $\{x_i, y_i\}$ для некоторого $i < n$; но такой позиции нельзя достичь уменьшив меньшую кучу или уменьшив обе кучи на одинаковое число бобов, ведь это сделало бы разницу между двумя кучами $n$ или больше.
Также нельзя её достичь, уменьшив б\'{о}льшую кучу, ведь тогда был бы ещё один игрек для одного икса.
Таким образом, $\{x_n, y_n\}$ проигрышная.

Теперь есть возможность создать список проигрышных позиций любой длины.
Из этого легко вывести стратегию Алекса.
Если он столкнётся с $\{x_i , y_i\}$, он убирает один или два боба и надеется на ошибку.
Если он видит $\{x_i , z\}$ для $z > y_i$, он уменьшает $z$ до $y_i$.
Если он видит $\{x_i , z\}$ с $x_i < z < y_i$, то есть разница $d = z - x_i < i$, он берет из обеих куч, чтобы дойти до $\{x_d , y_d\}$ (если $z = y_j$ для некоторого $j < i$, то у него также есть вариант уменьшить $x_i$ до $x_j$).
Если он видит $\{y_i , z\}$ с $y_i \le z$, то может уменьшить $z$ до $x_i$, а может иметь и другие варианты.

Однако потребуется значительное время, чтобы вычислить все проигрышные позиции, скажем до тысячи бобов в каждой куче.
Может можно найти более явное описание проигрышных позиций?

Как мы уже знаем, $x_n$ лежит между $n$ и $2n$ для каждого $n$, ведь $x_n$ стоит сразу после всех $x_i$ и некоторых $y_i$ при $i < n$.
Разумно предположить, что $x_n$ примерно равно $rn$, для некоторого $r$ между $1$ и $2$.
Если это так, то $y_n$ должно быть примерно равным $rn + n = (r + 1)n$.

Если это подтвердится, то $n$ чисел иксов между $1$ и $x_n$ примерно равномерно распределены, и, следовательно, доля $r/(r + 1)$ от их количества будет соответствующих игрекам ниже $x_n$.
Таким образом, у нас около $nr/(r + 1)$ игреков ниже $x_n$, и вместе с $n$ иксами, всего получается $x_n$ чисел; то есть
\[n+n\frac{r}{r+1}=nr,\]
что даёт нам $r + 1 = r^2$ или $r = (1 + \sqrt{5})/2$ --- знакомое \emph{золотое сечение}.

Наверное теперь к вам на ум пришло блестящее наблюдение --- поскольку $r$ иррационально и $\tfrac1r+\tfrac1{r^2}=1$, числа $r$ и $r^2$($= r + 1$) подходят на роль $p$ и $q$ в решении «Надёжных мигалок» из главы 3.
Как мы знаем, любое положительное целое число представлено единственным способом как $\lfloor pm\rfloor$ для некоторого целого $m$, \emph{либо} как $\lfloor qn\rfloor$ для некоторого целого $n$.

А теперь уже возникает подозрение, что $x_n=\lfloor rn\rfloor$, а $y_n=\lfloor r^2 n\rfloor$.
Конечно же, эти значения обладают желаемым свойствам:
каждое $x_n$ --- наименьшее положительное число, не из $x_1, \dots , x_{n-1}$ или $y_1, \dots, y_{n-1}$, иначе его было бы невозможно получить.
Остаётся, проверить, что $\lfloor r^2 n\rfloor - \lfloor rn\rfloor = n$, но это легко, ведь $r^2 n - rn$ равно целому числу $n$,
поэтому и разница их целых частей обязана быть $n$ --- готово!

Чтоб развлечься давайте найдём ход Алекса из предложенных примеров позиций.
Обратите внимание, что $12 000/r$ чуть меньше $7417$, и $7417r = 12 000.9581\dots$ так что $12 000$ это один из иксов, точнее $x_{7417}$.
Соответствующее значение $y_{7417}$ равно $\lfloor 7417r^2\rfloor = 19 417$, поэтому если в другой куче $20 000$ бобов, Алекс может выиграть, забрав из неё $20 000 - 19 417 = 583$ боба.
Если же в другой куче всего $19 000$ бобов, то Алекс может выиграть, уменьшив кучи одновременно до $\{x_{7000}, y_{7000}\} = \{11 326, 18 326\}$.

\begin{addedbytheeditors}
\textbf{Редакторам:}
Думаю стоит добавить диаграмму с отмеченными проигрышными позициями скажем до 30--40 --- могу заняться.
\end{addedbytheeditors}
