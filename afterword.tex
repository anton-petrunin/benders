\chapter{Послесловие}


\setlength{\epigraphwidth}{.83\textwidth}
\epigraph{Я надеюсь, что потомки благосклонно отнесутся ко мне не только за то, что я объяснил, но и за то, что я намеренно пропустил, дабы удовольствие открытия досталось другим.}{--- Рене Декарт (1596---1650), Геометрия}


Вы можете верить Декарту, а можете не верить, но не стоит верить мне, если я скажу, что намеренно пропустил кое-что дабы вы нашли это сами.
Тем не менее в мой сборник не попало \emph{огромное} множество отличных увлекательных головоломок, а ещё больше таких головоломок предстоит придумать.
Среди приведённых ссылок, многие доступны онлайн, там эти головоломки стоит поискать,
и конечно вы можете придумать свои собственные.

Головоломки не замена математическому образованию, но они помогают запомнить идеи, которые вы выучили, а ещё развлекают и развивают ум.
У этого сборника есть ещё дополнительная цель:
не допустить у читателя самолюбования своей математической интуицией.

Во всяком случае, мне это помогает.

\begin{flushright}
Питер Уилкнер\\
28 февраля 2007
\end{flushright}
