\chapter{Послесловие}


\setlength{\epigraphwidth}{.83\textwidth}
\epigraph{Я надеюсь, что потомки благосклонно отнесутся ко мне не только за то, что я объяснил, но и за то, что я намеренно пропустил, дабы удовольствие открытия досталось другим.}{--- Рене Декарт (1596---1650), Геометрия}


Верите вы Декарту или нет, не стоит верить мне, если я скажу, что намеренно пропустил кое-что дабы вы нашли это сами.
Однако в мой задачник не попало \emph{ужасно много} отличных увлекательных головоломок, а ещё больше их предстоит придумать.
Среди приведённых ссылок многие доступны онлайн; там эти головоломки стоит поискать,
и конечно вы можете придумать свои собственные.

Головоломки не замена математическому образованию, но они помогают запомнить идеи, которые вы выучили, а ещё развлекают и развивают ум.
Этот задачник преследует ещё и дополнительную цель:
помешать математической интуиции заниматься самолюбованием.

Во всяком случае, мне это помогает.

\begin{flushright}
Питер Уилкнер\\
28 февраля 2007
\end{flushright}
