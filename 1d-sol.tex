\section*{Источники и решения}

\subsubsection*{Укрепление сетки}

Эту интересную (и, возможно, практически полезную) головоломку подкинул мне геометрический гуру Боб Коннелли из Корнеллского университета; она основана на работе Этана Болкера и Генри Крапо \cite{8}.


\begin{figure}[ht!]
\centering
\includegraphics[scale=1]{pics/lattice2}
\caption{Недоукреплённая сетка и её граф.}
\label{pic:lattice2}
\end{figure}

\begin{figure}[t!]
\centering
\includegraphics[scale=1]{pics/lattice3}
\caption{Полностью укреплённая сетка и её граф.}
\label{pic:lattice3}
\end{figure}

Задачу полезно перевести на язык графов, но не самым очевидным образом (не нужно смотреть на граф с вершинами в сочленениях стержней).
Предположим, что скобы расставлены.
Рассмотрим граф $G$, вершины которого соответствуют строкам и столбцам сетки.
Каждое ребро в $G$ соответствует строке и столбцу, пересекающимся по закреплённой клетке, так что число рёбер в $G$ равно числу скоб.
Сетка с рис. \ref{pic:lattice1}, показана снова на рис. \ref{pic:lattice2}, но уже с её графом.

Предположим, что некоторая строка смежна в $G$ с некоторым столбцом.
Тогда вертикальные стержни этой строки перпендикулярны горизонтальным стержням этого столбца.
Если $G$ --- связный граф (то есть любые две вершины можно соединить путём), то все горизонтальные стержни должны быть перпендикулярны всем вертикальным.
Таким образом, все горизонтальные стержни параллельны друг другу,
также параллельны и все вертикальные.
Теперь ясно, что сетка жёсткая.

С другой стороны, предположим, что граф несвязен.
Пусть $C$ --- его \emph{компонента}, то есть связный кусок $G$, рёбра из которого не идут к остальным вершинам $G$.
Тогда ничто не мешает любому вертикальному стержню в строке $C$ или любому горизонтальному стержню в столбце $C$ крутиться относительно остальных стержней в сетке.

Таким образом, жёсткость в точности означает связность $G$.
Поскольку $G$ имеет $2n$ вершин, если он связен, то у него как минимум $2n - 1$ ребро
(если это для вас новость, то попробуйте доказать индукцией по числу вершин).
Следовательно, чтобы сделать сетку жёсткой, нужно как минимум $2n - 1$ скоб.

Обратите внимание, что скобы нельзя ставить где попало.
На рисунке 21 показана полностью закреплённая сетка $3 \times 3$, и её граф.
Попробуйте подсчитать сколькими способами можно укрепить сетку $3 \times 3$ используя минимальное число скоб  (пять штук).
Есть теорема в теории графов о том, что у каждого связного графа есть \emph{остовое дерево}; то есть связный подграф с минимальным числом рёбер.
Она позволяет сделать следующий вывод: \emph{если закреплено больше чем $2n - 1$ клеток и сетка жёсткая, то можно удалить все кроме $2n - 1$ скоб, сохраняя жёсткость.}

\subsubsection*{Путешествие по острову}

Вариант этой головоломки попал на веб-страницу «The Puzzle Toad» \cite{bohman-pikhurko-frieze-sleator} из упомянутой выше книги «Московские математические олимпиады» Г. А. Гальперина и А. К. Толпыго \cite{23}.

Между перекрёстками текущее состояние Алоисия можно охарактеризовать тройкой, состоящей из ребра, на котором он находится, направления движения и типа последнего поворота (вправо или влево).
Эта тройка полностью определяет будущие и прошлые положения Алоисия.
Поскольку таких троек конечное число, настанет момент, когда Алоисий впервые попадёт в одну и ту же тройку, и это может произойти только на его стартовом ребре!

\begin{addedbytheeditors}
\textbf{Редакторам:} Добавил фразу про будущее и прошлое
\end{addedbytheeditors}


\subsubsection*{Провода под Гудзоном}

Вариант этой головоломки называют задачей Грэма --- Нолтона;
её пропагандировал Мартин Гарднер.
Для электриков это просто задача идентификации кабельных линий.
В версии Гарднера можно было замыкать любое число проводов на любом берегу и также проверять их на любом берегу.
Следующее решение было предложено Роландом Спрэгой в его книге \cite{54}, а также в недавней статье трёх молодых специалистов по информатике Навина Гойала, Сачина Лодхи и Муту Мутукришнана \cite{33}.
Оно удовлетворяет нашим дополнительным ограничениям и требует только две операции на каждом конце (таким образом, потребуется три переправы через реку, не считая дополнительной для размыкания проводов перед их использованием).
Однако решение не единственное, и если ваше трёхпереправное решение отличается, то оно может быть не хуже.

Пометим концы проводов $w_1$, $w_2, \dots, w_{50}$ на западном берегу
и $e_1, \dots, e_{50}$ на восточном. %??? n>50
При первом посещении западного берега соединим $w_1$ с $w_2$, $w_3$ с $w_4$, $w_5$ с $w_6$ и так далее, но последнюю пару $w_{49}$ и $w_{50}$ соединять не будем.
Затем будем изучать провода на восточном берегу, пока не найдём все пары.
Например, мы можем обнаружить, что $e_4$ соединён с $e_{29}$, $e_2$ с $e_{15}$, $e_8$ с $e_{31}$ и так далее, а концы $e_{12}$ и $e_{40}$ остались без пары.
Затем мы едем на западный берег, рассоединяем все пары и соединяем $w_2$ с $w_3$, $w_4$ с $w_5$ и так далее, оставив $w_1$ и $w_{50}$ без соединения.
Опять изучаем восточные концы, пока не найдём все пары.
Продолжая пример, пусть $e_{12}$ соединён с $e_{15}$, $e_{29}$ с $e_2$, и $e_4$ с $e_{31}$, а концы $e_{40}$ и $e_8$ остались без парных.

Удивительно, но этой процедуры достаточно для идентификации всех проводов!

Тот восточный конец провода, который был спарен в первый раз, но не во второй (в нашем примере это $e_8$), должен соответствовать $w_1$.
Следовательно восточный конец провода, с которым $e_8$ был спарен в первый раз (у нас это $e_{31}$), должен соответствовать $w_2$.
Но тогда $w_3$ должен соответствовать восточному концу провода, с которым $e_{31}$ был спарен во второй раз, а именно $e_4$.
Продолжая таким образом, мы находим, что $w_4$ соответствует $e_{29}$ (парному к $e_4$ на первом круге), $w_5$ соответствует $e_2$ (парному к $e_{29}$ на втором круге) и так далее.
В конце концов мы видим, что $w_{50}$ соответствует $e_{40}$.

Если число проводов (скажем, $n$) нечётно, то в первый раз можно оставить без пары только $w_n$, а во второй --- $w_1$, и всё сработает примерно так же.

\subsubsection*{Жуки на четырёх прямых}

Эта головоломка мне досталась от Мэтта Бэйкера из Технологического института Джорджии.
Иногда её называют \emph{задачей четырёх путешественников};
её можно увидеть на веб-сайте «Cut the knot» \cite{cut-the-knot}.

В наиболее изысканном решении, которое мне известно, требуется подняться из плоскости в пространство, добавив ось времени.
Предположим, что все встречаются, кроме (возможно) третьего и четвёртого жука.
Проведём ось времени перпендикулярно плоскости с жуками, и пусть $g_i$ --- график $i$-го жука в пространстве.
Поскольку каждый жук ползёт с постоянной скоростью, каждый такой график является прямой;
его проекция на плоскость с жуками --- это та самая прямая, по которой ползёт жук.
Два жука встречаются тогда (и только тогда) когда их графики пересекаются.

Прямые $g_1$, $g_2$ и $g_3$ находятся в одной плоскости, так как они попарно пересекаются.
То же самое относится и к тройке  $g_1$, $g_2$ и $g_4$.
Следовательно, все четыре графика лежат в одной плоскости.
Конечно же, $g_3$ и $g_4$ не параллельны, ведь не параллельны их проекции.
Таким образом, эти две прямые обязаны пересечься в своей плоскости,
а это и значит, что третий жук встретит четвёртого.

\subsubsection*{Пауки на кубе}

У этой головоломки тот же источник, что и у «Путешествия по острову» выше.

\begin{figure}[ht!]
\centering
\includegraphics[scale=1]{pics/cube}
\caption{Два чёрных ребра под контролем, и муравей ловится в серой зоне.}
\label{pic:cube}
\end{figure}

Для поимки муравья можно заставить двух пауков охранять по одному ребру.
Для охраны ребра $PQ$ паук сначала выгоняет с него муравья, если это необходимо, а затем бегает по ребру так, что он всегда хотя бы в три раза ближе к $P$ (и к $Q$) чем муравей.
Это возможно, ведь если запрещено использовать ребро $PQ$, то путь от $P$ до $Q$ вдоль рёбер куба в три раза длиннее самого ребра.

Если охранять два \emph{противоположных} ребра (другие варианты также работают), то в оставшейся части куба без этих рёбер и их концов не будет циклов (см. рис. \ref{pic:cube}).
Значит, третий паук сможет преследовать муравья до конца охраняемого ребра, где тот встретит свою грустную участь.

\subsubsection*{Вменяемые мыслители}

Эту головоломку мне предложил Саша Разборов из Института перспективных исследований;
с его слов я знаю, что она была кандидатом на Международную математическую олимпиаду, однако была отвергнута как слишком сложная.
Она была рассмотрена и решена в статье Э. Голеса и Х. Оливоса \cite{31}.

Нам надо доказать, что мнения установятся или будут меняться с периодом в пару недель.
Будем думать о каждом знакомстве как о паре стрелок, по одной в каждом направлении.
Назовём стрелку \emph{обидной}, если мнение перевёртовца в начале стрелки отличается от мнения его знакомого на конце стрелки на \emph{следующей неделе}.

Рассмотрим стрелки, выходящие от некого превёртовца Клайда на неделе $t - 1$, во время которой Клайд выступает (скажем) за торговый центр.
Предположим, что из них $m$ обидных.
Если Клайд все ещё (или снова) за торговый центр на неделе $t + 1$, то число, скажем $n$, обидных стрелок, указывающих на Клайда на неделе $t$, будет в точности равно~$m$.

Однако, если Клайд против торгового центра на неделе $t + 1$, то $n$ будет строго меньше $m$, так как большинство его друзей были против торгового центра на неделе $t$.
Следовательно, большинство стрелок от Клайда были обидными на неделе $t - 1$, а на неделе $t$ только меньшинство обидных стрелок направлены к Клайду.

Всё это остаётся верным и если Клайд был против торгового центра на неделе $t - 1$.

Но вот какое дело: \emph{каждая} стрелка начинается у \emph{кого-то} на неделе $t - 1$ и заканчивается у кого-то на неделе $t$.
Таким образом, общее число обидных стрелок между неделями $t - 1$ и $t$ не увеличивается и даже строго уменьшается, за исключением одного случая --- когда каждый перевёртовец имел такое же мнение на неделе $t - 1$, как и на неделе $t + 1$.

Общее число обидных стрелок в данную неделю не может бесконечно уменьшаться и в конечном итоге должно достичь некоторого числа, с которого уже никогда не опустится.
В этот момент каждый перевёртовец либо сохранит своё мнение навсегда, либо будет менять его туда-сюда каждую неделю.

\medskip

Задачу можно значительно обобщить, например, добавив веса вершинам (это означает, что мнения одних более ценны, чем мнения других), разрешив петли (то есть разрешив учитывать своё текущее мнение), введя механизмы разрешения конфликтов и даже установив различные пороги для смены мнений «за» и «против».

\subsubsection*{Лемминг на шахматной доске}

Эту замечательную головоломку придумал Кевин Пурбху, ещё будучи старшеклассником в Торонто.
С тех пор он защитил диссертацию по математике в Университете Калифорнии в Бёркли и
сейчас проходит постдоктурантуру в Университете Британской Колумбии,
занимаясь чем-то под названием \emph{тропическая геометрия}.
Сам я узнал головоломку от Рави Вакила из Стэнфордского университета.

Лемминг действительно обречён.
Один из способов это понять (найденный независимо Вакилом и мной) --- представить, что лемминг может перемещаться на любую соседнюю клетку, но должен при этом повернуться в направлении стрелки, которую он там обнаружит.
Лемминг не может повернуться на 360°, обойдя цикл;
ведь если бы он мог, то можно уменьшать такой цикл, пока не придём к противоречию.
Но настоящий лемминг, если он хочет остаться на доске, в конечном итоге должен обойти цикл, и когда это произойдёт, ему придётся повернуть на 360°.

Собственное решение Пурбху, с его школьных лет, использует индукцию.
Если лемминг остаётся на доске, он, как мы уже отметили, должен будет обойти цикл.
Пусть $C$ --- цикл с наименьшей возможной площади (на любой доске), на котором это может произойти; будем считать, что леминг обходит его по часовой стрелке.
Обрежем всю доску до $C$ и того, что он окружает.
Затем повернём все стрелки на 45° по часовой стрелке.
Это приведёт к меньшему циклу!

\begin{addedbytheeditors}
Пользуясь нехитрой техникой, можно свести головоломку к следующему утверждению про векторные поля: \emph{Векторное поле без нулей на плоскости не имеет замкнутых интегральных линий.}
Решение получится сложней, но возможно полезней.
\end{addedbytheeditors}
