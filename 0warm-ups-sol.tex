\section*{Источники и решения}

\subsubsection*{Половина роста}

Родители маленьких детей знают ответ: два года!
(То есть между вторым и третьим днём рождения.)
Да, человечек растёт очень нелинейно.

Задача предложена Джеффом Стейфом из Университета Чалмерса в Швеции.

\subsubsection*{Шарики в мешочках}

Потребуется четырнадцать шариков.
Положите пустой мешочек в мешочек с одним шариком, 
далее второй мешочек в третий, с ещё одним шариком, затем третий в четвёртый, с ещё одним шариком, и так далее.
Таким образом в $i$-м мешочке будет $i-1$ шарик (и столько же мешочков).

Если вы не догадались засовывать мешочки в мешочки, или решили, что так нечестно, то вам понадобится $0 + 1 \z+ \z\dots \z+ 14 \z= 15 \times 7 \z= 105$ шариков.

Задача предложена Диком Плотцем из Провиденса, штат Род-Айленд.

\subsubsection*{Степени двойки}

Ответ восемь.
Четвёрка слов «дважды», «две», «пары» и «двойняшка» может натолкнуть на мысль, что должно получиться $2^4 \z= 16$.
Но двойняшка — это всего один человек.

Классическая загадка.

\subsubsection*{Катящийся карандаш}

Мой коллега Лори Снелл подловил меня на этой задачке.
А вы попались?
Похоже, что ответ должен быть $\tfrac15$, но поскольку $5$ нечётно, карандаш будет лежать гранью вниз и ребром вверх.
Таким образом, ответ $0$ или, если хотите, $\tfrac25$, в зависимости от вашего толкования термина \emph{вверх}, но уж всяко не $\tfrac15$.

Эта головоломка приведена в провокационной книге Чамонта Ванга \cite{58}.

\subsubsection*{Портрет}

Это древняя загадка;
она приводится в классической книге Рэймонда Смаллиана \cite{55}.

«Сын моего отца» может означать лишь самого хозяина, поскольку у него нет ни братьев, ни сестёр.
Значит, на портрете — сын хозяина.

\subsubsection*{Странная последовательность}

Эту загадку переслал мне Кит Кохон, юрист из Агентства по охране окружающей среды.
Это конец алфавита в обратном порядке, то есть ZYXW, но буква Z повёрнута на 90° (вправо или влево), и каждая последующая буква повёрнута на дополнительные 90°.
Следующей должна стоять повёрнутая буква V, то есть < или >.

\subsubsection*{Параметр языка}

Ответ: 1 (один).
Эта загадка отчасти математическая; её придумала Тина Кэрролл, аспирантка Технологического института Джорджии. 
Каждое число — это первое натуральное число, название которого в данном языке (количественное числительное) состоит из нескольких слогов.

\begin{addedbytheeditors}
В оригинале вопрос был про английский язык.
Сверившись с веб-страницей \cite{numerals}, можно найти этот параметр для других языков;
например он равен 2 для татарского, 3 для баскского, 4 для эстонского, 5 для айнского, 6 для (северно)китайского. \pr
\end{addedbytheeditors}

%7  --- юкатекский
%8  --- ???
%9  --- ???
%10 --- чеченский и дзонг-кэ
%%% неправда, в чеченском 11:  https://ru.wiktionary.org/wiki/%D1%86%D1%85%D1%8C%D0%B0%D0%B9%D1%82%D1%82%D0%B0
%11 --- ???
%12 --- ???
%13 --- ???
%14 --- французкий
%15 --- ???

\subsubsection*{Вниманию параскаведекатриафобов}

Удивительно, но правда.
Насколько мне известно, это обнаружил Банкрофт Браун (как и автор этих строк, профессор математики Дартмутского колледжа), который привёл свои расчёты в журнале American Mathematical Monthly \cite{11}.
На это мне указал мой нынешний коллега Дана Уильямс.

Нетрудно проверить, что из 4800 месяцев в 400-летнем цикле григорианского календаря 13-е число выпадает на пятницу 688 раз.
Воскресенье и среда приходятся по 687 раз, понедельник и вторник по 685, а четверг и суббота только по 684.
При подсчёте нужно помнить, что годы, кратные 100, не являются високосными, если только (как 2000 год) они не делятся на 400.

Происхождение суеверия относительно пятницы 13-го обычно связывают с датой приказа французского короля Филиппа IV (Филипп Красивый) о разгроме ордена тамплиеров.

Потренировавшись, можно выучиться определять день недели любой даты в истории, даже учитывая прошлые календарные сложности
(по крайней мере, на это способен такой человек, как глубокоуважаемый Джон Конвей из Принстонского университета).
А для тех ленивых смертных, что живут сегодняшним днём, полезно помнить, что в любом году
04.04, 06.06, 08.08, 10.10, 12.12, 09.05, 05.09, 07.11, 11.07 и последний день февраля выпадают на один и тот же день недели.
(Это ещё легче запомнить, если вы играете в крэпс ежедневно с 9 до 5.)
В 2007 году, этот день --- среда;
перед невисокосным годом он сдвигается на один, и на два перед високосным.

\begin{addedbytheeditors}
Считается, что слово параскаведекатриафобия было придумано доктором Дональдом Досси, который говорил своим пациентам буквально следующее: «Когда вы научитесь его произносить, вы вылечитесь!».\pr
\end{addedbytheeditors}%??? нужно ли это???


\subsubsection*{Честная игра}

Подбросьте гнутую монету \emph{дважды} в надежде получить орёл и решку.
В случае, если сначала выпал орёл, считаем, что выпал «ОРЁЛ»;
если сначала выпала решка, считаем, что «РЕШКА».
Если выпадут две решки или два орла, то опыт придётся повторить.

Мне напомнил об этой головоломке Тамаш Ленгель из Маккалестерского колледжа;
её решение приписывается великолепному математику и пионеру информатики  Джону фон Нейману и иногда называется «трюком фон Неймана».
Оно основано на том, что даже если монета гнутая, последующие броски являются (по крайней мере, должны быть) независимыми событиями.
Конечно же придётся предположить, что гнутая монета может приземлиться на любую сторону!

Вышеупомянутую схему можно улучшить, уменьшив среднее число бросков.
Например, получив орёл-орёл при первой паре бросков и решку-решку при второй, можно считать результат «ОРЛОМ» (тогда конечно же решку-решку, за которой следует орёл-орёл, надо считать «РЕШКОЙ»).
Возможны и другие улучшения.
Статья Шербана Наку и Юваля Переса \cite{44} выдавливает последнюю каплю из минимизации ожидаемого числа бросков, независимо от вероятностей получения орла и решки.

В последние годы вопрос извлечения «честных» %безпристрастных???
случайных битов из ненадёжных случайных источников становится важным в теории вычислений, --- ему посвящены многие исследования, в которых достигнуты существенные прорывы.

\subsubsection*{Кривые на картофелинах}

{\sloppy

Рассмотрите пересечение картофелин!
Другими словами, представьте, что каждая картофелина это призрак, и воткните одну в другую.
Пересечение их поверхностей будет кривой на каждой из них; эти кривые и следует нарисовать.

}

Эту милую головоломку можно найти (среди прочего) в книге \cite{5}\footnote{Источник указан так: From Dieter Gebhardt (private communication).}.

\begin{addedbytheeditors}
Несмотря на столь простое решение, точная математическая формулировка задачи остаётся неясной.

Пересечение поверхностей картофелин может быть фракталом, не содержащим замкнутых кривых, даже если сами поверхности гладкие.
В случае, если поверхности гладкие, картофелины легко расположить так, чтобы пересечение было гладкой замкнутой кривой.
(Для этого можно воспользоваться леммой Сарда.)
То же можно сделать и при более слабых предположениях.

Однако без дополнительных предположений вопрос остаётся открытым \cite{agol};
то есть неизвестно, \emph{содержат ли две произвольные вложенные сферы в евклидовом пространстве пару конгруэнтных замкнутых кривых}. 
Похоже, что вопрос открыт даже если обе сферы имеют конечную площадь.
Это предположение кажется разумным, ведь как отметил Пер Александерсон,
«Я стараюсь не покупать картофель с бесконечной площадью поверхности --- его слишком долго чистить.»

Следующая вариация задачи была предложена на 35 Турнире городов (2013/14), автор --- Е.~Бакаев:  \textit{Космический аппарат сел на неподвижный астероид, про который известно только, что он представляет собой шар или куб. Аппарат проехал по поверхности астероида в точку, симметричную начальной относительно центра астероида. Всё это время он непрерывно передавал свои пространственные координаты на космическую станцию, и там точно определили трёхмерную траекторию аппарата. Может ли этого оказаться недостаточно, чтобы отличить, по кубу или по шару ездил аппарат?}
\pr
\end{addedbytheeditors}

\subsubsection*{Победа на Уимблдоне}

Кажется очевидным, что лучше всего выиграть два сета (для победы в мужском финале требуется выиграть три сета из пяти), а в третьем сете --- вести 5:0 по
геймам и вести со счётом 40-0 в шестом гейме.
(Возможно, вы предпочтёте подавать в шестом гейме, но если ваша подача так же плоха, как моя, то лучше, чтоб подавал соперник --- тогда можно молиться о его двойной ошибке, которая принесёт вам победу).

Но не так быстро!
Такой счёт даст вам три шанса, но можно добыть шесть --- три на вашей подаче и ещё три на подаче Роджера.
Как и раньше, вы выиграли два первых сета, но в третьем получили 6:6 по геймам и ведёте 6:0 на тай-брейке.

Амит Чакрабарти из Дартмута предложил ещё одно улучшение, основанное на том, что по традиции полный счёт теннисного матча включает в себя счёт всех сетов, а при результате гейма 6:6 в него включается также и счёт тай-брейка. Тогда можно запросить, например, чтобы счёт был 6-0, 6-6 (9999-9997), 6-6 (6-0).
Идея (этически спорная, конечно) состоит в том, что пока работала магия, ваш соперник настолько устал в тай-брейке второго сета, что теперь с большей вероятностью оплошает в одном из шести предстоящих матч-пойнтов.

\begin{addedbytheeditors}
Примечания для не знающих правил тенниса.
Матч-пойнт --- ситуация, когда выигрыш всего одного очка приводит к завершению матча.
Тай-брейк --- розыгрыш партии, который
случается при счёте 6:6 в решающем сете, --- он играется до 7 очков или до преимущества
одного из игроков в 2 очка (то есть 7:6 ещё не победа на тай-брейке, для победы нужно 8:6, 9:7 и так далее).
Смена подающего на тай-брейке происходит через две подачи, начиная со второй.
При счёте 6:0 сначала будет ещё одна ваша подача (один матч-пойнт), потом две подачи Роджера (ещё два матч-пойнта), снова две ваши (+2) --- и даже если в этот момент уже будет 6:5, то следующая подача Роджера всё равно будет шестым подряд матч-пойнтом.\pr
\end{addedbytheeditors}

\subsubsection*{Макаронные циклы}

Эта старинная задача пришла ко мне от коллеги из Дартмутского колледжа, Даны Уильямса.
Нужно вычислить вероятность создания цикла на каждом соединении концов.
Тогда, из \emph{линейности матожидания}, можно заключить, что ожидаемое число циклов это сумма полученных вероятностей.

При соединении $i$-го конца берётся конец цепи, и из оставшихся $101 - 2i$ концов лишь один из них (противоположный конец этой цепи) приводит к циклу.
Следовательно, вероятность того, что ваше $i$-е соединение добавит цикл, равна $1/(101 - 2i)$, поэтому ожидаемое общее число циклов равно 
\[1/99 + 1/97 + 1/95 +\dots + 1/3 + 1/1 = 2{,}93777485\dots\]
--- меньше трёх циклов!

Если у нас $n$ макаронин и $n$ большое, то матожидание числа циклов близко к половине $n$-го гармонического числа --- примерно половина натурального логарифма $n$.

\begin{addedbytheeditors}
В книге Мартина Гарднера \cite[p. 198]{26} описан другой вариант этой задачи, связанный со следующим гаданием; см. также \cite{meshalkin,bavrin-fribus}.
Девушка зажимает в руке шесть длинных травинок, её подружка наугад попарно связывает сначала верхние, а потом нижние концы.
Если все шесть травинок окажутся связанными в кольцо, то девушка, связавшая их, в этом году непременно выйдет замуж.
В версии Гарднера спрашивалась вероятность такого исхода.
Это, конечно, другой вопрос, но логика решения --- та же самая. \pr
    
\end{addedbytheeditors}

\subsubsection*{Рулетка для ротозеев}

Я услышал эту историю от Элвина Берлекэмпа на конференции «Gathering for Gardner VII».
Позже она появилась в замечательном разделе головоломок журнала Emissary \cite[весна/осень 2006 года]{3}.

Игра в рулетку очень выгодна для казино (американский вариант ещё выгодней европейского, в котором нет двойного зеро).
Ясно, что если повторять невыгодную ставку достаточно долго, то скорее всего окажешься в проигрыше.
Средний убыток каждой однодолларовой ставки составляет $1 - (1/38) \times 36 = 1/19$ доллара, то есть примерно 5~центов.

Однако 105 долларов это не так уж много!
Элвину достаточно выиграть три раза, чтобы оказаться в плюсе.
В этом случае он получит 108 долларов за свои 105.
Вероятность того, что он никогда не выиграет, составляет $(37/38)^{105} \sim 0{,}0608$;
выиграет ровно раз, $105 \times (1/38) \times (37/38)^{104} \z\sim 0{,}1725$;
и два раза, $105 \times (104/2) \times (1/38)^2 \times (37/38)^{103} \sim 0{,}2425$.
Таким образом, вероятность оказаться в плюсе равна единице минус сумма этих трёх значений, то есть $0{,}5242$ --- больше половины!

Конечно же? это не значит, что Элвин может дурачить Лас-Вегас.
Ведь когда ему \emph{не} удастся получить три победы (а это случится приблизительно в 48\% случаев!), он потеряет как минимум 33 доллара, то есть намного больше, чем получит при трёх выигрышах.
\emph{В среднем} Элвин потеряет $105 \times (1/19) \sim 5{,}53$ долларов.

Рассмотрим более жёсткий вариант этой задачи.
Предположим, что у Элвина есть 255 долларов (а ему нужно 256 для регистрации на конференции).
Тогда лучше всего сделать ставку в 1, затем 2, затем 4, 8, 16, 32, 64 и, наконец, 128 долларов на красное (или чёрное).
Первый раз, когда он выигрывает, он получает в два раза больше своей ставки и прекращает игру с 256 долларами, ровно то, что ему нужно.
Он потерпит неудачу только если проиграет все свои $8$ ставок (и все свои деньги), что произойдёт с вероятностью всего $(20/38)^8 < 0{,}006$.

Проделайте это сами, если не страшно потерять 255 долларов.
Так можно посетить казино и в 99\% случаев остаться в плюсе.
Ну а потом лучше бросить играть в азартные игры, настоятельно рекомендую.
